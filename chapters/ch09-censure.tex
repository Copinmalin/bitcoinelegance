% Copyright (c) 2022 Ludovic Lars
% This work is licensed under the CC BY-NC-SA 4.0 International License

\chapter{La résistance à la censure}
\label{ch:censure}

L'un des problèmes de notre époque est la censure financière. Avec le développement de l'économie mondialisée (reposant notamment sur Internet), le recours aux intermédiaires financiers est de plus en plus courant. Cette évolution fait que le problème devient aujourd'hui un problème général, expérimenté par une part grandissante de la population.

Bitcoin constitue une solution à ce problème. L'une des caractéristiques primordiales de Bitcoin est en effet sa résistance à la censure, c'est-à-dire le fait qu'il est difficile pour une entité quelconque d'empêcher la réalisation d'un paiement. En permettant «~aux paiements en ligne d'être envoyés directement d'une partie à l'autre sans passer par une institution financière~», il contourne l'arsenal de contrôles financiers qui gangrènent nos moyens d'épargne et de paiement modernes.

La résistance à la censure est, comme la confirmation des transactions, un mécanisme économique. Elle se fonde de manière essentielle sur la preuve de travail telle qu'elle est utilisée dans l'algorithme de consensus de Nakamoto. De ce fait, les alternatives proposées comme les algorithmes de preuve d'enjeu montrent une résistance à la censure bien plus faibles.

Dans ce chapitre, nous verrons d'abord comment la censure financière est mise ne place dans le monde bancaire aujourd'hui et pourquoi elle devrait constituer un problème de plus en plus général à l'avenir avec le déploiement des MNBC. Puis, nous décrirons comment la censure peut s'exercer dans Bitcoin et comment le système peut y résister. Nous dirons enfin pourquoi les propositions alternatives ne suffisent pas.

\section*{Qu'est-ce que la censure financière~?}
\addcontentsline{toc}{section}{Qu'est-ce que la censure financière~?}

% https://www.universalis.fr/encyclopedie/censure/
%
% Censure (sens étroit) = restriction
% Censure (sens large) = manifestation spécifique de l'autorité qui vient entraver un discours ou une action
% Censure financière = inhibition de l'expression ou de l'action d'une entité par la restriction directe de son activité financière

% Description introductive de la censure
La notion de censure peut paraître étrange de prime abord quand on parle de monnaie. Au sens courant, la censure désigne la restriction de l'expression, notamment par l'interdiction de la diffusion de certaines idées. Néanmoins, il est possible de la comprendre dans un sens plus large, qui mêle le paiement et l'expression.

% Origine du terme, insitution des censeurs à Rome
Le terme de censure vient du latin \emph{cēnseō} signifiant «~évaluer, estimer, déclarer, juger~». Il trouve son origine dans une institution importante de la République romaine, celle des censeurs, deux magistrats qui avaient pour charge de procéder au dénombrement des citoyens et de leurs biens (le \emph{census}), de collecter les impôts, de superviser les travaux publics, de gérer la liste des personnes admises au Sénat (l'\emph{album senatorium}) et de veiller au maintien des «~bonnes mœurs~» de la population en administrant des blâmes ou des peines temporaires. La première fonction des censeurs a donné sa signification au mot recensement. La seconde aux concepts de cens et de suffrage censitaire. Et la dernière a été à l'origine de ce que nous appelons la censure.

% Censure religieuse
Au Moyen Âge, le mot latin \emph{censura} a été repris par le catholicisme pour prendre un sens religieux et se limiter ainsi au discours, et en particulier aux textes. La censure s'apparentait alors à un blâme (sens encore parfois employé, notamment en matière de critique littéraire) ou à une interdiction. Elle se caractérisait par la relecture et la correction des ouvrages rédigés pour s'assurer que tout était conforme au dogme de l'Église catholique romaine.

% Censure politique
Néanmoins, l'apparition de l'imprimerie au \textsc{xv}\ieme{}~siècle a bouleversé les choses~: le nombre de livres a explosé, et ce faisant, a retiré le contrôle que la hiérarchie catholique avait sur la publication des écrits, contrôle qui a été transféré à l'État. La censure a par conséquent acquis son sens politique actuel, en désignant l'examen que le pouvoir étatique fait préalablement des livres, journaux, pièces de théâtre,~etc., pour en permettre ou en prohiber la publication ou la représentation. Par la suite, le terme a fini par nommer toute atteinte à la liberté d'expression, quel que soit le support, que cela se fasse avant (censure a priori) ou après la diffusion (censure a posteriori).

% Censure privée
Avec le développement des médias de masse (journaux, radio, télévision) et surtout des médias sociaux, le terme a acquis un sens élargi et on s'est mis à parler de censure pour tout choix d'édition pris par une entité privée vis-à-vis de ses clients ou de ses utilisateurs. Cette censure privée n'est pas une atteinte à la liberté d'expression au sens strict, mais elle pose problème lorsque le domaine est monopolisé par un petit nombre d'acteurs bénéficiant souvent d'un avantage légal ou d'une subvention étatique. De plus, cette censure peut être directement l'émanation d'une intervention politique, la plateforme en question ne faisant qu'appliquer les directives générales du pouvoir\sendnote{Cet état de fait a été illustré par les Twitter Files, publiés à partir de décembre 2022, suite à la prise de possession de la plateforme de microblogage Twitter par Elon Musk, qui ont révélé les manœuvres internes et l'intervention de l'État fédéral des États-Unis dans la politique de censure de la plateforme. Le FBI a notamment reconnu la véracité de cette situation~: «~La correspondance entre le FBI et Twitter ne montre rien d'autre que des exemples de nos engagements traditionnels, de longue date et toujours en cours, entre le gouvernement fédéral et le secteur privé, qui impliquent de nombreuses entreprises dans de multiples secteurs et industries. Comme le montre la correspondance, le FBI fournit des informations essentielles au secteur privé afin de lui permettre de se protéger et de protéger ses clients. Les hommes et les femmes du FBI travaillent chaque jour pour protéger le public américain. Il est regrettable que des théoriciens du complot et d'autres personnes alimentent la désinformation auprès du public américain dans le seul but de tenter de discréditer l'agence.~» (Evan Perez, Donie O'Sullivan, Brian Fung, \eng{No directive: FBI agents, tech executives deny government ordered Twitter to suppress Hunter Biden story}, 23 décembre 2022~: \url{https://edition.cnn.com/2022/12/23/politics/twitter-files-elon-musk-fbi-hunter-biden-laptop/index.html}).} % "The correspondence between the FBI and Twitter show nothing more than examples of our traditional, longstanding and ongoing federal government and private sector engagements, which involve numerous companies over multiple sectors and industries. As evidenced in the correspondence, the FBI provides critical information to the private sector in an effort to allow them to protect themselves and their customers. The men and women of the FBI work every day to protect the American public. It is unfortunate that conspiracy theorists and others are feeding the American public misinformation with the sole purpose of attempting to discredit the agency."

% Censure financière (sens étroit)
Cependant, cette censure de l'expression peut également être réalisée par l'atteinte de l'activité économique de celui qui s'exprime. En effet, en restreignant la capacité à gagner de l'argent d'une personne et en lui faisant comprendre que son discours pose problème, on peut l'amener à taire ce discours. C'est dans ce contexte qu'a émergé le concept de censure financière, ou \eng{financial censorship} en anglais, que l'organisation internationale \eng{Students for Liberty} définit comme le fait de «~restreindre l'activité financière d'une entité privée, de manière à inhiber ses opérations, avec l'intention implicite de la réduire au silence\sendnote{Students for Liberty, \eng{Financial Censorship}~: \url{https://studentsforliberty.org/blog/freedom-of-expression/financial-censorship/}.}~». C'est aussi le sens que lui donne l'\eng{Electronic Frontier Foundation}\sendnote{Electronic Frontier Foundation, \eng{Financial Censorship}~: \url{https://www.eff.org/issues/financial-censorship}.}. % Financial censorship involves the restriction of a private entity's financial activity, in such a way as to inhibit their operations, with the implicit intention of rendering them silent.

% Censure financière (sens large)
Mais les répercussions du contrôle financier ne s'arrêtent pas à l'expression et peuvent concerner l'action humaine en général. Ainsi, la censure financière peut se saisir dans un sens plus large, une signification adoptée par exemple par trois chercheurs de l'université d'État de San José (Marco Pagani, George Whaley et David Czerwinski) qui affirment que «~la censure financière se produit lorsqu'une institution financière refuse ses services à une partie en raison des opinions exprimées, des actions ou du secteur d'activité de cette partie\sendnote{Marco Pagani, George Whaley, David Czerwinski, \eng{Frameworks for Assessing Financial Censorship and Its Implications}, 2022~: \url{https://articlegateway.com/index.php/JAF/article/download/4989/4759}.}~». % Financial censorship occurs when a financial institution denies its services to a party because of that party's expressed views, actions, or line of business.

% Censure comme restriction financière arbitraire
Enfin, on peut comprendre la censure financière comme la restriction financière elle-même à condition qu'elle repose sur un critère subjectif externe (respect de normes arbitraires) et non pas sur une donnée économique objective, comme par exemple le paiement d'une commission. Cette censure peut être appliquée de manière publique (interdiction légale d'une transaction), privée (par une banque par exemple) ou les deux. Cette définition conserve toujours en elle l'idée de modeler le comportement extérieur de la personne par l'intervention sur ses finances. C'est notamment ce sens qui est donné à la censure dans Bitcoin.

% Définition générale
Au sens général, la censure financière consiste donc à restreindre directement l'activité financière d'une entité de façon à inhiber son expression ou son action. L'idée est d'influencer l'individu par le contrôle sur la monnaie dont il se sert, un outil qui est essentiel à sa survie économique. Aujourd'hui, la censure s'applique essentiellement au crédit bancaire, dont les transferts sont hautement réglementés par le pouvoir. Demain, elle pourra concerner la monnaie numérique gérée par la banque centrale.

\section*{La banque et la censure}
\addcontentsline{toc}{section}{La banque et la censure}

% Censure financière avec l'argent liquide
La censure financière s'exerce par le contrôle sur le transfert de monnaie, de sorte qu'elle peut difficilement être mise en place au travers de l'argent liquide physique. En effet, celui-ci (qu'il prenne la forme de pièces de métal précieux ou de billets fiduciaires) permet l'échange direct et confidentiel de personne à personne, ce qui empêche l'application de toute restriction en dehors de quelques cas particuliers.

% Censure financière avec un compte bancaire
En revanche, dans le domaine bancaire, le client dispose d'un compte courant sur lequel la banque le crédite un montant et gère les transferts. La restriction financière est de ce fait beaucoup plus simple~: la banque peut sélectionner les transferts, geler le compte momentanément et même refuser le retrait d'argent.
% C'est également le cas des systèmes de paiement numériques comme Paypal.

% Accroissement
C'est donc tout naturellement que l'accroissement de la censure financière a coïncidé avec la bancarisation de la société, qui a eu lieu à partir des années 1960 en Occident\sendnote{\url{https://books.openedition.org/pur/121053?lang=fr}~; \url{https://www.the-american-interest.com/2019/02/25/bigger-fewer-riskier-the-evolution-of-u-s-banking-since-1950/}}, et qui s'est caractérisée par la généralisation de l'usage du compte courant et des moyens de paiement apparentés comme le chèque bancaire, la carte de crédit et le virement. En quelques décennies, le paiement a migré vers le domaine bancaire (légalement favorisé) bien plus commode à utiliser, en parallèle de la restriction de plus en plus grande appliquée au paiement en espèces (légalement défavorisé). D'où la meilleure efficacité de la censure~: si le liquide ne permet plus de gérer ses affaires convenablement, alors la possibilité de se retirer complètement du système bancaire n'est plus une option viable.

% Surveillance financière
Cette censure s'est mise en place par l'intermédiaire de la surveillance financière, qui est aujourd'hui particulièrement fréquente dans l'industrie bancaire. Les banques ont en effet tendance à surveiller leurs clients et à intervenir dans le cas où elles constatent un comportement «~suspect~» de leur part, en empêchant leurs virements ou en gelant leurs comptes. Mais elles ne font pas cela de gaieté de cœur. Elles ne procèdent pas à la surveillance de leurs clients pour les «~protéger~», mais pour se protéger elles-mêmes contre les complications légales qui pourraient suivre le traitement d'un virement non autorisé par les réglementations.

% Lutte contre le blanchiment d'argent : guerre contre la drogue
Cette réglementation s'est développée au cours des années, à mesure que l'activité bancaire se popularisait. À partir des années 70, le prétexte de la lutte contre le blanchiment d'argent (notamment dans le cadre de la guerre contre la drogue) s'est imposé comme le principal prétexte derrière les restrictions imposées au banques. Aux États-Unis notamment, la réglementation bancaire s'est particulièrement durcie suite à l'adoption du \eng{Bank Secrecy Act} de 1970, qui se proposait de lutter contre le blanchiment d'argent.

% Lutte contre le blanchiment d'argent : web
Puis, avec l'apparition du web dans les années 1990, l'utilisation des banques internationales a demandé une réglementation accrue. Différents organismes de surveillance ont ainsi été créés. Le Groupe d'action financière (GAFI), un organisme intergouvernemental émettant régulièrement des recommandations de normes réglementaires et de sanctions économiques, a été créé en juillet 1989 dans le but de lutter contre le blanchiment d'argent. Le Financial Crimes Enforcement Network (FinCEN), le bureau du département du Trésor des États-Unis qui collecte et analyse les informations sur les transactions financières, a été formé dans ce sens le 25 avril 1990. L'équivalent français, la cellule TRACFIN (Traitement du renseignement et action contre les circuits financiers clandestins), est apparu en juillet 1990. Du côté européen, la première directive de l'Union Européenne relative à la prévention de l'utilisation du système financier aux fins du blanchiment de capitaux est datée du 10 juin 1990\sendnote{\url{https://eur-lex.europa.eu/legal-content/FR/TXT/PDF/?uri=CELEX:31991L0308&from=FR}}.

% Lutte contre le financement du terrorisme
Enfin, après les attentats islamistes du 11 septembre 2001, un autre prétexte est apparu~: la lutte contre le financement du terrorisme. Celle-ci s'est matérialisée aux États-Unis par l'adoption du \eng{PATRIOT Act} en octobre 2001, dont le Titre 3 concerne les restrictions financières. En France, la loi du 15 novembre 2001 relative à la sécurité quotidienne a requalifié «~le fait de financer une entreprise terroriste~» comme un acte de terrorisme en lui-même\sendnote{Code pénal, Article 421-2-2, 15 novembre 2001~: \url{https://www.legifrance.gouv.fr/codes/article_lc/LEGIARTI000006418433}.}. La surveillance financière s'est renforcée en conséquence.

% LCB-FT
Ces deux évolutions forment la base de ce qu'on appelle généralement la lutte contre le blanchiment des capitaux et le financement du terrorisme (LCB-FT) en France et les normes AMF/CFT (pour \eng{Anti-Money Laundering/Combating the Financing of Terrorism}) aux États-Unis.

% Connaissance du client
Ce resserrement se caractérise notamment par la connaissance du client (\eng{Know Your Customer} ou KYC), une pratique également appelée vigilance à l'égard de la clientèle, qui consiste à vérifier l'identité, la conformité et les risques liés à chaque client. Cette exigence d'identification s'est insérée dans tous les services financiers aujourd'hui.

% Fin du secret bancaire
En conséquence, le secret bancaire, c'est-à-dire l'obligation pour les banques de ne pas livrer des informations sur leurs clients à des tiers, a fini par disparaître\sendnote{Anthony Amicelle et Jean Bérard, \eng{Vers la fin du secret bancaire ou de la vie privée ?}, 2019~: \url{https://journals.openedition.org/conflits/21291}.}, y compris en Suisse\sendnote{Mathilde Damgé, \eng{Comment la Suisse a renoncé au secret bancaire}, 11 février 2015~: \url{https://www.lemonde.fr/evasion-fiscale/article/2015/02/11/comment-la-suisse-a-renonce-au-secret-bancaire_4572485_4862750.html}}. L'usage d'un compte bancaire aujourd'hui présuppose la surveillance générale des transactions et l'inspection minutieuses des opérations les moins usuelles. Ainsi, il est aujourd'hui impossible de virer une importante somme d'argent d'un compte à un autre sans devoir fournir une justification.

% Résumé par Jonathan Thornbug
Cette situation du domaine financier a été résumée en janvier 2009 par Jonathan Thornbug sur la liste de diffusion en réponse à Satoshi Nakamoto qui décrivait les utilisations qu'on pouvait faire de Bitcoin~:

\begin{quote}
«~Dans le monde moderne, aucun État important ne veut autoriser des transactions financières internationales intraçables au-delà d'un certain seuil de taille assez modeste. (Les phrases d'accroche habituelles sont des choses comme "blanchiment de l'argent de la drogue", "évasion fiscale", et/ou "financement de groupes terroristes"). À cette fin, les transactions financières électroniques sont actuellement surveillées par divers États et leurs agences, et toutes les transactions, sauf les plus petites, sont désormais assorties de diverses exigences en matière d'identification des personnes à chaque extrémité.\sendnote{Jonathan Thornbug, \eng{Bitcoin v0.1 released}, \wtime{17/01/2009 16:49:45 UTC}~: \url{https://www.metzdowd.com/pipermail/cryptography/2009-January/015016.html}.}~»
\end{quote} % "In the modern world, no major government wants to allow untracable international financial transactions above some fairly modest size thresholds.  (The usual catch-phrases are things like "laundering drug money", "tax evasion", and/or "financing terrorist groups".) To this end, electronic financial transactions are currently monitored by various governments & their agencies, and any but the smallest of transactions now come with various ID requirements for the humans on each end."

\section*{Les cas de censure financière}
\addcontentsline{toc}{section}{Les cas de censure financière}

% Comme on l'a dit, la censure financière peut concerner l'expression, l'action ou l'activité professionnelle d'une personne. Il s'agit donc d'appliquer la censure dans le but d'entraver une action dérangeante, comme un discours dissonant, une opposition politique non tolérée, un comportement inapproprié, une profession jugée immorale,~etc. Si vous dépendez d'un compte bancaire, vous devez vous taire sur certains sujets, ne pas vous engager dans certaines actions politiques et ne pas exercer certaines professions, auquel cas vous pourriez être punis indirectement.

Au cours des dernières années, les cas célèbres de censure financière manifeste se sont multipliés, à tel point qu'il est impossible d'en faire une liste exhaustive. Citons-en quelques uns des plus connus en Occident.

% Nous nous bornerons ici à évoquer le système financier occidental, et notamment celui des États-Unis. Le pays américain est censé être le pays le plus libre au monde et donne une bonne idée de la progression de cette censure.

% WikiLeaks
L'exemple le plus cité est probablement le blocus financier contre WikiLeaks mis en place par Mastercard, Visa, Western Union, Bank of America et d'autres acteurs, en décembre 2010, dans le but de faire taire l'organisation. En juin 2011, un communiqué de WikiLeaks a indiqué que le blocus financier avait fait disparaître de ses 95~\% des revenus\sendnote{WikiLeaks\eng{Banking Blockade}, \wtime{24 octobre 2011, 13:00 UTC}~: \url{https://wikileaks.org/Banking-Blockade.html}.}.

% Opération Choke Point
Un autre cas, qui visait cette fois la profession des victimes, est l'opération Choke Point mise en place entre 2013 et 2017 par le département de la Justice des États-Unis\sendnote{\url{https://en.wikipedia.org/wiki/Operation_Choke_Point}, \url{https://www.wsj.com/articles/SB10001424127887323838204578654411043000772}, archive~: \url{https://archive.is/bF8KZ}}. Cette opération avait pour but d'«~étouffer~» certains secteurs d'activité en restreignant leur accès au crédit et à d'autres services bancaires. Les activités jugées «~à haut risque~» incluaient le prêt sur gages ou sur salaire, le jeu d'argent, la pornographie, l'escorting, mais aussi la vente de tabac et de produits pharmaceutiques, la vente de pièces de monnaie, les services de rencontre ou encore l'organisation des clubs de voyage. La vente d'armes à feu et de munitions était aussi concernée~: l'entreprise de Cody Wilson, appelée Defense Distributed et spécialisée dans la diffusion de schémas de conception d'armes à feu fabriquées par imprimante 3D, en a fait les frais et a vu son compte bancaire chez Chase être fermé et ses comptes PayPal et Stripe bloqués\sendnote{Kelsey Bolar, \eng{Firearms Sellers Say They’re Being Choked Off From Payment Processors}, 12 janvier 2015~: \url{https://www.dailysignal.com/2015/01/12/firearms-sellers-say-theyre-choked-off-payment-processors/}.}.

% La censure financière concerne également l'opinion politique, qu'elle contribue par là à modeler. Ainsi, les activistes politiques dont les positions sont jugées extrêmes par les élites ont régulièrement des problèmes avec les différents services financiers qu'ils utilisent.

% Extrême-droite étasunienne (InfoWars)
En 2018, c'est l'opinion politique qui a fait les frais de la censure. On a ainsi vu de nombreuses personnalités et organisations d'extrême-droite être bannies des divers réseaux sociaux et perdre leur accès à divers services financiers. La plus emblématique d'entre elles est Alex Jones, fondateur du site de réinformation InfoWars, qui, outre sa purge des médias sociaux durant l'été 2018, a vu son compte PayPal être clôturé\sendnote{\url{https://www.washingtonpost.com/technology/2018/09/21/paypal-bans-alex-jones-saying-infowars-promoted-hate-or-discriminatory-intolerance/}}. On peut aussi citer les cas de Gab.com (média social fondé par ), Milo Yiannopoulos (banni de PayPal pour avoir fait un salut nazi\sendnote{\url{https://www.timesofisrael.com/paypal-suspends-milo-yiannopoulos-over-nazi-based-trolling-of-jewish-journalist/}}) ou encore de Robert Spencer (chroniqueur du blog anti-islam Jihad Watch, exclu de Patreon suite à la pression de Mastercard\sendnote{\url{https://twitter.com/Patreon/status/1029551216886341634}}).

% Des comptes d'organisations liées au mouvement Antifa (Atlanta Antifa, Antifa Sacramento, Anti-Fascist Network) ont également été suspendus de PayPal en novembre 2018\sendnote{\url{https://www.newsweek.com/paypal-cancels-alt-right-antifa-accounts-1209838}}.

% Égalité et Réconciliation
En France, cette censure est incarnée par l'exemple d'Égalité et Réconciliation, l'association de l'antisioniste Alain Soral, généralement classifiée à l'extrême-droite même si elle relève plus du populisme. E\&R a ainsi été exclue de PayPal en août 2018 au cours de la purge générale. De manière générale, l'association a vu plusieurs de ses comptes bancaires (Banque postale, BNP Paribas, Banque populaire) être fermés\sendnote{Égalité et Réconciliation, \eng{Soutenez-nous~: la Banque populaire ferme le compte en banque d'Égalité \& Réconciliation}, 6 février 2022~: \url{https://www.egaliteetreconciliation.fr/Soutenez-nous-la-Banque-populaire-ferme-le-compte-en-banque-d-Egalite-Reconciliation-67155.html}.}.

% Manifestations de Hong Kong (2019-2020)
Toujours dans le domaine politique, mais en Chine cette fois-ci, on peut citer le cas du mouvement contre l'amendement de la loi d'extradition par le gouvernement de Hong Kong, série de manifestations ayant eu lieu entre mars 2019 et juillet 2020, qui a dû subir les interventions du conglomérat bancaire international HSBC, probablement sous pression de l'État central chinois. En novembre 2019, la filiale de Hong Kong a décidé de fermer un compte utilisé pour soutenir le mouvement de protestation~; en décembre 2020, elle gelé le compte du démocrate Ted Hui\sendnote{\url{https://hongkongfp.com/2020/12/07/hsbc-re-freezes-accounts-belonging-to-family-of-exiled-democrat-ted-hui-amid-hong-kong-police-money-laundering-probe/}}~; et elle refuse aujourd'hui aux Hongkongais ayant fui au Royaume-Uni l'accès légitime à leurs fonds de pension\sendnote{\url{https://www.lefigaro.fr/flash-eco/hsbc-complice-de-violation-des-droits-humains-a-hong-kong-selon-un-rapport-parlementaire-20230208}}.

% Covid-19, Viruswaarheid
Plus récemment, la pandémie de Covid-19 a fourni d'autres exemples de censure financière. De nombreux activistes opposés aux mesures coercitives comme le confinement, le port du masque et la vaccination obligatoire, ont ainsi été largement censurés, généralement accusées de propager la désinformation. Le groupe d'action néerlandais Viruswaarheid -- s'opposant à la distanciation sociale, au confinement, au couvre-feu et au programme de vaccination -- a ainsi vu son compte bancaire utilisé pour recevoir des donations être fermé par ING Bank en février 2021\sendnote{Andreas Kouwenhoven et Wilmer Heck, \eng{De complotdenker bankiert maar elders, zegt de bank}, 17 août 2021~: \url{https://www.nrc.nl/nieuws/2021/08/17/de-complotdenker-bankiert-maar-elders-zegt-de-bank-a4055125}~; archive~: \url{https://archive.is/8LI0k}.}.

% Convoi de la liberté, 2022
Mais l'exemple qui ressort du lot est ce qui s'est passé au Canada en février 2022 dans le cadre du mouvement de protestation canadien contre l'obligation vaccinale anti-Covid-19 imposée aux personnes entrant sur le territoire par voie terrestre, couramment appelé le Convoi de la liberté. En réaction à cette restriction introduite le 15 janvier 2022 par le gouvernement, des camionneurs (principaux concernés) ont fait route jusqu'à Ottawa pour s'y rejoindre le 29 janvier et exprimer leur mécontentement à l'égard du pouvoir, notamment par des coups de klaxons répétitifs. Le mouvement était globalement non violent et familial. Plus tard, des blocages ont été organisés, dont notamment celui du pont Ambassadeur, faisant la jonction avec les États-Unis en Détroit et Windsor et cœur névralgique des échanges commerciaux entre les deux pays, sur lequel la circulation a été interrompue entre le 8 et le 13 février. Ces blocages ont servi de prétexte pour l'intervention du gouvernement fédéral canadien de Justin Trudeau, qui a déclaré l'état d'urgence le 14 février en invoquant la Loi sur les mesures d'urgence.

Ce mouvement a dû faire face à une censure financière drastique. D'abord, il a été victime des plateformes de financement participatif, qui ont annulé ces différentes campagnes pour payer le déplacement des camionneurs. La première campagne réalisée sur GoFundMe, ayant recueilli plus de 10 millions de dollars canadiens a ainsi été retirée de la plateforme le 4 février pour cause de «~promotion de la violence et du harcèlement~» suite à la suggestion du maire d'Ottawa\sendnote{Radio-Canada, \eng{La campagne de sociofinancement du convoi des camionneurs retirée de GoFundMe}, 4 février 2022~: \url{https://ici.radio-canada.ca/nouvelle/1859918/retrait-campagne-gofundme-convoi-camionneurs-2022}.}. De même, les fonds recueillis par les campagnes organisées sur la plateforme chrétienne GiveSendGo -- totalisant environ 9 millions de dollars canadiens -- ont pu être gelés par le gouvernement de l'Ontario à partir du 10 février, interdisant par là leur distribution\sendnote{Stephanie Taylor, \eng{Ontario court freezes access to donations for truckers' protest from GiveSendGo}, 10 février 2022~: \url{https://www.ctvnews.ca/canada/ontario-court-freezes-access-to-donations-for-truckers-protest-from-givesendgo-1.5776674}.}.

Après l'entrée en vigueur de l'état d'urgence, le mouvement a dû subir une restriction bancaire. Le gouvernement a en effet décidé de geler les comptes bancaires personnels ou professionnels en lien avec le mouvement~: 280 comptes comptant 8 millions de dollars canadiens au total ont été gelés\sendnote{Bill Curry, Marsha McLeod, \eng{Deputy Minister of Finance describes race against time to prevent economic damage from border blockades}, 17 novembre 2022~: \url{https://www.theglobeandmail.com/politics/article-emergencies-act-inquiry-michael-sabia/}}. L'année suivante, la vice-première ministre du Canada, Chrystia Freeland, a déclaré à la Commission sur l'état d'urgence que le gel des comptes bancaires était un «~outil puissant pour décourager la participation et inciter les manifestants à abandonner\sendnote{\url{https://apnews.com/article/canada-government-justin-trudeau-ottawa-montana-9c1e37aa86d4315703e69f7794637e7f}~; \url{https://twitter.com/TPostMillennial/status/1626686280108068872}.}~».

% Sanctions économiques à l'égard de la Russie
À la fin du mois de février 2022, un autre évènement important est survenu~: l'invasion de l'Ukraine par la Russie. En conséquence, les sanctions économiques\sendnote{Les sanctions économiques internationales concernant le domaine financier rentrent dans la catégorie de la censure financière. Celles-ci ont en effet pour but premier d'empêcher le commerce avec la population gouvernée par un État ennemi. Le cas des Russes n'est pas un cas isolé, et de nombreuses autres populations n'ont pas accès aux services financiers disponibles pour les Occidentaux, comme les Palestiniens par exemple (Electronic Frontier Foundation, \eng{Why Is PayPal Denying Service to Palestinians?}, 12 octobre 2021~: \url{https://www.eff.org/deeplinks/2021/10/why-paypal-denying-service-palestinians}).} imposées par les États occidentaux à l'encontre des cityens russes, déjà en place depuis 2014 et l'annexion de la Crimée, se sont considérablement durcies. En ce qui concerne le domaine financier, certaines banques russes ont été exclues du système SWIFT, tout financement en Russie est prohibé, l'achat de roubles également et la fourniture de «~services de portefeuille, de compte ou de conservation de crypto-actifs~». % Source : https://eur-lex.europa.eu/legal-content/EN/TXT/?uri=CELEX%3A02014R0833-20221217

De manière générale, les virements vers la Russie ont été interdits, de sorte que les citoyens russes exilés ne pouvaient plus envoyer d'argent à leur famille. C'est aussi le cas des Ukrainiens dont les proches sont restés sur le territoire occupé par l'armée russe, comme cette Ukrainienne réfugiée en France qui ne pouvait pas envoyer un virement bancaire de 100~euros à sa famille, dont le compte bancaire était domicilié à Donetsk\sendnote{\url{https://journalducoin.com/analyses/un-an-guerre-ukraine-petite-histoire-valeriia-binance/}}.

% RT France
De côté occidental, les mesures se sont également faites ressentir au niveau financier. La chaîne d'information RT France, alors déjà interdite de diffusion en France, mais qui continuait sur Internet, a vu ses avoirs être gelés le 20 janvier 2023 sur initiative de l'Union européenne, ce qui l'a contraint à annoncer sa fermeture définitive\sendnote{Le Parisien, \emph{RT France, branche française de la chaîne russe, annonce sa fermeture}, 23 janvier 2022~: \url{https://www.leparisien.fr/international/rt-france-branche-francaise-de-la-chaine-russe-annonce-sa-fermeture-21-01-2023-YMOTSTASWZAF3KSGCYHAFFMG6U.php}.}.

% Activités liées aux cryptomonnaies
Enfin, pour finir sur les différents cas de censure financière, on ne peut pas ne pas évoquer les activités liées aux cryptomonnaies, qui subissent la censure générale des banques et autres organismes financiers. L'achat de cryptomonnaies est entravé par les banques qui interdisent régulièrement à leurs clients (toujours en prétendant les «~protéger~») d'envoyer des fonds vers les plateformes d'échange de cryptomonnaie\sendnote{Jean-Luc de Bitcoin.fr, \emph{Les banques et Bitcoin –- Classement de janvier 2023
}, 9 janvier 2023~: \url{https://bitcoin.fr/bitcoin-et-les-banques-classement-de-janvier-2023/}}. De plus, les entreprises du secteur peinent à ouvrir un compte bancaire\sendnote{Dans son livre \emph{Cryptomonnaie~: la nouvelle guerre}, François-Xavier Thoorens explique par exemple comment lui et sa famille ont été expulsés de leur banque familiale après avoir voulu ouvrir un compte professionnel pour recevoir des fonds récupérés lors de l'ICO de Ark (pp. 91 -- 97). Mais son cas est loin d'être une exception.}.

% La censure financière mène à Bitcoin
La censure financière est donc de plus en plus fréquente, et s'appliquent à de nombreuses personnes, de bords politiques différents, de nationalités différentes, de professions différents. Elle se fait bien souvent sans décision juridique spécifique et donne un caractère ésotérique, caché, arbitraire à l'application du pouvoir réel. Et cette censure a pour effet de pousser les gens qui l'expérimentent à s'intéresser à Bitcoin. En effet, l'expérience d'une telle restriction provoque nécessairement le désir de trouver un moyen de la contourner, quand bien même celle-ci serait légère. Lorsque nous prenons pleinement conscience de la censure non plus comme un risque abstrait mais comme une réalité concrète, nous sentons en nous le besoin de nous en libérer et nous prémunir de ce danger à l'avenir. L'épreuve de la censure nous marque à jamais et nous confirme la proposition de valeur de Bitcoin\sendnote{Cet effet de l'expérience de la censure a été décrit par Nick Szabo au micro de Peter McCormack en 2019~: «~Certaines personnes doivent être frappées par la réalité. Si vous êtes censuré par une banque, comme c'est de plus en plus le cas -- et c'est d'ailleurs l'un des risques de la centralisation numérique, c'est que les gens soient censurés et les activistes politiques de différents bords commencent à découvrir qu'on peut aller voir les banques et faire taire ses ennemis politiques et les gens qui font des choses qu'on ne veut pas qu'ils fassent, on les fait taire. On n'a pas nécessairement besoin de faire passer une loi, on peut convaincre certains régulateurs ou certains politiciens, et puis ils mettent la pression sur les banques, et boum~: c'est notre loi de facto juste là. Ça se produit de plus en plus souvent parce que la centralisation numérique rend les choses si vulnérables à ça. Il s'agit donc d'une autre tendance et tout dépend de la vitesse à laquelle elle se développe, car à chaque fois que quelqu'un est censuré, boum~: c'est une réalité qui s'impose à lui et il devient fan de Bitcoin.~» (What Bitcoin Did Podcast, \eng{Nick Szabo on Cypherpunks, Money and Bitcoin}, 1\ier{} novembre 2019~: \url{https://www.whatbitcoindid.com/podcast/nick-szabo-on-cypherpunks-money-and-bitcoin})}. C'est le cas de l'auteur de ce livre qui a vu son compte bancaire être gelé sans préavis, sans que la banque ne mentionne la raison derrière cette suspension, et qui n'a pu récupérer ses fonds que six mois plus tard\sendnote{Ludovic Lars, Twitter, \wtime{15/02/2022 10:42 UTC}~: \url{https://twitter.com/lugaxker/status/1493536121678147586}.}.

% "Well some people have to be hitting over the head with reality. If you're censored via bank, as people increasingly are, and by the way, that's one of the risks of digital centralization, is people get censored and the political activists from various points of view are starting to discover that you can go to banks and get your political enemies shut down and people doing things you don't want them to do, you get them shut down. You don't necessarily need to pass a law, you can convince some regulators or convince some politicians and then they lean on the banks and boom, that's your de facto law right there. That's increasingly happening because digital centralization makes things so vulnerable to that. So that's another trend and it depends how fast that grows because every time somebody gets censored, boom, that's a reality over their head, they become a Bitcoin fan."

\section*{Censure et monnaie numérique de banque centrale}
\addcontentsline{toc}{section}{Censure et monnaie numérique de banque centrale}

% Monnaies numériques de banque centrale

% Si la censure financière constitue aujourd'hui une complication occasionnelle.
La tendance est donc claire~: avec l'utilisation intensive des comptes bancaires en lieu et place des espèces, le pouvoir de censure financière est devenu de plus en plus prédominant. Ainsi, même si cette censure reste occasionnelle, nous pouvons nous attendre à ce qu'elle constitue un problème grandissant. Plus précisément, elle pourrait devenir une contrainte générale dans les décennies à venir avec le déploiement progressif des monnaies numériques de banque centrale (MNBC) et la disparition conjointe de l'argent liquide.

Comme on l'a dit précédemment\sendnote{Voir la section sur les monnaies numériques de banque centrale dans le chapitre~\ref{ch:adversaire}.}, la numérisation de la monnaie constitue la prochaine étape dans l'évolution de la monnaie étatique. Depuis 2016, les banques centrales autour du monde s'évertuent à concevoir des systèmes qui pourraient être utilisés par le grand public et les communications à ce sujet se multiplient depuis 2020.

% Monnaie numérique : seigneuriage + contrôle financier
Une telle monnaie numérique permettraient de récupérer un revenu de seigneuriage supplémentaire en supprimant le coût de la production l'argent liquide remplacé et en récupérant une part de l'activité monétaire qui a lieu aujourd'hui par l'intermédiaire du crédit bancaire. Mais elle permettrait aussi (ce qui nous intéresse ici) d'exercer un contrôle financier total sur les transactions des citoyens en centralisant la gestion du système entre les mains de la banque centrale et des organismes agréés.

% Surveillance financière
Ce contrôle s'accompagnerait bien entendu d'une surveillance financière accrue, qui serait justifiée par les mêmes prétextes utilisés aujourd'hui, comme la lutte contre le blanchiment d'argent et le financement du terrorisme. Ceci pourrait conduire à l'instauration d'un système panoptique, où la surveillance se ferait à l'insu du surveillé\sendnote{Le panoptique (en anglais, \eng{panopticon}) est un type d'architecture carcérale imaginée par le philosophe utilitariste Jeremy Bentham et son frère Samuel à la fin du \textsc{xviii}\ieme{}~siècle. L'objectif de la structure panoptique est de permettre à un gardien, logé dans une tour centrale, d'observer tous les prisonniers, enfermés dans des cellules individuelles autour de la tour, sans que ceux-ci puissent savoir s'ils sont observés.}. Les banques centrales nient vouloir aller dans cette direction, mais le fait est qu'elles ne rendront jamais leurs systèmes strictement confidentiels réservant toujours un droit de regard aux autorités compétentes.

% Disparition progressive de l'argent liquide
Cette surveillance financière pourrait être affermie par la disparition progressive de l'argent liquide, qui a déjà commencé à certains endroits du monde. C'est le cas de la Suède, où la question de la disparition de l'argent liquide est déjà discutée et où l'État fait tout pour mettre à disposition des moyens de paiement numérique innovants\sendnote{sweden.se, \eng{A cashless society}~: \url{https://sweden.se/life/society/a-cashless-society}.}. C'est aussi le cas de la Chine, où l'essentiel des transferts se font par l'intermédiaire de systèmes de paiement mobile comme Alipay ou WeChat Pay, et qui a également lancé son programme pilote de MNBC. Ce n'est pas un hasard si ces deux pays ont été les premiers à envisager sérieusement de développer une monnaie numérique.

% Démonétisation des espèces en Inde
La guerre contre l'argent liquide sévit déjà dans certains pays au travers de démonétisation de certains billets en circulation. En Inde en novembre 2016, le gouvernement de Narendra Modi a décidé de démonétiser les billets de 500 et 1000 roupies, équivalant à 7,5 et 15~\$, et représentant à eux seuls 86~\% de la monnaie en circulation. La justification de cette mesure était la lutte contre la contrefaçon de faux billets, l'évasion fiscale et l'économie informelle. Les billets pouvaient être échangés contre d'autres billets ou être déposés sur un compte en banque, mais seulement à condition que le porteur puisse attester de la provenance des sommes\sendnote{Ninon Renaud, Michel De Grandi, \emph{En Inde, la démonétisation des grosses coupures provoque l'émoi}, 13 novembre 2016~: \url{https://www.lesechos.fr/2016/11/en-inde-la-demonetisation-des-grosses-coupures-provoque-lemoi-216048}.}.

% Démonétisation des espèces au Nigéria
Au Nigéria, le gouvernement a tenté d'appliquer une mesure similaire début 2023, par la limitation des retraits et la démonétisation des grosses coupures, dans le but de contrôler l'inflation, de lutter contre la contrefaçon et de promouvoir de la naïra électronique (eNaira) lancée par la banque centrale en octobre 2021\sendnote{Simi Jolaoso, \eng{Nigeria's naira shortage: Anger and chaos outside banks}, 14 février 2023~: \url{https://www.bbc.com/news/world-africa-64626127}.}. Celle-ci a cependant donné lieu à une pénurie de billets dans le pays, menant la Cour suprême du Nigéria à retarder l'application de cette décision\sendnote{VOA Afrique, \emph{Nigeria: la Cour suprême prolonge la validité des anciens billets de nairas}, 3 mars 2023~: \url{https://www.voaafrique.com/a/pénurie-d-argent-au-nigeria-la-cour-suprême-prolonge-la-validité-des-anciens-billets/6988471.html}.}.

% Pratique de la démonétisation
La pratique de la démonétisation n'est pas nouvelle puisqu'elle avait été utilisée en Europe après la Seconde Guerre mondiale pour enrayer les effets inflationnistes du faux-monnayage et pour détruire les profits du marché noir, ce qui avait fait d'ailleurs dire au personnage du Dabe dans \emph{Le cave se rebiffe} qu'«~en matière de monnaie, les États ont tous les droits et les particuliers aucun~!\sendnote{Le Dabe à Charles dans \emph{Le cave se rebiffe} (1961).}~».

% Généralisation de la monnaie numérique et programmabilité
Une fois que la monnaie numérique mise en place et l'argent liquide largement limité, les gens respectueux de la loi n'auraient d'autre choix que d'utiliser ce système surveillé. Le système pourrait déterminer le montant que vous dépensez, pour quoi vous l'utilisez, avec qui vous commercez. De plus, en tant que système informatique, il pourrait être facilement programmé de façon à imposer des conditions de dépense pour chaque montant de monnaie possédé par l'utilisateur. Une telle programmabilité permettrait permettra aux autorités en charge d'orienter le comportement politique, économique et moral des individus dans le sens désiré, ce qui donnerait à la censure financière une portée comme jamais vue auparavant.

% Comportement économique : transmission de la politique monétaire
Au niveau économique d'abord, cela permettrait d'améliorer ce que les banquiers centraux appellent la transmission de la politique monétaire, c'est-à-dire le processus par lequel les décisions de politique monétaire affectent l'économie en général et le niveau des prix en particulier. Aujourd'hui cette transmission est essentiellement assuré par la modification des taux d'intérêt directeurs. Demain, elle pourrait se faire par la programmation de la monnaie. Cela permettrait notamment de transformer le système d'aides sociales en un système de subvention direct qui imposerait la dépense rapide dans un domaine économique précis, dans le but de le stimuler.

% Comportement moral : mœurs, écologie
Ensuite au niveau moral, cela permettrait d'orienter les paroles et les actions des gens dans un sens particulier. Les mœurs nous viennent en tête, conformément à la tâche allouée aux censeurs de la Rome antique. Mais dans notre société moderne, cela pourrait être aussi fait dans le cadre de la lutte contre le changement climatique, en récompensant le comportement «~écologique~», comme la location d'un vélo pour se déplacer, et en punissant l'attitude «~pollueuse~», comme la consommation de viande. Cette possibilité fait entrevoir l'instauration d'un système de crédit social à la chinoise.

% Comportement politique : disparition du cadre légal, violation de l'État de droit
Enfin d'un point de vue politique, ce système permettrait de réduire l'opposition au pouvoir en sanctionnant ceux qui pensent mal, ceux qui manifestent contre, etc. Le pouvoir politique pourrait raffermir sa position en appliquant les interventions, non plus de manière publique et légale (conformément à l'idée d'État de droit au sens de \eng{Rechtsstaat}), mais de façon cachée et discrétionnaire. Cela pourrait constituer les prémices d'un régime totalitaire où l'État saurait tout, contrôlerait tout, et où il n'y aurait plus besoin de lois\sendnote{«~Ce qu'il allait commencer, c'était son journal. Ce n'était pas illégal (rien n'était illégal, puisqu'il n'y avait plus de lois), mais s'il était découvert, il serait, sans aucun doute, puni de mort ou de vingt-cinq ans au moins de travaux forcés dans un camp.~» -- George Orwell, \eng{1984}, 1949.}. La MNBC serait un outil puissant de surveillance financière de masse, pouvant œuvrer à la réalisation d'un avenir orwellien dans lequel les individus n'auront plus aucune vie privée et plus aucun pouvoir de résistance à l'autorité.

% Gestion par une intelligence artificielle
Cette censure financière ayant lieu à une échelle jamais vue auparavant. Par conséquent, il serait difficile de la mettre en place par une gestion manuelle des êtres humains. C'est pour cela qu'elle serait probablement déléguée à un algorithme doté d'une intelligence artificielle, qui détecterait les mauvais paiements et les bloquerait instantanément. Le système de MNBC pourrait ainsi nous mener à une situation qui rappellerait celle décrite par Saint Jean dans son Apocalypse~:

\begin{quote}
«~Par ses manœuvres, tous, petits et grands, riches ou pauvres, libres et esclaves, se feront marquer sur la main droite et sur le front, et nul ne pourra rien acheter ni vendre s'il n'est pas marqué au nom de la Bête ou au chiffre de son nom.\sendnote{Ap \wtime{13:16-17}.}~»
\end{quote}

Dans ce monde dystopique dont nous pouvons à peine imaginer les ramifications, l'espoir serait représenté par Bitcoin, dont l'une des deux caractéristiques fondamentales est la résistance à la censure. Notre intérêt est donc de comprendre comment la censure pourrait s'exercer dans le système et comment cette résistance pourrait intervenir.

\section*{La censure dans Bitcoin}
\addcontentsline{toc}{section}{La censure dans Bitcoin}

% Bitcoin n'est pas incensurable
Si Bitcoin est réputé résistant à la censure, cela ne veut pas pour autant dire qu'il est «~incensurable~» contrairement à ce qui est parfois affirmé à des fins de simplification. La censure dans Bitcoin n'est pas seulement possible, mais elle est probable dans le cas où son adoption aurait dépassé un certain stade.

% Définition dans Bitcoin
Lorsqu'on parle de Bitcoin, le terme de censure possède un sens précis~: il désigne l'action d'empêcher une transaction d'être réalisée sur une base économiquement irrationnelle, en entravant son inscription pérenne dans la chaîne de blocs. Cela rejoint l'idée de restreindre l'activité financière d'une entité dans le but de modeler son comportement.

% Une transaction constitue du discours
Cette censure s'apparente également à de la censure du discours, car on empêche l'individu de s'exprimer en publiant une transaction signée. En effet, Bitcoin est de l'expression pure~: les données sont publiques et les utilisateurs diffusent de nouvelles données qui sont traitées par les mineurs.

% Extrapolation
La façon dont peut s'exercer la censure peut être déduite de ce qui existe déjà dans le monde bancaire et de ce qui se met progressivement en place dans les cryptomonnaies. Ce qui suit est donc une extrapolation raisonnable de la réalité.

% Prétextes
Tout d'abord, les prétextes justifiant la censure dans Bitcoin sont connus. D'une part, ceux utilisés dans la finance traditionnelle peuvent être appliqués à Bitcoin, comme la lutte contre le blanchiment d'argent, le financement du terrorisme et la protection des épargnants~: la cryptomonnaie permet en effet d'éviter l'impôt, de financer tous les projets imaginables et de participer à des escroqueries. D'autre part, de nouveaux prétextes émergent comme la dévaluation de la monnaie locale (un instrument déflationniste est une concurrence déloyale) ou la lutte contre le changement climatique (le minage émet du CO\textsubscript{2}).

% Réglementations
De ces prétextes, les autorités tirent des réglementations générales qui s'appliquent à l'échelle internationale, comme cela c'est déjà le cas dans le système bancaire mondial. Les différentes juridictions se basent sur les recommandations du GAFI, dont le rôle premier est la lutte contre le blanchiment d'argent et le financement du terrorisme, et elles sont fortement poussées à appliquer ces recommandations sous peine de subir les sanctions économiques des États-membres\sendnote{Le 21 février 2021, le GAFI a par exemple appelé ses 39 États-membres à «~appliquer des contre-mesures~» contre ce pays qui refuse d'appliquer les normes de lutte contre le terrorisme, ce qui a renforcé au passage les sanctions déjà appliquées des États-Unis}. Le FMI peut également être mis à profit, celui-ci ayant pour but d'assurer la stabilité du système monétaire mondial (donc de protéger les monnaies des États-membres).

% Listes noires
Cela permet de constituer des listes noires d'adresses ne rentrant en conformité avec ces réglementations et ces listes sont distribuées aux divers acteurs financiers réglementés. On peut notamment la liste dressée par l'\eng{Office of Foreign Assets Control} (OFAC), l'organisme dépendant du Trésor étasunien en charge d'appliquer les sanctions internationales des États-Unis dans le domaine financier\sendnote{U.S. Department of the Treasury, \eng{Treasury Sanctions IRGC-Affiliated Cyber Actors for Roles in Ransomware Activity}, 14 septembre 2022~: \url{https://home.treasury.gov/news/press-releases/jy0948}~; \url{https://home.treasury.gov/policy-issues/financial-sanctions/recent-actions/20220914}.}, qui fait autorité dans le domaine financier en raison de l'extraterritorialité du droit étasunien.

% Censure (réglementation + listes noires) de l'économie
Le censure s'applique ainsi déjà dans une partie de l'économie basée sur Bitcoin. Tous les acteurs qui se conforment aux réglementations bloquent les bitcoin (et autres cryptomonnaies) provenant des adresses présentes sur les listes noires et gèlent les comptes de l'utilisateur jusqu'à ce qu'il se justifie. Toutefois, cette pratique conserve un caractère partiel et implicite~: les transactions en elles-mêmes ne sont pas encore explicitement interdites, mais les fonds ne doivent pas être envoyés aux intermédiaires financiers réglementée, comme les plateformes de change ou les processeurs de paiement\sendnote{Depuis le début de l'année 2021, BitPay demande à ses clients européens de s'inscrire et de vérifier leur identité avant de pouvoir effectuer un achat.}. Cette situation pousse certaines plateformes d'échange à faire beaucoup de zèle dans le domaine en refusant des bitcoins provenant de mélanges de pièces et geler les comptes des personnes le faisant\sendnote{Jamie Redman, \eng{As FATF Regulations Galvanize, Crypto Mixing Applications Are Targeted}, 27 décembre 2019~: \url{https://news.bitcoin.com/as-fatf-regulations-galvanize-crypto-mixing-applications-are-targeted/}~; 6102bitcoin, \eng{CoinJoin Flagging}~: \url{https://6102bitcoin.com/coinjoin-flagging/}.}, en l'absence d'une réglementation explicite\sendnote{Sur Ethereum, les adresses liées au contrat de mélange Tornado Cash ont été placées sur la liste de l'OFAC~; mais sur Bitcoin, aucune loi ni liste liée au mélange n'est connue~: il y a juste une suspicion généralisée.}.

% Réglementation de l'industrie minière
La réglementation peut également s'étendre à l'industrie minière. Comme on l'a dit\sendnote{Voir la section sur l'industrie minière dans le chapitre~\ref{ch:confirmation}.}, l'activité minière tend naturellement à se centraliser, par l'agrégation de la puissance de hachage en fermes de minage, par le rassemblement des hacheurs en coopératives minières et par l'utilisation de relais de communication par ces coopératives. Ces gros acteurs sont donc facilement identifiables et se soumettent aux réglementations concernant les transactions à traiter, ce qui pourrait amener une censure sur le réseau.

% Censure passive
La censure par les mineurs peut être exercée tout d'abord de manière passive, ce qui consiste à refuser de confirmer des transactions pour des raisons qui ne seraient pas économiquement rationnelles, typiquement sous la pression du régulateur. Ce type de censure avait été envisagé par la coopérative du groupe Marathon, qui avait déclaré en 2021 vouloir pratiquer le «~minage de blocs propres\sendnote{«~Les avantages de la coopérative minière comprennent, entre autres~: la participation aux bénéfices, les mineurs recevant des remises en fonction de leur contribution en taux de hachage~; une transparence accrue, toutes les informations financières étant vérifiées par un cabinet d'audit financier tiers basé aux États-Unis~; des efforts de lobbying pour améliorer la politique et l'environnement réglementaires en Amérique du Nord pour les mineurs~; un "minage de blocs propres" qui respecte les normes de conformité de l'Office of Foreign Asset Control (OFAC) et réduit le risque de miner des blocs comprenant des transactions liées à des activités illicites.~» -- Communiqué de Marathon et de DMG Blockchain Solutions, \eng{Marathon Patent Group and DMG Blockchain Solutions to Form the Digital Currency Miners of North America (DCMNA) and Launch North America's First Cooperative Mining Pool}, 5 janvier 2021~: \url{https://web.archive.org/web/20210128112455/https://www.marathonpg.com/news/press-releases/detail/1220/marathon-patent-group-and-dmg-blockchain-solutions-to-form}.}~», avant de se rétracter sous la pression populaire. Cette filtration est mise en place sur Ethereum avec les validateurs qui utilisent des relais d'optimisation de MEV\sendnote{La valeur extractible maximale (\eng{maximal extractable value}), initialement appelée valeur extractible par les mineurs (\eng{miner extractable value}) est la valeur maximale que le validateur peut générer en modifiant l'ordre ou en excluant des transactions au sein de son bloc, profitant des différentes irrégularités des contrats autonomes, notamment en ce qui concerne les places de marché décentralisées. En octobre 2022, la quantité de validation passant par des relais appliquant ce type d'optimisation dépassait les 50~\%, indiquant la potentialité d'une attaque. -- \url{https://journalducoin.com/ethereum/ethereum-transactions-bientot-censuree-ofac/}.} qui respectent les normes de l'OFAC et par conséquent n'incluent pas les transactions sales. Les participants à ces relais sont principalement des plateformes d'échange \textcolor{darkgray}{aujourd'hui}\sendnote{MEV Watch~: \url{https://www.mevwatch.info/}.}. % Benefits of the mining pool include, but are not limited to: profit sharing, whereby miners receive rebates based on their contributed hashrate; increased transparency as all financial information will be audited by a third-party U.S.-based financial audit firm; lobbying efforts to improve the policies and regulatory environment in North America for miners; "clean block mining" that adheres to the Office of Foreign Asset Control's (OFAC's) compliance standards and reduces the risk of mining blocks that include transactions linked to nefarious activities.

Cette censure passive n'est pas très problématique car elle demande que 100~\% de la puissance de calcul s'y conforme pour être effective. Les mineurs dissidents (qui ignorent les réglementations) débloquent la situation en validant les transactions suspectes. Seuls les délais de confirmation sont affectés. Mais cela devient autrement plus grave lorsque les mineurs conformistes commencent à refuser les blocs contenant les mauvaises transactions~: on entre alors dans une période de censure active.

% Censure active
La censure active consiste à empêcher des transactions d'être confirmées en rendant orphelins tous les blocs qui les contiennent. Sur Bitcoin, elle nécessite de disposer de la majorité de la puissance de calcul du réseau. Il s'agit donc d'une attaque des 51~\%.

Le coût d'une telle attaque peut être colossal suivant la puissance de calcul déployée sur le réseau comme on l'a vu précédemment\sendnote{On a vu dans le chapitre~\ref{ch:confirmation}, que le coût d'une telle attaque se chiffre en milliards de dollars sur le réseau Bitcoin principal.}. Mais il serait justifié par le développement des activités illégales évitant l'impôt et le seigneuriage. En effet, le profil-type de l'attaquant est l'État dont le pouvoir de prélèvement repose grandement de son contrôle sur la monnaie\sendnote{Ce à quoi Bitcoin s'oppose est décrit dans le chapitre~\ref{ch:adversaire}.}~: c'est pourquoi il se fiche de réduire (voire de détruire) l'utilité de Bitcoin ce faisant.

% Intensification du conflit (guerre contre Bitcoin, marché noir)
Une telle attaque serait précédée d'une déclaration de guerre contre Bitcoin. Toute la tolérance vis-à-vis des utilisateurs disparaîtrait, et ce qui n'était pas officiel le deviendrait~: toutes les transactions qui ne sont pas explicitement autorisées serait déclarées interdites. L'utilisation serait criminalisée, et le minage aussi.

% Cela s'accompagnerait probablement d'une mise en avant d'une alternative~: soit une version plus conforme de Bitcoin, soit une monnaie numérique de banque centrale (MNBC), présentée comme plus sûre et moins volatile.

% Cooptation, déploiement de matériel et attaque
Ce durcissement permettrait de coopter plus largement tous les regroupements miniers, auxquelles les directives étatiques seraient transmises. L'État pourrait aussi réquisitionner ou acheter son propre matériel de hachage. En somme, il disposerait à un moment donné d'une puissance de calcul majoritaire. Une fois la puissance de calcul rassemblée, l'attaque serait mise à exécution.

% Prolongement de l'attaque
La censure active est pernicieuse car il suffit que 51~\% l'applique pour qu'elle continue. Son prolongement dans le temps peut finir par constituer un nouveau point de Schelling. Par conséquent, les mineurs économiquement rationnels ont tout intérêt à appliquer la censure comme le montre un article de Juraj Bednar sur le sujet\sendnote{Juraj Bednar, \eng{Bitcoin censorship will most likely come, pt 2}, 18 novembre 2020~: \url{https://juraj.bednar.io/en/blog-en/2020/11/18/bitcoin-censorship-will-most-likely-come-pt-2/}.}. L'attaquant ne nécessite donc pas de disposer toujours de la majorité du taux de hachage.

% Cas extrême, attaque Goldfinger
Même dans le cas où les utilisateurs refusent de se conformer aux normes de surveillance étatiques, ils ne peuvent rien faire contre la censure. L'attaque prend alors d'une destruction totale de l'utilité de la chaîne par le minage de blocs vides. On parle d'une attaque Goldfinger, nommée comme cela en 2013 par référence au principal antagoniste du film de James Bond du même nom sorti en 1964, qui souhaite irradier stock d'or américain sécurisé au dépôt de Fort Knox dans le but de le rendre temporairement inutilisable et d'augmenter la valeur du reste de l'or\sendnote{Joshua A. Kroll, Ian C. Davey, Edward W. Felten, \eng{The Economics of Bitcoin Mining, or Bitcoin in the Presence of Adversaries}, 2013~: \url{https://asset-pdf.scinapse.io/prod/2188530018/2188530018.pdf}.}.

Ainsi, il est tout à fait possible d'exercer de la censure dans Bitcoin. Toutefois, ce n'est ni facile, ni définitif, car il existe un mécanisme au sein du protocole  permettant de lutter contre ce type d'attaque~: le mécanisme de résistance à la censure.

\section*{Le mécanisme de résistance à la censure}
\addcontentsline{toc}{section}{Le mécanisme de résistance à la censure}

% Définition de la résistance à la censure
La résistance à la censure désigne la difficulté d'entraver arbitrairement les transactions. Elle est couramment évoquée et constitue l'une des deux grandes promesses de Bitcoin~: permettre à quiconque d'envoyer des fonds à n'importe qui d'autre, quel que soit le moment, où que se trouve le destinataire dans le monde pourvu qu'il dispose d'un accès à Internet.

% La résistance à la censure est essentielle
La résistance à la censure constitue un élément essentiel de Bitcoin. Si elle n'existait pas, le système ne pourrait tout simplement pas survivre en tant que tel~: il deviendrait un système contrôlé centralement par une autorité décidant des bonnes et des mauvaises transactions~; il devrait s'adapter, tel PayPal ou GoldMoney, ou périr, à l'instar de e-gold ou de Liberty Reserve. De plus, le pouvoir absolu sur la sélection des transactions permettrait à cette autorité d'exercer \eng{de facto} une influence irrésistible sur le protocole au travers de l'application de soft forks, ce qui mènerait \emph{in fine} à la destruction de la politique monétaire originelle\sendnote{Pour connaître comment les règles de consensus peuvent être imposées par des soft forks, se référer au chapitre~\ref{ch:determination}.}. Sans elle, la proposition de valeur de Bitcoin s'effondrerait.

% Pas de description par Satoshi
Cependant, cette résistance à la censure n'a jamais été décrite explicitement par Satoshi Nakamoto. Dans ses interventions, le créateur de Bitcoin a expliqué comment son système était sécurisé économiquement contre la double dépense, ce qui était déjà une grande évolution par rapport aux systèmes décentralisés précédents. Mais il n'a en revanche pas indiqué comment le système pouvait s'opposer à la censure, c'est-à-dire au blocage partiel ou total par une entité hostile. Il semblait se reposer sur la bonne volonté des mineurs «~honnêtes~», pensant même qu'il y aurait «~probablement toujours des nœuds prêts à traiter les transactions gratuitement\sendnote{Satoshi Nakamoto, \eng{Bitcoin v0.1 released}, \wtime{08/01/2009 19:27:40 UTC}~: \url{https://www.metzdowd.com/pipermail/cryptography/2009-January/014994.html}.}~», cette résistance allant de soi.

% Mécanisme de résistance à la censure mis en lumière par Eric Voskuil
Le mécanisme de résistance à la censure de Bitcoin a été mis en lumière en 2018, par le développeur et auteur Eric Voskuil, qui a montré qu'il reposait de manière essentielle sur les frais de transaction\sendnote{Ce mécanisme a initialement été décrit par Eric Voskuil en janvier 2018~: \url{https://github.com/libbitcoin/libbitcoin-system/wiki/Other-Means-Principle/77d7556a14f89d1704f1bb97ca0aed04606363d0}. Voir aussi Eric Voskuil, \emph{Cryptoéconomie}, «~Propriété de résistance à la censure~» (p. 24) et «~Principe des autres moyens~» (p. 48).}. Comme dans le cas de la résistance à la double dépense, la propriété de résistance à la censure n'est pas absolue mais économique~: c'est une régulation portée par les frais des transactions prohibées.

% Principe de la majorité
La sécurité minière, on le rappelle, repose sur un principe majoritaire~: la quantité de puissance de calcul contrôlée par les mineurs honnêtes doit être supérieure par rapport à celle des attaquants\sendnote{«~Le système est sécurisé tant que les nœuds honnêtes contrôlent collectivement plus de puissance de calcul qu'un groupe de nœuds qui coopéreraient pour réaliser une attaque.~» -- Satoshi Nakamoto, \eng{Bitcoin: A Peer-to-Peer Electronic Cash System}, 31 octobre 2008.}. L'important n'est pas que le taux de hachage de Bitcoin soit le plus haut possible~; c'est que les mineurs disposant d'une puissance de calcul non négligeable soient prêts à miner systématiquement toutes les transactions payant un montant correct de frais et à toujours construire leurs blocs à partir de la plus longue chaîne. % Pas de censue passive et pas de recoordination de chaîne

% Sécurité = activité × distribution × participation
Ainsi, cette sécurité ne dépend pas uniquement de la puissance de calcul. Elle dépend aussi de la distribution de cette puissance de calcul et de la fraction de mineurs par rapport au reste de l'humanité\sendnote{Eric Voskuil, \emph{Cryptoéconomie}, «~Modèle de sécurité qualitatif~» (p. 59).}. En effet, un taux de hachage qui serait concentré dans les mains d'un seul mineur créerait une sécurité équivalente à celle d'un système centralisé, dépendante du mineur en question. Aussi, un réseau équitablement distribué et déployant une grande quantité puissance de calcul aura plus de risque d'être coopté s'il comporte un petit nombre de mineurs que s'il en comporte un grand nombre.

% Compensation du risque
La solution au problème de la censure provient des mineurs dissidents, qui sont prêts à confirmer des transactions litigieuses ou décrétées comme illégales par le pouvoir. Le risque pris par ces mineurs doit alors être compensé économiquement.

% Anonymat du mineur
Le mineur dissident doit rester anonyme et doit pouvoir miner dans la clandestinité. D'où le fait qu'il n'ait jamais à s'identifier dans le protocole. Le signalement des blocs minés par les coopératives minières par l'intermédiaire de la transaction de récompense est une démarche purement optionnelle et volontaire, ayant pour but de rassurer les utilisateurs (leurs clients) sur la distribution du système.

% Rôle de la création monétaire
La part du revenu du minage provenant de la création monétaire joue un rôle mineur dans la lutte contre la censure, que cette dernière soit passive ou active. D'une part, cette partie de la récompense est la même pour tous les mineurs, ce qui fait qu'elle n'influe pas sur leur choix économique d'inclure une transaction ou non dans leurs blocs. D'autre part, la potentielle chute de l'utilité (et donc du revenu de minage) du système provoquée par une censure active (attaque), ne saurait empêcher le censeur d'arriver à ses fins. Les motivations de ce dernier sont en effet particulières~: il ne cherche pas à réaliser un profit direct mais à contrôler, voire détruire, le système en décrétant quelles sont les transactions autorisées et les transactions non autorisées.

% Rôle des frais
En revanche, les frais de transaction sont eux essentiels au mécanisme de résistance à la censure. D'une part, ils luttent contre la censure passive en incitant les mineurs à confirmer la transaction. D'autre part, ils découragent la censure active en donnant à la branche censurée une importance économique plus grande.

% Intégration des frais
Les frais sont intégrés au protocole. Ils sont donc généralement l'association directe avec la transaction. Ceux-ci peuvent être payés par les utilisateurs qui envoient les transactions. Mais ils peuvent également être payés par le commerçant qui les reçoit, par l'application d'une réduction au prix des biens vendus.

% Attaque
Dans le cas d'une attaque de censure active, le censeur acquiert plus de la moitié de la puissance de calcul du réseau et rejette ouvertement un groupe de transactions défini (par une liste noire par exemple) en refusant les blocs qui contiendrait l'une d'entre elles. La chaîne du censeur est considérée par les nœuds honnêtes comme la chaîne correcte car elle est plus longue.

% Réaction
C'est alors que les frais de transaction interviennent. Les initiateurs des transactions censurées, voyant que leurs transactions ne sont pas confirmées, augmentent leurs frais. C'est un comportement naturel que l'on a déjà observé lors des périodes de congestion du réseau, comme au sommet de la bulle de 2017, lorsque les frais médians par transaction avaient dépassé les 30~\$\sendnote{\url{https://bitinfocharts.com/comparison/bitcoin-median_transaction_fee.html#alltime}}. En particulier, il est logique de payer une grande quantité de frais pour transférer de fortes sommes, celles-ci étant plus à risque que les petits transferts\sendnote{Dans Bitcoin, les frais sont aujourd'hui payés proportionnellement la charge de données pour le nœud (taille ou poids de la transaction). Cependant, la menace de plus en plus claire de la censure pourrait pousser les gens à payer des frais proportionnels au montant transféré comme cela se fait dans le domaine financier en général.}.

% Supplément de frais
Cette augmentation crée un supplément de frais, qui constitue la différence entre les frais de toutes les transactions et ceux des transactions non autorisées, et qui peut être observé dans la mempool des nœuds honnêtes. C'est ce supplément (et ce supplément uniquement) qui incite les mineurs dissidents à déployer plus de puissance de calcul au cours du temps~: plus l'économie supprimée est importante, plus la puissance de calcul résultante est grande.

% Riposte
Les mineurs dissidents se coordonnent en privé ou par la voie d'un signalement pour planifier une riposte. Une fois que la puissance de calcul est jugée suffisante, ils se mettent à confirmer les transactions censurées. Puisque leur puissance de calcul est majoritaire, leur chaîne devient la plus longue et la chaîne du censeur est invalidée. De cette manière, la censure est vaincue, du moins jusqu'à une nouvelle offensive de l'ennemi.

% Mécanisme ancré dans le protocole
Ainsi, le mécanisme de résistance à la censure est profondément ancré dans le protocole. La preuve de travail, le caractère anonyme du minage, le système de frais intégré~: ce sont autant d'éléments permettant de coordonner un marché des frais qui permet de repousser le censeur. Il est impossible d'estimer la partie de l'économie censurée, l'envergure de l'attaque étatique ou le montant de frais que les utilisateurs seraient prêts à payer, de sorte qu'on ne peut pas garantir l'incensurabilité de Bitcoin. Mais le mécanisme n'en est pas moins valide.

% Négligence du rôle des frais de transactions
Le rôle des frais de transaction, explicité par Eric Voskuil, a été négligé par certains protocoles cryptoéconomiques. C'est le cas d'Ethereum qui a fait le choix de brûler une partie des frais du réseau dans le but de rendre l'éther déflationniste au sens monétaire avec l'activation de l'EIP-1559 le 5 août 2021\sendnote{Ludovic Lars, \emph{Ethereum face à une catastrophe annoncée~? Censure et volatilité~: l'EIP-1559, le cauchemar des mineurs}, 15 juillet 2021~: \url{https://journalducoin.com/analyses/eip-1559-changement-nefaste-ethereum/}.}. Ethereum a également choisi de passer en preuve d'enjeu en septembre 2022, ce qui marque un autre pas vers l'acceptation de la censure.

\section*{L'importance de la confidentialité}
\addcontentsline{toc}{section}{L'importance de la confidentialité}

Le censure financière est étroitement apparenté à la surveillance des transactions. Cela vaut pour le monde bancaire comme on l'a vu. Mais cela vaut aussi pour Bitcoin.

% --- Deux manières de protéger sa richesse ---

Il existe deux manières de protéger sa richesse~: par la force physique et par la dissimulation.

% Défense physique
La première méthode consiste à se prémunir contre le vol soit directement en défendant soi-même ses biens (si besoin à l'aide d'une arme à feu), soit indirectement par le biais des services de police étatiques ou par l'intermédiaire de services de protection privés (gardes du corps, quartiers sécurisés, agence de protection), très prisés des personnes très fortunées. Cette méthode est importante et utile contre les criminels communs, mais elle est relativement inefficace contre la puissance dominante locale dont nous sommes à la merci -- l'État.

% Défense par dissimulation
C'est pourquoi les individus ont plus souvent recours à la seconde méthode, qui consiste à dissimuler leur richesse pour ne pas que le voleur en ait connaissance et ne puisse pas s'en emparer directement. Cela permet de dissuader le voleur usant la menace de violence d'aller plus loin~: il pourrait nous interroger pour savoir où se trouve notre richesse, mais cela représenterait un coût supplémentaire (proportionnel à notre refus de lui livrer ces informations) qui constituerait un frein à sa recherche.

% Définition de la confidentialité
Cette méthode est appelée liée à la confidentialité (ou la protection de la vie privée) qui est le fait de réserver des informations à un petit nombre de personnes déterminées. La confidentialité est distincte du secret dans le sens où la personne peut choisir de révéler sélectivement des informations (confidence). Dans le contexte financier, il s'agit généralement de faire en sorte qu'une transaction ne soit connue que des participants.

% La confidentialité crée la liberté individuelle
La confidentialité forme la base de la liberté individuelle dans la société et constitue un caractéristique essentielle pour tout le monde. Elle sert en effet à \emph{créer une asymétrie} entre le faible et le fort, entre l'individu et l'État, de sorte que ce dernier ne puisse pas empiéter absolument sur les droits du premier. L'État voudra vous persuader du contraire, en vous disant que vous n'avez rien à craindre si vous n'avez rien à cacher\sendnote{«~Je dis que quiconque tremble en ce moment est coupable~; car jamais l'innocence ne redoute la surveillance publique.~» -- Maximilien de Robespierre, \emph{Discours du 11 germinal, an II}, 31 mars 1794~: \url{https://fr.wikisource.org/wiki/Discours_lors_de_la_séance_de_la_Convention_du_11_germinal_an_II}.}, mais il n'y a rien d'historiquement plus faux, comme l'ont montré les exemples de totalitarisme au \textsc{xx}\ieme{}~siècle.

% --- Lien entre résistance à la censure et confidentialité ---

De ce fait, puisque la censure financière est issue de l'initiative étatique, la résistance à la censure est donc généralement intrinsèquement liée à la confidentialité. Sans confidentialité, il n'y a pas de résistance à la censure individuelle~; et sans résistance à la censure, il n'y a pas de confidentialité individuelle.

% Confidentialité => résistance à la censure
D'une part, la résistance à la censure de l'utilisateur individuel repose sur la confidentialité du système. Si l'État connaît toutes vos transactions, il peut vous sanctionner pour une transaction non autorisée, quand bien même celle-ci serait traitée par le réseau.

Certains promoteurs de BTC mettent en avant la transparence du protocole, en faisant un avantage par rapport aux systèmes bancaires opaques, insistant sur le pseudonymat et réservant la propriété d'anonymat aux «~cryptomonnaies confidentielles~» comme Monero. Mais il s'agit d'une mécompréhension~: les données dans Bitcoin sont transparentes afin de permettre le consensus et l'audit, et Monero ne fait qu'implémenter un compromis différent sur la transparence des transactions.

% Résistance à la censure => confidentialité
D'autre part, la confidentialité de l'utilisateur dépend de la résistance à la censure du système. En effet, si l'État dispose d'un contrôle total sur la sélection des transactions, alors il peut choisir de ne confirmer que les transactions qui dévoilerait

Cette dépendance est souvent remise en question par certains partisans de Monero qui estiment que la confidentialité par défaut du système protège les utilisateurs de la censure, considérant que l'État ne peut pas censurer une transactions qu'on ne connaît pas. Néanmoins, cette vision est particulièrement naïve car les utilisateurs peuvent toujours révéler les informations relatives à leurs adresses aux organismes de surveillance\sendnote{Dans Monero et dans les systèmes apparentés, la révélation des transactions liées à une adresse se fait par l'intermédiaire de clés d'inspections (\eng{view keys}).}. C'est pour cela qu'un État pourrait choisir de refuser toute transaction non transparente. La résistance à la censure dépend aussi du coût de l'attaque, qui serait beaucoup moins coûteuse à réaliser dans le cas de Monero, \textcolor{darkgray}{probablement 200 fois moins coûteuse que sur BTC (mars 2023)}.

% Surveillance des intermédiaires
La confidentialité et la résistance à la censure sont donc interdépendantes dans Bitcoin. De ce fait, la surveillance constitue aujourd'hui un problème systémique. D'abord, les plateformes de change entre monnaies traditionnelles et cryptomonnaies appliquent des normes de connaissance du client (KYC) et de lutte contre le blanchiment (AML) similaires au système bancaire classique. De plus, le GAFI leur recommande aujourd'hui d'appliquer la «~règle du voyage~» (\eng{travel rule})  recommandée par le GAFI\sendnote{Recommandation 16 du GAFI, mise à jour en juin 2019 pour inclure le transferts d'actifs virtuels~: \url{https://www.fatf-gafi.org/fr/publications/Recommandationsgafi/Recommandations-gafi.html}.}, c'est-à-dire de vérifier systématiquement l'adresse de retrait de leurs clients, et la FINMA l'impose déjà aux acteurs suisses\sendnote{\url{https://www.finma.ch/en/news/2019/08/20190826-mm-kryptogwg/}}. Cette recommandation pourrait être appliquée par la mise en place du protocole de preuve de propriété d'adresse (AOPP) proposé en janvier 2022\sendnote{\url{https://aopp.group/}}. Ensuite, les sociétés d'analyse de chaîne telles que Chainalysis, interprètent les données et les fournissent à leurs clients qui sont les agences étatiques, les institutions financières et les grandes entreprises du domaine.

% Lutte contre la surveillance
En face, il se forme une lutte contre la surveillance. C'est le cas par exemple du mélange de pièces, ou CoinJoin, qui permet de brouiller les pistes. Mais ces méthodes doivent être utilisées, et elles doivent rester à jour de l'avancée de la surveillance. Cette aspect sera développé dans le chapitre~\ref{ch:rouages}.

% Importance de la confidentialité
Ainsi, l'importance de la confidentialité est certaine. Vous ne pouvez pas faire ce que vous voulez si vous agissez publiquement dans le monde. La plupart du temps, il suffit d'être discret et de vaquer à vos occupations. Comme l'écrivait le fabuliste Florian~: «~Pour vivre heureux vivons cachés.\sendnote{Le Grillon, \url{https://fr.wikisource.org/wiki/Collection_complète_des_œuvres_de_M._de_Florian/Fables/2/Le_Grillon}.}~»

\section*{Interventions humaines dans le consensus}
\addcontentsline{toc}{section}{Interventions humaines dans le consensus}

Lorsqu'on parle de la possibilité de censure, une autre alternative a tendance à séduire les personnes qui n'ont pas suffisamment creusé la question~: celle d'intervenir socialement sur la chaîne. Il s'agit d'avoir recours au «~consensus social\sendnote{Pour reprendre l'expression d'Arthur Breitman utilisée dans la première description formelle de Tezos en août 2014. -- Arthur Breitman, \eng{Tezos: A Self-Amending Crypto-Ledger}, 3 août 2014~: \url{https://tezos.com/position-paper.pdf}.}~», c'est-à-dire le mécanisme de détermination du protocole.

Deux idées semblent avoir un certain succès~: l'UASF anti-censure et l'UAHF de changement de preuve de travail.

% --- UASF anti-censure ---

La première idée est de rejeter la censure en invalidant la branche du censeur partiellement ou totalement, c'est-à-dire en portant atteinte au principe de la chaîne la plus longue. Cela peut se faire en rendant les blocs de la chaîne du censeur invalides ou en imposant la validité de la chaîne concurrente par un point de contrôle temporaire. Une telle mesure constituerait un soft fork (c'est-à-dire une restriction des règles de consensus) qui devrait être activé par les utilisateurs à un horodatage ou à une hauteur de bloc donné. Elle provoquerait une scission car non soutenue par la majorité de la puissance de calcul.

% Invalidation des blocs du censeur
Le premier cas peut être mis en place par mais celui-ci n'est facile à implémenter que si le censeur marque ses blocs d'une manière ou d'une autre\sendnote{Cette méthode pourrait être mise en place dans l'esprit de l'\eng{User Resisted Soft Fork} proposé par Michael Folkson en avril 2022 en réaction menace d'activation de la mise à niveau OP\_CHECKTEMPLATEVERIFY. -- Michael Folkson, \eng{[bitcoin-dev] User Resisted Soft Fork for CTV}, \wtime{21/04/2022 16:45:20 UTC}~: \url{https://lists.linuxfoundation.org/pipermail/bitcoin-dev/2022-April/020262.html}.}. Cela a été réalisé par Bitcoin ABC le 1\ier{} décembre 2020, pour contrer la censure active d'un mineur mécontent\sendnote{Un seul bloc (bloc 662~687, \longstring{00000000000000000709b858a6a0c8610e604e77072ef4407763afb0780ce712})de l'attaquant a été invalidé, faisant que 172 blocs ont été mis de côté, et que la chaîne non censurée est devenue la chaîne correcte. -- \url{https://twitter.com/nikzh/status/1333884127070851073}, \url{https://twitter.com/nikzh/status/1333893457920876550}}.

% Point de contrôle
On peut aussi inclure un point de contrôle (\eng{checkpoint}) dans le protocole. Un point de contrôle est un bloc considéré comme valide par défaut. Ce mécanisme a été implémenté dans le logiciel de Bitcoin dès juillet 2010 dans le but d'éviter une recoordination de chaîne\sendnote{Satoshi Nakamoto, \eng{Bitcoin 0.3.2 released}, \wtime{17/07/2010 21:35:51 UTC}~: \url{https://bitcointalk.org/index.php?topic=437.msg3807\#msg3807}.} et certains de ces points de contrôle sont encore présents dans Bitcoin Core\sendnote{Voir \url{https://github.com/bitcoin/bitcoin/blob/24.x/src/chainparams.cpp\#L148-L164}. Le plus récent est celui du bloc 295~000 miné le 9 avril 2014 (au même moment de l'arrivée de Wladimir van der Laan au poste de mainteneur en chef) et ayant pour identifiant \longstring{00000000000000004d9b4ef50f0f9d686fd69db2e03af35a100370c64632a983}.}. Dans cette logique, il suffit d'imposer un bloc comme obligatoire pour invalider la chaîne du censeur. Cela a été réalisé par Bitcoin SV en août 2021\sendnote{\url{https://twitter.com/BitcoinAssn/status/1422668065024663554}}, qui subissait une censure active.

% Vitalik
Cette idée était déjà évoquée par Vitalik Buterin en 2016 dans le cas de la preuve d'enjeu~:

\begin{quote}
«~Sur des échelles de temps moyennes à longues, les humains sont assez bons pour le consensus. Même si un adversaire avait accès à une puissance de hachage illimitée, et qu'il parvenait à réaliser une attaque des 51~\% contre une blockchain majeure en inversant ne serait-ce que le dernier mois d'histoire, il serait beaucoup plus difficile de convaincre la communauté que cette chaîne est légitime que de simplement distancer la puissance de hachage de la chaîne principale. [...] Ces considérations sociales sont ce qui protège finalement toute blockchain à long terme, que la communauté de la blockchain l'admette ou non (notez que Bitcoin Core admet cette primauté de la couche sociale).\sendnote{Vitalik Buterin, \eng{A Proof of Stake Design Philosophy}, 30 décembre 2016~: \url{https://medium.com/@VitalikButerin/a-proof-of-stake-design-philosophy-506585978d51}, \url{https://vitalik.ca/general/2016/12/29/pos_design.html}.}~»
\end{quote}

% "On medium to long time scales, humans are quite good at consensus. Even if an adversary had access to unlimited hashing power, and came out with a 51\% attack of any major blockchain that reverted even the last month of history, convincing the community that this chain is legitimate is much harder than just outrunning the main chain's hashpower. They would need to subvert block explorers, every trusted member in the community, the New York Times, archive.org, and many other sources on the internet; all in all, convincing the world that the new attack chain is the one that came first in the information technology-dense 21st century is about as hard as convincing the world that the US moon landings never happened. These social considerations are what ultimately protect any blockchain in the long term, regardless of whether or not the blockchain's community admits it (note that Bitcoin Core does admit this primacy of the social layer).
%
% However, a blockchain protected by social consensus alone would be far too inefficient and slow, and too easy for disagreements to continue without end (though despite all difficulties, it has happened); hence, economic consensus serves an extremely important role in protecting liveness and safety properties in the short term."

% Efficacité de l'intervention humaine
Ce type de recours social peut fonctionner ponctuellement et pour un certain temps. Mais il ne constitue en rien un mécanisme robuste de résistance à la censure, bien au contraire. Cela crée beaucoup trop d'instabilité, en faisant reposer le consensus sur l'accord social. Une puissance hostile qui aurait pour but de détruire le système parviendrait facilement à créer des scissions multiples impossibles à départager par un facteur objectif, en faisant pression sur les différents meneurs de la communauté.

% Pourquoi l'UASF anti-censure est une mauvaise idée
L'intervention directe de l'accord social dans la confirmation des transactions est par conséquent une très mauvaise idée. Même dans les cas où les participants sont d'accord pour dire qu'un tel évènement est indésirable, ils sont souvent en total désaccord sur la manière de traiter le problème, ainsi que l'a montré la scission entre Ethereum et Ethereum Classic. Les être humains sont capables se mettre d'accord à long terme, comme le témoigne la convergence vers un petit nombre de langues, de religions, de monnaies. Néanmoins, à court terme ce n'est très certainement pas le cas. D'où le recours à un mécanisme de consensus automatisé, qu'est le minage.

% --- UAHF de changement de preuve de travail de preuve de travail ---

Une autre mesure proposée, moins subjective mais plus perturbatrice, est la modification de la fonction de preuve de travail. En effet, celle-ci permettrait faire cesser l'attaque à court terme puisqu'elle rendrait le matériel du censeur invalide, lui faisant supporter une lourde perte au passage. Il s'agirait d'un hard fork (c'est-à-dire une modification incompatible des règles de consensus) qui devrait être activé par les utilisateurs à un horodatage ou à une hauteur de bloc donné.

% Option nuéclaire
Cette option extrême a été soutenue par les développeurs luke-jr\sendnote{\url{https://www.reddit.com/r/Bitcoin/comments/3fg0jw/could_a_cartel_of_pool_operators_collude_to/ctoat0d/}} et Gregory Maxwell\sendnote{\url{https://www.reddit.com/r/bitcoinxt/comments/41pbmf/maxwell_considers_changing_the_pow_algorithm_in/}} lors de la guerre des blocs en 2015 -- 2016. Elle a également été évoquée par le développeur en chef de Bitcoin ABC Amaury Séchet en novembre 2018 qui l'a qualifiée d'«~option nucléaire [...] de dernier recours\sendnote{\url{https://twitter.com/deadalnix/status/1061947426096009216}}~».

% Pourquoi le hard fork de changement de preuve de travail est une mauvaise idée
Mais il s'agit généralement d'une mesure ayant des effets plus néfastes que le statu quo. Premièrement, elle ferait subir en parallèle les mêmes pertes aux mineurs honnêtes et aux mineurs dissidents. Deuxièmement, la valeur serait répartie entre deux chaînes. Troisièmement, le coût d'une attaque serait drastiquement réduit à court terme. Quatrièmement, les mineurs perdraient confiance dans le protocole et s'assureraient contre le risque d'un nouveau changement, réhaussant le coût de la sécurité minière par rapport au coût de l'attaque. Et cinquièmement, la nouvelle distribution ne serait pas forcément meilleure que l'ancienne, les gros mineurs pouvant déployer du capital plus facilement.

% Effet néfaste de l'intervention humaine
De manière générale, l'intervention humaine à court terme est loin d'être désirable. Si la chaîne subit une attaque minière, il est probable qu'elle soit aussi attaquée au niveau social. Les interventions ont ainsi toutes les chances de se multiplier, faisant sombrer la chaîne dans une spirale de scissions et la menant à l'insignifiance économique. Le cas de Bitcoin Cash est le plus éclairant~: en raison de hard forks programmés tous les six mois, la chaîne a subi deux scissions majeures après sa séparation d'avec BTC (en 2018 avec BSV et en 2020 avec XEC), ce qui a mené l'ensemble à être valorisé \textcolor{darkgray}{à moins de 1~\% de la valeur agrégée de BTC}. En outre, si cette observation est valable pour les cryptomonnaies en construction, qui peuvent se permettre ces interventions en raison de la petitesse et de l'homogénéité de leur économie, elle l'est d'autant plus pour une version mature de Bitcoin qui soutiendrait une économie plus grande et plus diversifiée.

\section*{Variantes des consensus par preuve de travail}
\addcontentsline{toc}{section}{Variantes des consensus par preuve de travail}

% Alternatives, variantes
Devant le risque de censure, des alternatives à l'algorithme de consensus de Nakamoto ont été proposées. L'alternative la plus connue est la preuve d'enjeu, qui sera décrite dans la section suivante. Les autres alternatives sont des variantes de l'algorithme de Nakamoto par preuve de travail, dont les trois principales sont le minage combiné, la preuve d'espace et la finalisation anticipée. % protection contre la recoordination profonde, "checkpoints automatiques"

% --- Minage combiné ---

La première proposition est le minage combiné\sendnote{Aljosha Judmayer, Alexei Zamyatin, Nicholas Stifter, Artemios G.  Voyiatzis, Edgar Weippl, \eng{Merged Mining: Curse of Cure?}, 22 août 2017~: \url{https://eprint.iacr.org/2017/791}.}. Le minage combiné, ou \eng{merged mining} en anglais, est l'action de miner plusieurs chaînes en simultané par la réutilisation du travail fourni sur une chaîne parente pour la validation des chaînes filles ou auxiliaires.

Le procédé a été décrit par Satoshi Nakamoto en décembre 2010, dans un message concernant BitDNS, le projet de système distribué de noms de domaine. Le créateur de Bitcoin écrivait ainsi sur le forum~:

\begin{quote}
«~Je pense qu'il serait possible que BitDNS forme un réseau complètement séparé et possède une chaîne de blocs distincte, tout en partageant la puissance de calcul avec Bitcoin. Le seul chevauchement consisterait à faire en sorte que les mineurs puissent rechercher des preuves de travail pour les deux réseaux simultanément.

Les réseaux n'auraient besoin d'aucune coordination. Les mineurs adhéreraient aux deux réseaux en parallèle. Ils scanneraient SHA de telle sorte que s'ils obtenaient un résultat, ils pourraient résoudre les deux en même temps. Une solution pourrait concerner un seul des réseaux si l'un d'eux présente une difficulté moindre.

Je pense qu'un mineur externe pourrait appeler getwork sur les deux programmes et combiner le travail. Peut-être appeler Bitcoin, en tirer du travail, le remettre à getwork sur BitDNS pour le combiner en un travail commun.

Au lieu d'une fragmentation, les réseaux partageraient et augmenteraient la puissance de calcul totale de chacun. Cela résoudrait le problème des réseaux multiples, qui constituent un danger les uns pour les autre si la puissance de calcul disponible se concentre sur l'un d'entre eux. Au lieu de cela, tous les réseaux du monde partageraient la puissance de calcul combinée, augmentant ainsi la puissance totale. Il serait plus facile pour les petits réseaux de se lancer en puisant dans une base existante de mineurs.\sendnote{Satoshi Nakamoto, \eng{Re: BitDNS and Generalizing Bitcoin}, \wtime{09/12/2010 21:02:42 UTC}~: \url{https://bitcointalk.org/index.php?topic=1790.msg28696\#msg28696}.}~»
\end{quote}

% "I think it would be possible for BitDNS to be a completely separate network and separate block chain, yet share CPU power with Bitcoin.  The only overlap is to make it so miners can search for proof-of-work for both networks simultaneously.
%
% The networks wouldn't need any coordination.  Miners would subscribe to both networks in parallel.  They would scan SHA such that if they get a hit, they potentially solve both at once.  A solution may be for just one of the networks if one network has a lower difficulty.
%
% I think an external miner could call getwork on both programs and combine the work.  Maybe call Bitcoin, get work from it, hand it to BitDNS getwork to combine into a combined work.
%
% Instead of fragmentation, networks share and augment each other's total CPU power.  This would solve the problem that if there are multiple networks, they are a danger to each other if the available CPU power gangs up on one.  Instead, all networks in the world would share combined CPU power, increasing the total strength.  It would make it easier for small networks to get started by tapping into a ready base of miners."

% Preuves de travail auxiliaires
Le minage combiné consiste à réutiliser des preuves de travail partielles d'une chaîne parente comme des preuves de travail valides sur une chaîne fille. Ces preuves de travail, dite «~auxiliaires~» et abrégées en AuxPOW, sont des sous-produits du minage de la chaîne parente, et ne coûtent rien à produire de plus. La seule charge supplémentaire est la gestion de la chaîne fille.

% Récompense des mineurs de la chaîne fille
Les mineurs de la chaîne fille reçoivent des récompenses supplémentaires qui sont constituées de la création monétaire locale (si la chaîne utilise une nouvelle unité de compte) et des frais de transaction. Les mineurs de la chaîne parente sont donc incités à tirer profit de cette nouvelle manne. La chaîne fille peut de ce fait disposer d'un taux de hachage conséquent assez rapidement.

% Facilitation de l'amorçage d'une nouvelle chaîne indépendante
Le minage combiné a été mis en avant comme une méthode permettant de faciliter l'amorçage d'une nouvelle cryptomonnaie, en bénéficiant de l'industrie minière établie. Ce type d'algorithme de consensus a ainsi été mis en place sur Namecoin par rapport à Bitcoin et sur Dogecoin par rapport à Litecoin.

% Chaîne latérale
Il est aussi suggéré pour les chaînes latérales. Il est ainsi implémenté de manière hybride dans RSK. Il est plus largement envisagé par Paul Sztorc dans sa proposition de Drivechain\sendnote{Paul Sztorc, CryptAxe, \eng{BIP-301: Blind Merged Mining}, 23 juillet 2019~: \url{https://github.com/bitcoin/bips/blob/master/bip-0301.mediawiki}.}.

% Sécurité incertaine
Cependant, l'apport en sécurité du minage combiné par rapport au minage classique est relativement faible. Le procédé permet d'augmenter le nombre d'acteurs impliqués et de restreindre les attaquants possibles (ceux-ci devant être des mineurs de la chaîne principale), mais il ne modifie pas le coût de l'attaque, qui dépend du revenu minier de cette chaîne et, dans le cas de la censure, des frais de transaction.

% Exemple du Coiledcoin
Une illustration éclatante de ce fait est l'exemple de Coiledcoin (CLC), une cryptomonnaie alternative créée en janvier 2012 qui a subi une attaque de censure fatale peu de temps après son lancement. Celle-ci a été réalisée par le développeur de Bitcoin luke-jr par le biais de sa coopérative de minage, Eligius, sans qu'il n'en informe les hacheurs. Dans son message d'explication, il précisait qu'aucun membre n'avait subi de perte, le coût étant surtout le temps qu'il a passé à configurer le logiciel\sendnote{luke-jr, \eng{Re: [DEAD] Coiledcoin - yet another cryptocurrency, but with OP\_EVAL!}, \wtime{06/01/2012 18:56:03 UTC}~: \url{https://bitcointalk.org/index.php?topic=56675.msg678006\#msg678006}.}.

% Effets sur la sécurité de la chaîne parente
Le minage combiné a deux effets sur la sécurité minière de la chaîne parente. Tout d'abord, il augmente artificiellement la puissance de calcul déployée pour miner des blocs, ce qui paraît bénéfique de prime abord. Cependant, cette hausse artificielle ne fait rien contre la censure des transactions. De plus, le minage combiné entraîne une centralisation du minage, en raison de la charge que représente la gestion des chaînes auxiliaires~: si les chaînes auxiliaires sont importantes économiquement, les mineurs de la chaîne parente n'ont d'autre choix que de les miner pour rester rentables.

% Autre variante~: «~preuve de preuve~», notamment mise en place de manière hybride par le projet Veriblock de Jeff Garzik, qui consiste à placer une représentation de l'état de la chaîne sur une chaîne de blocs parente telle que celle de BTC. Komodo's Delayed Proof Of Work.

% --- Preuve d'espace ---

La deuxième alternative est la preuve d'espace (de l'anglais \eng{proof-of-space}), parfois aussi appelée preuve de capacité ou preuve de stockage, qui se base, non pas sur le calcul informatique, mais sur la capacité à garder des données en mémoire\sendnote{\url{https://eprint.iacr.org/2013/796}}. La ressource n'est plus la puissance de calcul, mais l'espace disque.

% Résistance aux ASIC
Cette idée a été partiellement incluse dans certains algorithmes hybrides de preuve de travail, dans le but de décourager le développement de matériel spécialisé (ASIC) et de favoriser le minage par processeurs accessibles au grand public (CPU et GPU). C'est le cas de la fonction scrypt (ou S-Crypt), une fonction de dérivation de clé coûteuse en mémoire adaptée par le mineur ArtForz pour être intégrée au sein de Tenebrix en septembre 2011\sendnote{Lolcust, \eng{[ANNOUNCE] Tenebrix, a CPU-friendly, GPU-hostile cryptocurrency}, \wtime{26/09/2011 00:09:44 UTC}~: \url{https://bitcointalk.org/index.php?topic=45667.msg544675\#msg544675}.}. Celle-ci a été reprise dans Litecoin\sendnote{Charlie Lee, \eng{Re: [ANN] Litecoin - a lite version of Bitcoin. Be ready when is launches!}, \wtime{09/10/2011 06:14:28 UTC}~: \url{https://bitcointalk.org/index.php?topic=47417.msg564414\#msg564414}.}.

% ETHash
C'est également le cas de l'ancienne fonction de minage d'Ethereum utilisé entre 2015 et 2022, ETHash, qui est une variante de l'algorithme Dagger-Hashimoto et qui rend le calcul de la preuve plus coûteux en mémoire par la nécessité de stocker un graphe acyclique orienté de plusieurs gigaoctets\sendnote{\url{https://ethereum.org/en/developers/docs/consensus-mechanisms/pow/mining-algorithms/ethash/}}. Ethereum utilisait de plus une version modifiée de l'algorithme de Nakamoto, GHOST, qui a pour intérêt de sélectionner la chaîne la plus lourde en prenant en compte les blocs orphelins\sendnote{\eng{Ethereum Whitepaper}, consulté le 11 mars 2023~: \url{https://ethereum.org/en/whitepaper/#modified-ghost-implementation}.}. Ethereum Classic utilise une variante de ETHash nommée ETCHash depuis novembre 2020\sendnote{\url{https://github.com/eth-classic/etchash/blob/main/README.md}}.

% RandomX
Un dernier exemple est l'algorithme RandomX, actif sur Monero depuis 2019, qui est conçu spécialement pour favoriser le minage par CPU\sendnote{\url{https://github.com/tevador/RandomX}}.

% Preuve d'espace
Au-delà des fonction de preuve de travail coûteuse en mémoire, on peut utiliser des algorithmes de preuve d'espace pure. C'est de la cas du système Chia Network, projet de Bram Cohen, qui se base sur les «~preuves d'espace et de temps~» pour déterminer la chaîne correcte\sendnote{Ludovic Lars, \eng{Face à la preuve de travail de Bitcoin, la preuve d'espace, une fausse solution écologique}, 19 mai 2021~: \url{https://journalducoin.com/analyses/preuve-espace-fausse-solution-ecologique/}.}.

% Apport supposé de ces algorithmes
Ces algorithmes sont censés être plus résistants à la censure en facilitant la participation du grand public et en améliorant de ce fait la distribution de la validation. Mais ils ne font que déplacer le problème.

Ce qu'il faut comprendre avec ces algorithmes fondés à des degrés divers sur la mémoire informatique, c'est qu'il s'agit de dépenser de l'énergie extérieure d'une autre manière. La preuve d'espace revient en fin de compte à une autre forme de travail. Et ce travail peut être optimisé, non seulement au niveau du matériel (des ASIC pour Litecoin, Ethereum et Ethereum Classic ont fini par être développés) mais aussi au niveau de l'industrie elle-même. Tout ce qui peut être fait, c'est rapprocher de l'optimisation matérielle de l'outil utilisé par le grand public.

% --- Finalisation anticipée des blocs ---

La troisième alternative est la finalisation anticipée des blocs. Celle-ci consiste à mettre en place des points de contrôle mobiles au sein du protocole, de sorte à considérer comme final tout bloc qui se trouverait en-dessous d'une certaine profondeur. Vitalik Buterin parle de subjectivité faible\sendnote{Vitalik Buterin, \eng{Proof of Stake: How I Learned to Love Weak Subjectivity}, 25 novembre 2014~: \url{https://blog.ethereum.org/2014/11/25/proof-stake-learned-love-weak-subjectivity}.}.

% Protection contre la recoordination profonde, Bitcoin Cash / eCash
Ce type d'algorithme a été mis en place par Bitcoin ABC le 20 novembre 2018 au sein de Bitcoin Cash, face à la menace d'attaque de la part du camp de Bitcoin SV\sendnote{Bitcoin ABC, \eng{Bitcoin ABC 0.18.5 Released}, 20 novembre 2018~: \url{https://www.bitcoinabc.org/2018-11-20-bitcoin-abc-0-18-5/}.} au travers d'une protection contre la recoordination profonde. Celle-ci consiste à considérer un bloc comme final au bout de 11 confirmations. Ce procédé est encore présent dans certaines implémentations de Bitcoin Cash et de eCash~/~Bitcoin ABC, et est appliqué par les grandes plateformes d'échange, ce qui en fait \emph{de facto} une règle de consensus. % autorise les recoordinations de 10 blocs (DEFAULT_MAX_REORG_DEPTH) https://github.com/Bitcoin-ABC/bitcoin-abc/commit/917d65774c40c6bfad500a660e581c8ea5e20df0

% MESS, Ethereum Classic
Dans Ethereum Classic, qui a subi de multiples attaques de double dépense en 2019 et en 2020, une variante de cette finalisation a été intégrée le 11 octobre 2020. L'algorithme en question est appelé \eng{Modified Exponential Subjective Scoring} (MESS) et consiste à attribuer différents scores aux branches concurrentes, privilégiant les segments vus les premiers par les nœuds aux segments vus ultérieurement. Il permet de réduire par 31 le coût d'une attaque\sendnote{Dean Pappas, \eng{An Elegant MESS -- The Fast Solution to 51\% attacks and Low Hash Rate}, 18 septembre 2020~: \url{https://medium.com/ethereum-classic-labs/an-elegant-mess-the-fast-solution-to-51-attacks-and-low-hash-rate-4e8f8347bdfe}.}.

% Problème de la subjectivité
Ces algorithmes réduisent en effet la probabilité d'une attaque opportuniste, car ils empêchent les recoordinations. Cependant, ils ont l'effet inverse sur les attaques de censure qui ont pour but de détruire l'utilité de la chaîne.

Ils posent en effet le problème de la subjectivité. Un nouveau nœud qui se synchronise avec le réseau peut être trompé par un attaquant en suivant la chaîne la plus longue et pas la chaîne considérée comme valide par le reste du réseau. Un attaquant (réalisant une attaque Goldfinger) peut facilement tirer profit de cette caractéristique en créant des chaînes concurrentes plus longues, causant la confusion.

Ce problème peut être atténué par une intervention sociale en décrétant un certain nombre de blocs comme valides par défaut. Mais on en revient alors à la situation discutée dans la section précédente.

% Conception idéale de Bitcoin
Idéalement, le concept de Bitcoin n'intègre aucun point de contrôle à l'exception du bloc de genèse défini préalablement, et la chaîne correcte est déterminée uniquement par la quantité de travail accumulée. Bien qu'il ait lui-même ajouté des points de contrôle manuels, Satoshi Nakamoto expliquait~:

\begin{quote}
«~Il n'y a aucun moyen pour le logiciel de savoir automatiquement si une chaîne est meilleure qu'une autre, sauf par la plus grande preuve de travail. Dans la conception, il était nécessaire qu'il se tourne vers la chaîne plus longue, quelle que soit la distance à parcourir.\sendnote{Satoshi Nakamoto, \eng{Re: checkpointing the block chain}, \wtime{16/08/2010 20:20:53 UTC}~: \url{https://bitcointalk.org/index.php?topic=834.msg9816\#msg9816}. -- Il ajoutait~: «~La seule exception à cela, ce sont les points de contrôle manuels que j'ai ajoutés. S'ils n'étaient pas là, il serait capable de se réorganiser en remontant jusqu'au premier bloc.~»}~»
\end{quote}

% "There is no way for the software to automatically know if one chain is better than another except by the greatest proof-of-work.  In the design it was necessary for it to switch to a longer chain no matter how far back it has to go.
%
% The only exception to that is the manual checkpoints I've added.  If it weren't for those, it would be able to reorg all the way back to the first block."

\section*{La preuve d'enjeu}
\addcontentsline{toc}{section}{La preuve d'enjeu}

% Définition de la preuve d'enjeu
La preuve d'enjeu, de l'anglais \eng{proof-of-stake} (PoS), est un procédé permettant à quelqu'un de démontrer son implication dans un système par le biais d'un algorithme de signature, dans le cadre de l'accès à un privilège. Elle intervient généralement dans le modèle de consensus des systèmes cryptoéconomiques gérant une unité de compte numérique. Les validateurs en charge de la confirmation des transactions sont alors sélectionnés selon le nombre d'unités qu'ils mettent en jeu (ou selon un autre paramètre lié).

% Minage virtuel
La preuve d'enjeu est parfois décrite comme du «~minage virtuel~» car les jetons numériques jouent le même rôle que l'énergie électrique dans les algorithmes basés sur la preuve de travail. Ainsi, dans de nombreuses cryptomonnaies, la probabilité de valider un bloc est proportionnelle au nombre de jetons en possession du validateur.

% Punition
Les jetons sont mis en jeu dans le sens où ils sont bloqués dans le protocole. De plus, ils peuvent être saisis en cas de comportement hostile au réseau, ce qui permet d'éviter le problème du «~rien à perdre~» (\eng{nothing-at-stake problem}) qui se poserait dans une l'implémentation naïve du procédé. En effet, si les validateurs peuvent valider plusieurs chaînes concurrentes en même temps, contrairement à la preuve de travail où l'énergie ne peut pas être dupliquée. Par exemple, l'algorithme de consensus d'Ethereum, Casper-FFG\sendnote{Vitalik Buterin et al., \eng{Combining GHOST and Casper}, 11 mai 2020~: \url{https://arxiv.org/pdf/2003.03052.pdf}.}), met en place une «~coupe des fonds~» (ou \eng{slashing}\sendnote{Vitalik Buterin, \eng{Slasher: A Punitive Proof-of-Stake Algorithm}, 15 janvier 2014~: \url{https://blog.ethereum.org/2014/01/15/slasher-a-punitive-proof-of-stake-algorithm}.}) pour sanctionner progressivement les validateurs qui ne respectent pas le protocole. Cela permet au réseau de se prémunir contre les attaques de courte portée.

%
De plus la preuve d'enjeu étant subjective, il est nécessaire d'y apporter une réponse. D'où l'utilisation de points de contrôles pour contrer les attaques de longue portée, ce qui crée le concept d'«~époques~».

% La complexité des algorithmes qui en dérivent, notamment induite par la résolution du problème du «~rien en jeu~» (nothing at stake). Cette complexité rend les modèles plus difficiles à analyser que celui de la preuve de travail.

% Apparition
L'idée de la preuve d'enjeu est une vieille idée puisqu'on la retrouve dans la conception de b-money, un système imaginé par le cypherpunk Wei Dai en 1998. Dans son modèle, chaque serveur devait déposer un certain montant de b-money sur un compte spécial pour participer aux opérations du réseau. Le montant servait de garantie pour pénaliser le serveur en cas de mauvaise conduite.

% Invention du terme
Le terme \eng{proof-of-stake} a été inventé par Sunny King et Scott Nadal, qui ont adapté le concept aux systèmes cryptomonétaire en août 2012 dans le livre blanc de PPCoin\sendnote{Sunny King et Scott Nadal, \eng{PPCoin: Peer-to-Peer Crypto-Currency with Proof-of-Stake}, 19 août 2012~: \url{https://web.archive.org/web/20121021014644/http://www.ppcoin.org/static/ppcoin-paper.pdf}.}. PPCoin, aujourd'hui connu sous le nom de Peercoin, mettait en effet en place un modèle hybride combinant énergie électrique et âge des pièces (preuve de conservation) pour sa validation.

% Variantes
De même que la preuve de travail peut être étendue en preuve de mémoire, la preuve d'enjeu peut être dérivée en plusieurs variantes. La preuve d'enjeu déléguée\sendnote{Dan Larimer, \eng{DPOS Consensus Algorithm - The Missing White Paper}, 29 mai 2017~: \url{https://hive.blog/dpos/@dantheman/dpos-consensus-algorithm-this-missing-white-paper}.} prend ainsi en compte les jetons possédés mais aussi les jetons délégués au validateurs. Il s'agit de la variante la plus répandue. Elle permet de mettre en place une preuve d'enjeu liquide (à la Tezos\sendnote{Jacob Arluck, \eng{Liquid Proof-of-Stake}, 30 juillet 2018~: \url{https://medium.com/tezos/liquid-proof-of-stake-aec2f7ef1da7}.}), mais a néanmoins pour inconvénient de centraliser la validation. Il existe également d'autres variantes comme la preuve de conservation (Peercoin), la preuve de vélocité (Reddcoin) ou la preuve d'importance (NEM).

% Deux catégories de preuve
De manière générale, on peut regrouper les mécanismes de résistance aux attaques Sybil des systèmes ouverts en deux catégories de preuve~: les preuves externes, basées sur l'utilisation de l'énergie dans le monde physique, et les preuves internes, basées sur l'état du registre virtuel. Il y a ainsi une auto-référence dans le cas de la preuve d'enjeu, ce qui pose problème.

% --- Résistance à la censure des deux types de preuve ---

Les défenseurs de la preuve d'enjeu prétendent que la preuve d'enjeu est plus sécurisée, car une attaque est plus coûteuse d'un ordre de grandeur\sendnote{\url{https://ethereum.org/en/developers/docs/consensus-mechanisms/pos/pos-vs-pow/#security}}. Une attaque de censure pourrait en outre faire baisser le prix de l'unité de compte, ce qui provoquerait une baisse de valeur du capital de l'attaquant. Nous affirmons l'inverse~: la preuve d'enjeu offre une résistance à la censure moins grande que la preuve de travail.

% Accumulation des jetons mis en jeu
Tout d'abord, réunir les jetons nécessaires est loin d'être une tâche impossible. Premièrement, tous les détenteurs ne sont pas impliqués dans le consensus, ce qui veut dire que seule la portion des jetons mis en jeu est concernée. Deuxièmement, contrairement à la preuve de travail qui exige 51~\% de la puissance de calcul pour perturber le système, le plupart des algorithmes par preuve d'enjeu sont des algorithmes classiques dont l'attaque ne nécessite que 34~\% des fonds en jeu. Troisièmement, une grande partie des jetons sont conservés par des acteurs centralisés qui offrent généralement des services de \eng{staking} (incitant l'accumulation) et qui sont réglementés et donc particulièrement sensibles à la cooptation étatique.

% Meilleure identification du validateur
Ensuite, un défaut de la preuve d'enjeu est qu'elle permet une meilleure identification du validateur (clé publique liée aux fonds sous séquestre), que dans le cas de la preuve de travail, où les mineurs peuvent diriger leur puissance de calcul vers la chaîne libre plus discrètement. La validation par preuve d'enjeu est donc moins confidentielle que le minage qui l'est complètement, ce qui pose problème dans la première étape de menace.

Enfin, et surtout la principale raison est le caractère interne de la preuve. Dans le cas de la preuve de travail, il est toujours possible de combattre la censure~: il suffit de réunir une puissance de calcul supérieure au censeur, en construisant des machines et en apportant une nouvelle énergie. Dans le cas de la preuve d'enjeu, il n'est pas possible de créer de nouvelles unités sans modifier les règles de consensus de sorte que le censeur, qui contrôle une majorité des jetons existants et touche par conséquent une majorité des jetons créés, est intouchable.

% --- Slashing social ---

Pour répondre à ce problème, les partisans de la preuve d'enjeu sur Ethereum prônent généralement le recours à l'accord social. Il ne s'agit pas seulement de sélectionner la chaîne valide manuellement comme on l'a vu précédemment, mais de rééquilibrer la distribution des jetons de façon à retrouver un système de validation qui ne censure pas. Cette mesure peut consister à créer des unités supplémentaires, mais alors se poserait la question de leur destination. Elle consiste donc plutôt à la destruction des jetons mis en jeu par le censeur, ce qui est appelé le \eng{slashing} social\sendnote{Eric Wall, \eng{The Case for Social Slashing}, 22 août 2022~: \url{https://ercwl.medium.com/the-case-for-social-slashing-59277ff4d9c7}.}.

Ce recours est notamment soutenu par Vitalik Buterin, qui écrivait par exemple en 2020 la chose suivante~:

\begin{quote}
«~Pour d'autres attaques plus difficiles à détecter (notamment une coalition de 51~\% censurant tous les autres), la communauté peut se coordonner pour réaliser un soft fork activé par les utilisateurs (UASF) minoritaire dans lequel les fonds de l'attaquant sont [...] largement détruits (dans Ethereum, cela se fait via le "mécanisme de fuite d'inactivité"). Aucun "hard fork pour supprimer les pièces" explicite n'est nécessaire~; à l'exception de la nécessité de coordonner l'UASF pour sélectionner un bloc minoritaire, tout le reste est automatisé et suit simplement l'exécution des règles du protocole.\sendnote{Vitalik Buterin, \eng{Why Proof of Stake (Nov 2020)}, 6 novembre 2020~: \url{https://vitalik.ca/general/2020/11/06/pos2020.html}.}~»
\end{quote}

% "For other, harder-to-detect attacks (notably, a 51% coalition censoring everyone else), the community can coordinate on a minority user-activated soft fork (UASF) in which the attacker's funds are [once again] largely destroyed (in Ethereum, this is done via the "inactivity leak mechanism"). No explicit "hard fork to delete coins" is required; with the exception of the requirement to coordinate on the UASF to select a minority block, everything else is automated and simply following the execution of the protocol rules."

\textcolor{darkgray}{Pour l'heure}, la mesure n'a jamais été appliquée sur Ethereum. Le cas qui s'en rapproche le plus est le contentieux entre la Fondation Tron de Justin Sun et la communauté historique de Steem qui s'est conclu par le gel des fonds de la première par une intervention externe de la communauté en mars 2020. Cette intervention a provoqué une scission entre le protocole Steem contrôlé par la Fondation Tron et la plateforme Hive\sendnote{Tim Copeland, \eng{Steem vs Tron: The rebellion against a cryptocurrency empire}, 18 août 2020~: \url{https://decrypt.co/38050/steem-steemit-tron-justin-sun-cryptocurrency-war}.}.

Le recours à l'accord social paraît une nouvelle fois être une bonne idée. Cependant, il s'agit clairement de jouer avec le feu~: le risque de créer la confusion et de provoquer une scission est largement sous-estimé.

De manière générale, c'est ce qui différencie la philosophie derrière la preuve d'enjeu de celle de la preuve de travail. Les défenseurs de la preuve d'enjeu ne modélisent pas la menace de la même manière. C'est pourquoi le modèle de sécurité de Bitcoin est bien plus exigeant que celui d'Ethereum.

% Comme l'écrivait Vitalik Buterin en 2014~:
%
% \begin{quote}
% «~Un hypothétique gouvernement oppressif suffisamment puissant pour semer la confusion sur la valeur réelle d'une empreinte de bloc datant d'un an serait également assez puissant pour vaincre tout algorithme de preuve de travail, ou pour semer la confusion à propos des règles du protocole de la chaîne de blocs.\sendnote{Vitalik Buterin, \eng{Proof of Stake: How I Learned to Love Weak Subjectivity}, 25 novembre 2014~: \url{https://blog.ethereum.org/2014/11/25/proof-stake-learned-love-weak-subjectivity}.}~»
% \end{quote} % "A hypothetical oppressive government which is powerful enough to actually cause confusion over the true value of a block hash from one year ago would also be powerful enough to overpower any proof of work algorithm, or cause confusion about the rules of blockchain protocol."

% Consommation d'énergie de la preuve d'enjeu réellement réduite ?

\section*{Consommation d'énergie et résistance à la censure}
\addcontentsline{toc}{section}{Consommation d'énergie et résistance à la censure}

Ainsi, la preuve de travail joue un rôle essentiel dans la résistance à la censure de Bitcoin. C'est tout le génie de Nakamoto~: avoir créé un système de consensus permettant l'objectivité et permettant donc de déterminer, même en présence d'un appareil de censure énorme.

Il s'avère que la mise en œuvre de cette preuve de travail consomme une importante quantité d'énergie électrique. Mais c'est cette consommation qui ancre le protocole dans le réel et elle est donc le prix à payer pour disposer d'un système réellement résistant à la censure. Elle ne peut donc pas être évitée.

La consommation d'énergie est l'un des arguments d'opposition à Bitcoin les plus récurrents, en raison de son supposé impact écologique\sendnote{La première critique de la consommation d'énergie de Bitcoin a été faite par l'ancien cypherpunk John Gilmore en janvier 2009~:

\begin{quote}
\footnotesize «~La dernière chose dont nous avons besoin est de déployer un système conçu pour brûler tous les cycles disponibles, consommant de l'électricité et générant du dioxyde de carbone, partout sur internet, afin de produire de petites quantités de dollars binaires pour faire passer des courriels ou des spams.~» (John Gilmore, \eng{Proof of Work -> atmospheric carbon}, \wtime{25/01/2009 22:40:45}~: \url{https://www.metzdowd.com/pipermail/cryptography/2009-January/015042.html})
\end{quote}

Cette préoccupation a conduit Hal Finney à écrire son troisième et dernier tweet sur Bitcoin~: «~Réfléchis à la manière de réduire les émissions de CO\textsubscript{2} que produiraient une mise en œuvre généralisée de Bitcoin.~» (Hal Finney, Twitter, \wtime{27/01/2009 20:14 UTC}~: \url{https://twitter.com/halfin/status/1153096538})}. Au vu de ce que nous avons dit ici, on peut assimiler cette opposition à une opposition à la résistance à la censure financière. La meilleure façon de réduire l'énergie consommée par Bitcoin, c'est de prôner la concurrence monétaire et bancaire afin qu'il devienne inutile.

L'abandon de la preuve de travail, déjà proposé par le passé\sendnote{Tyler Kruse, \eng{Change The Code: Not The Climate –- Greenpeace USA, EWG, Others Launch Campaign to Push Bitcoin to Reduce Climate Pollution}, 29 mars 2022~: \url{https://www.greenpeace.org/usa/news/change-the-code-not-the-climate-greenpeace-usa-ewg-others-launch-campaign-to-push-bitcoin-to-reduce-climate-pollution/}.}, pourrait donc être fatal pour Bitcoin. Cependant, Bitcoin dispose également d'un mécanisme de défense à ce niveau-là. Dans le chapitre suivant, nous verrons comment le protocole est déterminé par l'accord social.

\printendnotes
