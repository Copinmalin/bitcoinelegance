% Copyright (c) 2022 Ludovic Lars
% This work is licensed under the CC BY-NC-SA 4.0 International License

\chapter{Une guerre technologique}
\label{ch:cypherpunks}

\textcolor{green}{[Parler des techniques antérieures à Bitcoin en les rattachant à Bitcoin et aux déclarations de Satoshi] Communication, cryptographie, réseau d'ordinateurs}

La technologie est, étymologiquement, l'étude des techniques. Une technique est une méthode ou un ensemble de méthodes. Un outil (capital). % technique = ensemble des pratiques dont le but est la production d'objets destinés à satisfaire des besoins déterminés

L'évolution technique influence profondément notre société, notre manière d'interagir avec les autres êtres humains. L'émergence de l'imprimerie en Europe au \textsc{xv}\ieme{}~siècle a provoqué la réforme protestante au siècle suivant. La révolution industrielle, qui reposait sur l'invention de nouvelles machines, a donné sa force de base au socialisme technocratique. Et le développement de l'ordinateur est en train de nous modifier notre culture comme jamais auparavant.

Une technique n'est jamais neutre en soi. Elle a toujours une influence bonne ou mauvaise dans toutes les conséquences qu'elle peut avoir. Pourrait-on vraiment ne pas classer la bombe atomique... Une technologie peut être utilisée pour le bien ou pour le mal, pour libérer l'individu ou pour l'opprimer. Elle caractérise notre époque moderne.

La guerre qui nous intéresse est la guerre monétaire qui a lieu entre l'individu et l'État, qui souhaitent tous les deux avoir le contrôle sur leur monnaie. La guerre ne se joue plus au niveau des métaux précieux ou des billets de banque, qui peuvent conserver une certaine utilité. L'heure est à la monnaie numérique, qui est bien plus adaptée à nos moyens de communication et d'échange économique modernes. Satoshi en était conscient quand il déclarait dès novembre 2008 que Bitcoin pourrait permettre de «~remporter une bataille majeure dans la course aux armements~» et de «~conquérir un nouveau territoire de liberté pour plusieurs années~».

% Satoshi déclarait en novembre 2008 que Bitcoin pouvait permettre de «~remporter une bataille majeure dans la course aux armements~» et de «~conquérir un nouveau territoire de liberté pour plusieurs années~».

Bitcoin est fondé sur une longue liste de techniques modernes, en particulier la cryptographie. Il n'est pas sorti de nulle part~: il est le résultat de décennies de réflexions, de recherches et d'expérimentations. Examinons d'un peu plus près ces techniques et les idées qui leur sont liées.

\section{La télécommunication, la cryptographie et l'ordinateur}

\textcolor{green}{Télécommunications (transmission d'informations à distance) et cryptographie (sécurisation de la communication en présence de tiers malveillants).}

% --- Communication ---

Pour remonter dans le temps, il faut parler de la communication. La communication est le fait de transmettre des informations à autrui. Cette communication a longtemps été restreinte, du fait des limitations spatiales qui caractérisent. On communiquait beaucoup plus avec le voisin qu'avec le lointain, d'où l'existence de langues et de cultures différentes.

Toutefois, l'évolution technique a changé les choses. La télécommunication, ou la transmission d'information à distance (le préfixe télé- vient du grec ancien \foreignlanguage{greek}{t{\~h}le}, t{\~e}le, «~loin~»), s'est largement développée à partir de la fin du \textsc{xix}\ieme{}~siècle. On a d'abord vu apparaître le télégraphe électrique, qui permettait d'envoyer et de recevoir des messages écrits, appelés télégrammes, d'une manière rapide et fiable, à l'aide de codes. Puis, le téléphone, qui donnait la possibilité de transmettre des messages vocaux à distance. Ensuite, la radiocommunication qui fait usage des ondes radioélectriques pour transmettre de la l'information. Et enfin, la télévision (images) et le réseau Internet.

% --- Cryptographie ---

Cette évolution de la communication a créé le besoin de sécuriser l'information transmise. C'est pourquoi la cryptographie, qui existait déjà depuis l'Antiquité, a connu un essor sans précédent au cours du \textsc{xx}\ieme{}~siècle.

La cryptographie est la discipline mathématique qui a pour but la sécurisation de la communication en présence de tiers malveillants. À l'origine, il s'agit de dissimuler de l'information par une méthode de chiffrement, ce qui explique le mot, qui vient du grec ancien \foreignlanguage{greek}{kruptós}, kruptós («~caché~») et \foreignlanguage{greek}{gráfw}, gráphô («~écrire~»). Par la suite, la cryptographie s'est étendue à l'authentification de l'auteur d'un message avec la signature numérique et à la vérification de les données notamment par l'usage de fonctions de hachage. En résumé, cette discipline permet d'assurer la confidentialité (chiffrement), l'authenticité (signature) et l'intégrité (hachage) de l'information transmise.

% Pratique et étude des techniques de communication sécurisée en présence d'un comportement antagoniste

% Adversaire
La cryptographie porte ainsi en elle la notion d'adversaire ou d'antagoniste (de l'anglais \eng{adversary}) qui est une entité malveillante dont le but est d'empêcher les utilisateurs d'un cryptosystème de réaliser leurs buts. Il n'y a en effet pas besoin de cacher, d'authentifier ou de vérifier quoi que ce soit s'il n'y a pas de menace qui pourrait tirer profit de l'absence d'application de ces méthodes. C'est pourquoi le chiffrement s'est révélé particulièrement utile dans la guerre et c'est pourquoi il a été développé par les États.

% Chiffrement symétrique, ou chiffrement à clé secrète
Le chiffrement est un procédé par lequel on rend la compréhension d'un message impossible pour les personnes qui ne disposent pas d'une information spécifique, appelée une clé. Le chiffrement était initialement symétrique, c'est-à-dire que la clé de chiffrement et de déchiffrement étaient les mêmes et que les deux parties devaient avoir connaissance de cette clé secrète pour communiquer.

L'exemple typique de chiffrement symétrique est le code de César, ou chiffrement par décalage, qui est l'une des méthodes les plus simples et les plus connues pour chiffrer un texte. Le texte chiffré s'obtient en remplaçant chaque lettre du texte clair original par une lettre à distance fixe, toujours du même côté, dans l'ordre de l'alphabet. La clé est alors le nombre correspondant au décalage. Par exemple, un décalage de 21 lettres transforme le mot «~bitcoin~» en «~wdoxjdi~». Cette méthode tient son nom du fait que Jules César l'utilisait dans ses correspondances secrètes.

Le chiffrement symétrique posait néanmoins un problème logistique. La clé devait en effet être transmise entre les deux parties qui communiquaient et pouvait donc être interceptée. C'est pour cela que le chiffrement asymétrique, apparu plus tard, était révolutionnaire.

---





% Cryptanalyse
En particulier, dans le cas du chiffrement, il s'agit de déchiffrer des messages sans disposer de la clé de chiffrement, ce qui s'appelle la cryptanalyse.

% Ordinateurs
La cryptographie et la cryptanalyse ont motivé le développement des premiers ordinateurs au sens moderne du terme. Des machines de chiffrement et de déchiffrement ont ainsi été mises au point après la Grande Guerre (?), dans le but de communiquer sur le champ de bataille.


Cryptographie à l'origine du développement des premiers ordinateurs, spécialisés dans la cryptanalyse (machine de Turing).

Ordinateur.


\textcolor{red}{[article : préhistoire de bitcoin]}

\textcolor{gray}{Les machines à calculer existent depuis des siècles, mais l'informatique moderne ne remonte elle qu'à 1936, année lors de laquelle le mathématicien Alan Turing pose les bases théoriques de ce qu'est un ordinateur, avec son concept de machine universelle de Turing.}

\sendnote{Alan Turing, \eng{On Computable Numbers, with an Application to the Entscheidungsproblem}, 28 mai 1936~: \url{https://www.cs.virginia.edu/~robins/Turing_Paper_1936.pdf}.}

1938~: Torpedo Data Computer (États-Unis)~; 1939~: Z2 (Allemagne).

Alan Turing, qui avait contribué en 1936 à poser les bases théoriques de l'informatique ordinateur, avec son concept de machine universelle de Turing

\textcolor{gray}{Après la Seconde Guerre mondiale, les ordinateurs deviennent progressivement de plus en plus efficaces grâce à l'invention du transistor et du circuit intégré. Cela débouche finalement, au cours des années 1970, sur l'apparition de l'ordinateur personnel (personal computer en anglais), ordinateur destiné à l'usage d'une personne et dont les dimensions sont assez réduites pour tenir sur un bureau. Ce type d'ordinateur donne de cette manière la possibilité à chacun de concevoir et d'exécuter des programmes. L'exemple le plus célèbre est sans doute l'Apple II, conçu par Steve Wozniak et sorti en 1977 qui est le premier ordinateur personnel fabriqué à grande échelle.}

\textcolor{gray}{Grâce à cela, se développent également les systèmes d'exploitation standards. Unix, qui est présenté par AT\&T au public pour la première fois en 1973, rencontre un franc succès, à tel point qu'il servira de base à des systèmes d'aujourd'hui comme GNU/Linux et macOS. De même, DOS (l'ancêtre de Windows) est créé en 1981.}

\textcolor{gray}{Tout ceci est révolutionnaire pour l'époque, et de nombreuses innovations viendront de cette possibilité pour les individus de programmer des machines. Cette révolution marque aussi le début de la culture des hackers, ces individus désireux de détourner les systèmes informatiques de leur rôle prédéfini.}

---

% Technique (du grec ancien \foreignlanguage{greek}{téknh}, tékhnê signifiant «~art, métier, profession~» et désignant par extension l'habileté dans ces activités).
%
% Télégraphe (du grec ancien \foreignlanguage{greek}{t{\~h}le}, t{\~e}le, «~loin~» et \foreignlanguage{greek}{gráfw}, gráphô, «~écrire~») est un système destiné à transmettre des messages, appelés télégrammes, d'un point à un autre sur de grandes distances, à l'aide de codes pour une transmission rapide et fiable. Télégraphe optique depuis l'Antiquité. Télégraphe électrique au \textsc{xix}\ieme{}~siècle~: télégraphe de Cooke et Wheatstone (1837 / 1838).
%
% Téléphone (du grec ancien \foreignlanguage{greek}{t{\~h}le}, t{\~e}le, «~loin~» et \foreignlanguage{greek}{fwn{\'h}}, phôn{\~e}, «~voix~»). Fin du \textsc{xix}\ieme{}~siècle, début du \textsc{xx}\ieme{}~siècle.





\section{Le chiffrement asymétrique, la signature numérique et le hachage}

% --- Chiffrement asymétrique ---

% Chiffrement asymétrique, chiffrement à clé publique
Le chiffrement peut également être asymétrique, auquel cas deux clés différentes interviennent~: une clé privée et une clé publique. La clé de chiffrement est la clé publique, qui peut être connue de tous. La clé de déchiffrement est la clé privée, qui doit rester secrète. Le destinataire génère une paire de clés, garde la clé privée pour lui et partage la clé publique à son interlocuteur pour qu'il lui envoie des messages. La clé privée ne peut pas être retrouvée à partir de la clé publique, ce qui garantit la sécurité du procédé.

La méthode asymétrique la plus connue est le chiffrement RSA, créé en 1977 par Ronald Rivest, Adi Shamir et Leonard Adleman. Celui-ci se base sur des opérations algébriques permettant de chiffrer et déchiffrer un message. Sa sécurité provient de la difficulté à factoriser de très grands nombres en nombres premiers. Cet algorithme est utilisé très largement sur Internet, et en particulier dans le commerce électronique.

À aucun moment, la clé privée n'est révélée ce qui permet de minimiser le risque de fuite par rapport au chiffrement symétrique. Le chiffrement à clé publique constitue la première partie de la cryptographie asymétrique, l'autre partie étant la signature numérique.

\textcolor{gray}{L'émergence des ordinateurs a aussi permis l'apparition de nouvelles techniques cryptographiques. Jusqu'ici limités par les capacités cognitives de l'être humain, les systèmes cryptographiques ont pu grâce aux ordinateurs se reposer sur leur puissance de calcul.}

\textcolor{gray}{Tout commence en 1976 lorsque la cryptographie asymétrique (se basant sur la distinction entre une clé privée et une clé publique) est décrite par Whitfield Diffie et Martin Hellman. Avec celle-ci, on voit apparaître par la suite le concept de signature numérique, qui sera central dans Bitcoin : ~; l'algorithme de signature .}

1976, Diffie et Hellman, \eng{New Directions in Cryptography}~: \url{https://ee.stanford.edu/~hellman/publications/24.pdf}.

1978, Rivest, Shamir, Adleman, \eng{A Method for Obtaining Digital Signatures and Public-Key Cryptosystems}~: \url{https://people.csail.mit.edu/rivest/Rsapaper.pdf} $\implies$ RSA.

Découverte de manière indépendante par Ralph Merkle\sendnote{\url{https://www.ralphmerkle.com/1974/}}.

1985, Neal Koblitz, \eng{Elliptic Curve Cryptosystems} $\implies$ Cryptographie sur courbes elliptiques

% --- Signature numérique ---

la cryptographie sur courbes elliptiques apparaît en 1985

DSA est proposé en 1991. ECDSA en 1992.

% --- Hachage ---

\textcolor{gray}{À la même époque (1979), Ralph Merkle invente l'arbre de hachage (aujourd'hui appelé arbre de Merkle), qui est une structure de données permettant de pouvoir vérifier l'intégrité de l'ensemble sans les avoir nécessairement toutes au moment de la vérification. Les arbres de Merkle sont utilisés pour organiser les transactions au sein des blocs, et permettent aux portefeuilles légers de procéder à la vérification de leur paiement sans reposer entièrement sur un tiers de confiance.}

\sendnote{Ralph C. Merkle, \eng{Protocols for public key cryptosystems}, 1980~: \url{https://www.ralphmerkle.com/papers/Protocols.pdf}.}

Entre 1989 et 1991, plusieurs algorithmes de hachage (MD2, MD4, MD5) sont publiés. Ils sont créés par Ronald Rivest pour le MIT. L'algorihtme SHA-0 est créé en 1993. SHA-1 en 1995, et la

% --- Applications ---

Toutes ces découvertes inspirent les esprits curieux.

% David Chaum
Les cypherpunks ont été précédé par David Chaum (précurseur des cypherpunks). Les contributions de David Chaum dans le domaine de la cryptographie et des monnaies numériques font de lui un précurseur des cypherpunks, dont le mouvement n'a été créé qu'une décennie plus tard, en 1992. De par sa conception du respect de la vie privée, il préfigurait l'obsession des crypto-anarchistes pour l'anonymat, sans pour autant faire preuve de la même radicalité qu'eux...


\textcolor{red}{article eCash David Chaum}

\textcolor{gray}{DAVID CHAUM, LE PRÉCURSEUR DES CYPHERPUNKS David Chaum est un informaticien et cryptographe américain né en 1955. Il est connu pour avoir été un pionnier de la recherche sur les communications anonymes et pour avoir été le premier à conceptualiser l'argent liquide numérique.}

\textcolor{gray}{En 1981, David Chaum publie l'article "Untraceable Electronic Mail, Return Addresses, and Digital Pseudonyms" dans lequel il pose les fondations pour la recherche dans le domaine des communications anonymes. Basé sur la cryptographie asymétrique, la méthode décrite dans ce papier est destinée à être appliquée au courrier électronique ou aux élections.}

\textcolor{gray}{Un an après, il participe à la fondation de l'International Association for Cryptologic Research (IACR) lors de la conférence annuelle sur la cryptologie CRYPTO '82. Cette dernière aura pour rôle de soutenir la recherche dans le domaine en organisant des conférences et en éditant le Journal of Cryptology.}

\textcolor{gray}{Pendant l'année 1982, David Chaum publie aussi un papier intitulé "Blind signatures for untraceable payments" dans lequel il formalise un concept de monnaie électronique anonyme, concept qui deviendra par la suite la base théorique de eCash. Ce papier décrit aussi un procédé de signature aveugle (blind signature), dont les applications vont au-delà du simple paiement : aujourd'hui, ce procédé est notamment utilisé au sein de protocoles de mélange de bitcoins comme le protocole ZeroLink implémenté dans le portefeuille Wasabi.}

\textcolor{gray}{Ainsi, les contributions de David Chaum dans le domaine de la cryptographie et des monnaies numériques font de lui un précurseur des cypherpunks, dont le mouvement ne sera créé qu'une décennie plus tard, en 1992. De par sa conception du respect de la vie privée, il préfigurait l'obsession des crypto-anarchistes pour l'anonymat, sans pour autant faire preuve de la même radicalité qu'eux. De plus, tout comme les cypherpunks, il a "écrit du code" pour appliquer ses idées au monde réel, notamment au travers de son entreprise, DigiCash, qui a mis en place eCash.}

DC-nets.

David Chaum, 25 juillet 1995~:

\begin{quote}
«~Les "technologies de confidentialité" permettent aux personnes de protéger leurs propres informations et leurs autres intérêts, tout en maintenant une sécurité très élevée pour les organisations. Il s'agit essentiellement de la différence entre, d'une part, un système centralisé dont les participants sont privés de leurs droits (comme les animaux marqués électroniquement dans les parcs d'engraissement) et, d'autre part, un système dans lequel chaque participant est en mesure de protéger ses propres intérêts (comme les acheteurs et les vendeurs sur une place de marché).\sendnote{David Chaum, \emph{Testimony for US House of Representatives}, 25 juillet 1995, \url{https://web.archive.org/web/19970111170802/digicash.com/publish/testimony.html}.}~»
\end{quote}

% Philip Zimmermann
Philip Zimmermann. 1991~: \eng{Pretty Good Privacy}. Tout comme Chaum, il est resté à l'écart des cypherpunks.

% Stuart Haber et Scott Stornetta
\textcolor{gray}{L'horodatage de documents, utilisé dans Bitcoin pour lier les blocs les uns aux autres par références récursives, est décrit en 1991 par Stuart Haber et Scott Stornetta dans un article intitulé \eng{How to time-stamp a digital document}, article qui sera cité en 2008 dans le livre blanc. En 1995, les deux hommes mettent en œuvre leur idée en publiant chaque semaine une empreinte cryptographique dans le New York Times afin d'authentifier les documents des clients de leur société, baptisée Surety.}

\section{Internet, le partage d'informations et le logiciel libre}

% --- Internet ---

Internet (Arpanet) conçu pour résister aux attaques nucléaires dans le cadre de la Guerre froide.

Mai 1974, Cerf and Kahn, \eng{A Protocol for Packet Network Intercommunication}~: \url{https://www.cs.princeton.edu/courses/archive/fall06/cos561/papers/cerf74.pdf} $\implies$ Internet Protocol. IPv4 en 1981.

\eng{Advanced Research Projects Agency} (ARPA), «~Agence pour les projets de recherche avancée~», 1958 -- 1972, 1993 -- 1996

\eng{Defense Advanced Research Projects Agency} (DARPA) «~Agence pour les projets de recherche avancée de défense~», 1972 -- 1993, 1996


Réseau informatique, permettant aux utilisateurs de différents ordinateurs de communiquer. Internet = réseau de réseaux.

Usenet (1980), lancement sur Internet via NNTP (1986). Cité par Satoshi dans le livre blanc et dans ses différents messages.

Xanadu. Ted Nelson\sendnote{Ted Nelson a affirmé en 2013 que Satoshi était Shinichi Mochizuki, un mathématicien japonais~: \url{https://www.youtube.com/watch?v=emDJTGTrEm0}.}, créateur de l'hypertexte.

Internet Relay Chat (IRC) : 1988. un protocole de communication textuel sur Internet. Son utilisation diminuera à la fin des années 1990 avec l'apparition des messageries instantanées propriétaires comme MSN Messenger.

World Wide Web, le Web, la Toile. Cyberespace : terme forgé par William Gibson dans sa nouvelle \emph{Gravé sur Chrome} publiée en juillet 1982, le terme «~matrice~» est également utilisé en tant que synonyme.

\textcolor{red}{[article : préhistoire de bitcoin]}

\textcolor{gray}{Avec l'émergence des ordinateurs personnels, l'étape suivante a été logiquement de les relier entre eux pour communiquer. C'est ainsi que s'est développé Internet, le réseau des réseaux, qui rassemble actuellement des milliards d'ordinateurs.}

\textcolor{gray}{Internet est créé dans les années 1970 par l'intermédiaire du projet Arpanet géré par l'ARPA (aujourd'hui DARPA), une agence du département de la Défense des États-Unis se concentrant sur la recherche et le développement des nouvelles technologies destinées à un usage militaire. Il est lancé en 1969 dans le contexte de la guerre froide contre le bloc de l'Est. Il s'agit du premier réseau à transfert de paquets et il relie différentes universités des États-Unis. Il doit continuer à fonctionner en cas d'attaque. La suite de protocoles TCP/IP est créée en 1974 et est utilisée dans Arpanet à partir de 1983.}

\textcolor{gray}{Internet met néanmoins du temps à se démocratiser, et ce n'est qu'avec le développement de HTTP et du web au cours des années 1990 que le grand public commence à s'y intéresser. Cela débouchera à la fin du millénaire à une bulle financière appelée bulle technologique ou bulle Internet (et dot-com bubble par les anglophones), qui éclatera ensuite.}

% --- Logiciel libre ---

La notion de logiciel libre est décrite pour la première fois dans la première moitié des années 1980 par Richard Stallman~:

\begin{quote}
«~Premièrement, la liberté de copier un programme et de le redistribuer à vos voisins, qu'ils puissent ainsi l'utiliser aussi bien que vous. Deuxièmement, la liberté de modifier un programme, que vous puissiez le contrôler plutôt qu'il vous contrôle~; pour cela, le code doit vous être accessible.\sendnote{Richard M. Stallman, \eng{What is the Free Software Foundation?}, février 1986~: \url{https://www.gnu.org/bulletins/bull1.txt}.}~»
\end{quote}

Grandes entreprises (corporations), logiciels propriétaires.

Mouvement libriste. Projet GNU, lancé par Richard Stallman en 1983 au travers d'un courriel diffusé sur le newsgroup net.unix-wizards\sendnote{Richard Matthew Stallman, \eng{new UNIX implementation}, 27 septembre 1983~: \url{https://groups.google.com/g/net.unix-wizards/c/8twfRPM79u0/m/1xlglzrWrU0J}}, objectif de développer un système d'exploitation entièrement libre.

On distingue deux types de licences libres~: les licences permissives et les licences contaminantes dites \eng{copyleft}.

Licence MIT. Massachusetts Institute of Technology, X Window System, Expat XML parser library. Développée sous forme embryonnaire à partir de 1985 (X version 6), créée en 1987 (X11), version actuelle en 1998 (Expat)\sendnote{Gordon Haff, \eng{The mysterious history of the MIT License}, 26 avril 2019~: \url{https://opensource.com/article/19/4/history-mit-license}.}.

Mais la licence MIT n'est pas parfaite. Elle est libre, mais n'est pas copyleft, contrairement à la licence GPL (GNU General Public License).

La licence GPL a été créée en février 1989\sendnote{Leonard H. Tower Jr., \eng{New General Public License}, 25 février 1989~: \url{https://groups.google.com/g/gnu.announce/c/m0Jjj_64PeQ/m/8xL1xkVKJb8J?pli=1}.}. Version 2 en juin 1991. Version 3 en juin 2007. Notion de copyleft («~gauche d'auteur~» par opposition au droit d'auteur).



\section{Les cypherpunks}
% Le crypto-anarchisme

Les cypherpunks sont des gens qui prônent l'utilisation proactive de la cryptographie en vue d'assurer la confidentialité et la liberté des individus dans le cyberespace. Ils s'opposent à la surveillance, à la censure et à l'exploitation des données personnelles. Ils promeuvent le logiciel libre. % Les cypherpunks ne sont pas nécessairement des anarchistes~: ce sont des rebelles, mais ils ne sont pas tous drastiquement opposés à l'existence même de l'État

% Favier, Takkal Bataille : les cypherpunks sont des «~gens opposés à l'autoritarisme des gouvernements, refusant la censure et la possibilité de censure, refusant la surveillance de masse, souhaitant conserver la propriété de leurs données personnelles, pensant que pour cela l'anonymisation de leurs correspondances, de leurs données et de leurs transactions est un droit, que les logiciels libres sont plus sûrs que les logiciels propriétaires, aux sources fermées, et qu'une monnaie libre par rapport aux États et aux banques est une chose désirable et utile~».

% Cyberpunk
L'idée cypherpunk est directement issue du cyberpunk, un mouvement de pensée...

Tout d'abord, l'influence exercée par la science fiction est prépondérante. Les cypherpunks ont en effet pour vocation de chercher à réfléchir et s'adapter aux conséquences du développement technologique. En particulier, le mouvement est issu du genre cyberpunk, un genre inauguré par le roman Neuromancien de William Gibson publié en 1984, dont la spécificité est de décrire un futur dystopique où la haute technologie est omniprésente (augmentation par implants, intelligence artificielle, robots) et où les acteurs de la société sont les sujets d'une décadence morale profonde, caractérisée par la consommation à outrance (drogue, sexe, etc.) et par l'avarice des corporations.

Mais les cypherpunks ne sont pas des cyberpunks~: s'ils ont bien conscience des scénarios dystopiques pouvant dériver de l'évolution technologique (notamment en ce qui concerne la surveillance), ils ne sont pas pour autant absolument pessimistes. De ce fait, le mouvement cypherpunk constitue en quelque sorte une réaction au cyberpunk, dans le sens où il postule que la technologie peut amener les êtres humains à s'émanciper plutôt qu'à tomber dans l'esclavage mutuel.

Les cypherpunks basent notamment leurs réflexions sur une longue nouvelle de Vernor Vinge, intitulée True Names et publiée en 1980. Celle-ci (qui aborde les mêmes thèmes que le genre cyberpunk, sans en être\sendnote{«~Je ne me considère certainement pas comme un écrivain cyberpunk. [...] Je pense que True Names partage un grand nombre d'idées techniques avec les récits cyberpunk. Je vois également deux différences substantielles entre True Names et les récits que l'on appelle généralement "cyberpunk". Premièrement, le cyberpunk montre davantage les aspects les plus sombres de la société future. Dans certains cas, il s'agit simplement d'un style dur à cuire. Dans d'autres cas, il capture la douleur d'un changement social très rapide. Deuxièmement, le cyberpunk est souvent pessimiste quant à la possibilité de changements sociaux et technologiques, ce qui peut en fait être une très bonne chose.~» -- Vernor Vinge (interrogé par Michael Synergy), «~Hurtling Towards the Singularity~», \emph{Mondo 2000 issue 1}, 1989~: \url{https://archive.org/details/Mondo.2000.Issue.01.1989/page/n115/mode/2up}.}) aborde les thèmes de l'anonymat, des mondes virtuels et de la singularité technologique. L'histoire suit Roger Pollack, un individu agissant au sein d'un groupe de pirates dans une réalité virtuelle appelée « The Other Plane », où il utilise le pseudonyme de Mr. Slippery. Il ne doit surtout pas révéler son « Vrai Nom » (son nom civil) au risque de subir une « Vraie Mort » (par exécution étatique), de sorte qu'il est crucial qu'il maintienne son anonymat. % "I certainly don't think of myself as a cyberpunk writer. A couple of months ago, I was asked whether I thought True Names was cyberpunk. I wrote a note that appeared in Science Fiction Lovers. Let me just read that to you: "I think there's a large set of tech ideas that True Names shares with cyberpunk stories. I also see two substantial differences between True Names and the stories that are usually called 'cyberpunk.' One: cyberpunk shows more of future society's ugly underbelly. In some cases this is simply hardboiled style. In some cases, it captures the pain of very fast social change. Two: cyberpunk is often pessimistic about the possibility of social and technological change, which may in fact be a wonderfully good thing."

Les cypherpunks reprennent en ceci la vision idéaliste de la Silicon Valley et en particulier des technophiles visionnaires des années 1980 qui, s'inspirant grandement de l'école autrichienne d'économie et en particulier de Friedrich Hayek, pensaient que l'informatique et la mise en réseau des ordinateurs pouvaient conduire à l'émergence d'un ordre spontané utopique. Ces « high-tech hayekians\sendnote{Don Lavoie, Howard Baetjer, William Tulloh, «~\eng{High-Tech Hayekians: Some Possible Research Topics in the Economics of Computation}~», \eng{Market Process}, vol. 8, 1990~: \url{http://www.philsalin.com/hth/hth.html}.}~» qui avaient pour projet des systèmes permettant de profiter de cette évolution.

---


% Silicon Valley
Le mouvement cypherpunk est né sur la côte Ouest des États-Unis et plus spécifiquement dans la Silicon Valley.

Timothy C. May. Science-fiction, physique. A travaillé pour Intel de 1974 à 1986. Problème des particules alpha dans les circuits intégrés. À l'âge de 35 ans, jugeant qu'il avait assez de capital pour survivre, il prenait sa retraite pour se consacrer à ses passions politiques. % Tim May est plus connu pour avoir résolu le Problème des Particules Alpha ce qui nuisait à la fiabilité des circuits intégrés lorsque l'appareil atteint une taille critique où une particule alpha unique pourrait changer l'état d'une valeur stockée et causer une perturbation isolée. "From 1974 to 1976 he was involved in reliability physics studies of MOS chips, especially problems related to the high-temperature glass-sealing of NMOS RAMs. In February 1977 he discovered that trace amounts of radioactive elements in device packages caused random, single-bit "soft errors." He then determined that the upset mechanism was the passage of alpha particles through memory storage nodes. Since 1977 he has researched various aspects of this phenomenon, including materials analysis, charge collection studies, and cosmic ray studies."
% Né en 1951



Timothy May, \emph{Manifeste crypto anarchiste}, écrit en août 1988 suite à des discussions sur les implications de la cryptographie avec Phil Salin, sa femme et d'autres personnes\sendnote{Timothy C. May, \eng{Cyphernomicon}, 16.3.4.}. A posé les bases de la doctrine~: pseudonymat, confidentialité, liberté d'expression et libre échange.

\begin{quote}
«~Tout comme la technologie de l'imprimerie a altéré et réduit le pouvoir des corporations médiévales et la structure sociale de pouvoir, les méthodes cryptologiques altèrent fondamentalement la nature de l'interférence de l'État et des grandes entreprises dans les transactions économiques.\sendnote{Timothy C. May, \eng{The Crypto Anarchist Manifesto}, \wtime{22/11/1992 20:11:24 UTC}~: \url{https://cypherpunks.venona.com/date/1992/11/msg00204.html}.}~»
\end{quote} % "Just as the technology of printing altered and reduced the power of medieval guilds and the social power structure, so too will cryptologic methods fundamentally alter the nature of corporations and of government interference in economic transactions."

% Eric Hughes
Eric Hughes, jeune mathématicien et programmeur ayant grandi près de Washington et à Salt Lake City. Il avait travaillé brièvement pour DigiCash à Amsterdam avant de revenir sur la côte Ouest\sendnote{Timothy C. May, \eng{Hackers Conference Report}, \wtime{11/11/1992 08:55:26 UTC}~: \url{https://cypherpunks.venona.com/date/1992/11/msg00019.html}.}.

% John Gilmore
John Gilmore, cinquième employé de Sun Microsystems, lui aussi en retrait anticipée. Il est libertarien, contributeur important pour le projet GNU. Il a co-fondé l'Electronic Frontier Foundation (EFF), une ONG internationale de protection des libertés sur Internet. Il est également le co-fondateur de Cygnus Support, une entreprise spécialisée dans le support professionnel de composants fondés sur GNU. % Né en 1955

% Judith Milhon
St. Jude (Judith Milhon). Activiste ayant participé au mouvement des droits civiques dans les années 1960 et ayant été emprisonnée pour désobéissance civile\sendnote{Sean Dodson, \eng{Judith Milhon: making the Internet a feminist issue}, 8 août 2003~: \url{https://www.theguardian.com/technology/2003/aug/08/guardianobituaries.obituaries}.}. Hackeuse. Le terme cypherpunk (de cypher, chiffre, et, punk, voyou), calqué sur le genre de science-fiction cyberpunk, a été créé de manière humoristique par St. Jude durant la première réunion du mouvement le 19 septembre 1992.

L'activiste Judith Milhon est également présente lors de cette réunion fondatrice. Née en 1939, elle a participé au mouvement des droits civiques pour l'abolition des discriminations raciales dans les années 60 et a été emprisonnée pour désobéissance civile. Programmeuse, hackeuse, connue sous le nom de plume de St. Jude, elle est alors co-éditrice de la revue cyberpunk Mondo 2000. Elle est également la compagne d'Eric Hughes, malgré leur grande différence d'âge.

C'est elle qui a donné leur nom aux cypherpunks, sur le ton de la plaisanterie. Il s'agit d'un mot-valise calqué sur le terme cyberpunk, et composé de \eng{cipher}, chiffre, et de \eng{punk}, voyou. L'appellation dérive directement de l'idée de la cryptoanarchie. «~Je pense que vous êtes des cryptoanarchistes -- ce que j'appellerais des cypherpunks~!~», écrira-t-elle\sendnote{Judith Milhon, \eng{secretions}, \wtime{25/09/1992 10:01:26 UTC}~: \url{https://cypherpunks.venona.com/date/1992/09/msg00013.html}.}.

% Création de la mailing list
À la suite de cette réunion, Eric Hughes, avec l'aide de Hugh Daniel, a créé une liste de diffusion de courrier électronique nommée «~cypherpunks~». Le courriel de bienvenue\sendnote{Eric Hughes, \eng{No Subject}, \wtime{22 Sep 92 05:43:46 UTC}~: \url{https://cypherpunks.venona.com/date/1992/09/msg00001.html}.} est envoyé dans la soirée du 21 septembre (PDT). La liste était relayée par le serveur associé au nom de domaine \texttt{toad.com} qui appartient à John Gilmore.

La mailing list a accueilli de nombreuses discussions relatives à la cryptographie et à son utilisation concrète, dont notamment l'argent liquide électronique. Beaucoup de gens sont intervenus dès les premiers mois, dont notamment des personnalités comme le pirate téléphonique John Draper ou bien le cryptographe Hal Finney, qui travaillait alors sur PGP avec Phil Zimmermann. En novembre 1992, ce dernier affirmait~:

\begin{quote}
«~Nous voici confrontés aux problèmes de la perte de confidentialité, de l'informatique trompeuse, des bases de données massives, de l'augmentation de la centralisation -- et Chaum propose une direction à suivre complètement différente, une direction qui met le pouvoir entre les mains des individus plutôt que celles des États et des grandes entreprises. L'ordinateur peut être utilisé comme un outil pour libérer et protéger les personnes, plutôt que pour les contrôler.\sendnote{Hal Finney, \eng{Why remailers...}, \wtime{16/11/1992 01:30:02 UTC}~: \url{https://cypherpunks.venona.com/date/1992/11/msg00108.html}.}~»
\end{quote}

%
% \textcolor{red}{article RPOW JDC}
%
% \textcolor{gray}{Harold Thomas Finney II, dit “Hal Finney”, était un informaticien et cryptographe américain, connu pour avoir participé au mouvement cypherpunk et pour avoir travaillé sur le logiciel de chiffrement PGP dans les années 1990, ainsi que pour avoir été le premier destinataire d'une transaction en bitcoins. De même que Wei Dai et Nick Szabo, il est l'une des personnes les plus suspectées d'être Satoshi Nakamoto en raison de ses capacités uniques et de son expérience dans le domaine des monnaies numériques.}
%
%
% \textcolor{gray}{En 1974, il intègre le prestigieux Institut de technologie de Californie (abrégé communément en Caltech) à Pasadena, où il suit des études de sciences de l'ingénieur. Dès 1978, Hal se met à travailler pour APh Technological Consulting, une petite firme qui se charge de conceptualiser et de maintenir le système d'exploitation de la console de jeux vidéos Intellivision pour Mattel, et qui conçoit également certains jeux pour cette console et pour l'Atari.}
%
% \textcolor{gray}{C'est aussi à la même époque qu'émerge Internet, avec Prodigy et surtout le web, que les Finney adoptent dès le début. Cet accès facilité à Internet permet à Hal Finney de rentrer en communication avec ceux qui pensent comme lui, et qui possèdent la même indépendance d'esprit. Il devient ainsi un cypherpunk dès le lancement du mouvement et participe énormément sur les listes de diffusion.}
%
% \textcolor{gray}{À la même époque, il rentre en contact avec Philip Zimmermann, le créateur du logiciel de chiffrement PGP (Pretty Good Privacy). Enthousiasmé par le projet, Hal Finney commence à l'aider et travaille sur la version 2 du programme, qui est publiée dans le monde entier en septembre 1992. En février 1993, le gouvernement fédéral américain ouvre une enquête contre PGP, car le logiciel viole la loi sur l'exportation des produits cryptographiques, considérés alors comme des munitions. Cette affaire montrera à Hal Finney de quels moyens de pression dispose un État quand il n'aime pas ce que font ses citoyens. À l'abandon des charges en 1996, Hal Finney sera l'un des premiers employés à rejoindre PGP Inc., l'entreprise nouvellement formée pour assurer le développement du logiciel, où il travaillera jusqu'à sa retraite en 2011.}


% Appel à la pratique
Les cypherpunks appellent à la pratique~: il ne sert à rien de croire à quelque chose si cela ne se traduit pas par une action dans le monde. Cette nécessité de la pratique a été illustrée par le \emph{Manifeste d'un Cypherpunk} publié sur la liste par Eric Hughes en mars 1993. Il écrivait alors~:

\begin{quote}
«~Nous devons défendre notre propre vie privée si nous voulons en avoir une. Nous devons nous rassembler et créer des systèmes qui rendent possibles les transactions anonymes. Depuis des siècles, les gens défendent leur vie privée par des chuchotements, par l'obscurité, par des enveloppes, des portes fermées, des poignées de main secrètes et des messagers. Les techniques du passé ne permettaient pas une forte confidentialité, mais les techniques électroniques le permettent.

Nous, les Cypherpunks, nous consacrons à construire des systèmes anonymes. Nous défendons notre confidentialité avec la cryptographie, avec les systèmes anonymes de transfert de courriels, avec les signatures numériques, et avec la monnaie électronique.

Les Cypherpunks écrivent du code. Nous savons que quelqu'un doit écrire un logiciel pour défendre la vie privée, et puisque nous ne pouvons pas avoir de vie privée si nous ne le faisons pas tous, nous allons l'écrire. Nous publions notre code pour que nos collègues Cypherpunks puissent le mettre en pratique et expérimenter avec. Notre code est libre d'utilisation pour tous, dans le monde entier. Nous ne nous soucions guère que vous n'approuviez pas les logiciels que nous écrivons. Nous savons que les logiciels ne peuvent pas être détruits et qu'un système largement dispersé ne peut pas être arrêté.\sendnote{Eric Hughes, \eng{RANTS: A Cypherpunk's Manifesto}, \wtime{17/03/1993 19:51:06 UTC}~: \url{https://cypherpunks.venona.com/date/1993/03/msg00392.html}.}~»
\end{quote} % "We must defend our own privacy if we expect to have any.  We must come together and create systems which allow anonymous transactions to take place.  People have been defending their own privacy for centuries with whispers, darkness, envelopes, closed doors, secret handshakes, and couriers.  The technologies of the past did not allow for strong privacy, but electronic technologies do.
%
% We the Cypherpunks are dedicated to building anonymous systems.  We are defending our privacy with cryptography, with anonymous mail forwarding systems, with digital signatures, and with electronic money.
%
% Cypherpunks write code.  We know that someone has to write software to defend privacy, and since we can't get privacy unless we all do, we're going to write it. We publish our code so that our fellow Cypherpunks may practice and play with it. Our code is free for all to use, worldwide.  We don't much care if you don't approve of the software we write.  We know that software can't be destroyed and that a widely dispersed system can't be shut down."

Timothy May, Eric Hughes et John Gilmore sont apparus masqués en couverture du magazine Wired en 1993. Deux mois plus tard, en mai 1993, c'est la consécration : les cypherpunks font la une du magazine Wired, récemment fondé dans le but de parler de l'incidence culturelle, économique et politique des technologies émergentes. Tim May, Eric Hughes et John Gilmore apparaissent masqués sur la couverture, et un long article détaille leurs principes et leurs revendications.



Les cypherpunks ne sont pas nécessairement des anarchistes~: ce sont des rebelles, mais ils ne sont pas tous drastiquement opposés à l'existence même de l'État. Par exemple Hal Finney ne l'est pas (\textcolor{red}{source ?}).

BlackNet\sendnote{Timothy C. May, \eng{no subject (file transmission)}, 17 août 1993, \url{https://cypherpunks.venona.com/date/1993/08/msg00538.html}.} : succède à l'AMIX de Phil Salin, préfigure Wikileaks (information) et Silk Road (produits). «~\eng{CryptoCredits}~» : monnaie interne du système.

Fuite de l'information, lanceur d'alerte. Cryptome, John Young, 1996. WikiLeaks, Julian Assange, 2006. A écrit dans la mailing list à partir de 1995, a écrit un livre sur es cypherpunks.

Mitch Kapor, John Gilmore, John Perry Barlow, \eng{Electronic Frontier Foundation} (1990)



\textbf{Guerre contre la cryptographie.} L'épisode des cypherpunks coïncide avec la guerre contre la cryptographie orchestrée par l'État fédéral...

Philip Zimmermann. PGP. Crypto Wars. Joe Biden's S.266, Comprehensive Counter-Terrorism Act of 1991, 102nd Congress (1991-1992) \url{https://www.congress.gov/bill/102nd-congress/senate-bill/266/text}

«~Le Congrès estime que les fournisseurs de services de communications électroniques et les fabricants d'équipements de services de communications électroniques doivent veiller à ce que les systèmes de communications permettent au gouvernement d'obtenir le contenu en texte clair des communications vocales, de données et autres lorsque la loi l'autorise de manière appropriée.~»
% "It is the sense of Congress that providers of electronic communications services and manufacturers of electronic communications service equipment shall ensure that communications systems permit the government to obtain the plain text contents of voice, data, and other communications when appropriately authorized by law."

En juin 1991, Phil Zimmermann publie la version 1.0 de PGP sous licence libre et la diffuse aux États-Unis par le biais de bulletin board systems et de Usenet. En raison de la nature ouverte d'Internet, PGP devient vite disponible dans le monde entier. Mais il s'avère que cela viole la Réglementation américaine sur le trafic d'armes au niveau international (International Traffic in Arms Regulations en anglais ou ITAR) qui considère les produits cryptographiques comme des munitions.

En février 1993, l'État fédéral américain ouvre par conséquent une enquête contre Zimmermann pour « exportation de munitions sans licence » et ce dernier est contacté par le Service des douanes et de la protection des frontières des États-Unis. Cela déclenche une forte réaction de la part des cypherpunks, qui se basent sur PGP pour communiquer. En particulier, le serveur de courriel anonyme mis en place par Eric Hughes et Hal Finney utilise le logiciel de Zimmermann.

En réponse à cette loi absurde, les cypherpunks se mettent à partager le code de chiffrement, dans une démarche de désobéissance civile. En particulier, ils mettent au point un code composé de trois lignes de Perl (en réalité 4 avec le commentaire) qui permet de chiffrer et de déchiffrer un message avec RSA, et qu'ils partagent dans leurs courriels à partir de 1995. Le code est le suivant~:

\begin{Verbatim}[fontsize=\small]
#!/bin/perl -s-- -export-a-crypto-system-sig -RSA-3-lines-PERL
$m=unpack(H.$w,$m."\0"x$w),$_=`echo "16do$w 2+4Oi0$d*-^1[d2%Sa
2/d0<X+d*La1=z\U$n%0]SX$k"[$m*]\EszlXx++p|dc`,s/^.|\W//g,print
pack('H*',$_)while read(STDIN,$m,($w=2*$d-1+length($n)&~1)/2)
\end{Verbatim}

Le jeune britannique Adam Back va même jusqu'à imprimer ces quelques lignes de code sur des t-shirts qu'il vend aux autres cypherpunks (certains sont même vendus contre des cyberbucks). Certains vont même jusqu'à se faire tatouer le code sur leur corps.

Philip Zimmermann publie la version 2.6.2 de PGP dans un livre \eng{PGP: Source Code and Internals}\sendnote{\url{https://philzimmermann.com/EN/essays/BookPreface.html}}, dans le but de réduire au maximum la distinction entre le code et l'expression, qui est protégée par le premier amendement de la Constitution des États-Unis).

Les charges contre Zimmermann sont finalement abandonnées en janvier 1996. Le 15 novembre 1996, Bill Clinton signe l'Executive Order 13026 qui assouplit considérablement les restrictions sur l'exportation des produits cryptographiques.

Clipper. À côté de cela, l'État fédéral veut imposer son propre standard cryptographique, ce qui se manifeste au travers de la puce Clipper, un cryptoprocesseur permettant de chiffrer les messages vocaux et les données, et destiné à être implémenté dans les appareils électroniques vendus au grand public. Cette puce est développé et produit par la NSA. Elle repose sur un algorithme de chiffrement appelé Skipjack (classé secret) qui implémente un dispositif d'autorité de séquestre, de telle sorte que les agences étasuniennes peuvent déchiffrer les communications chiffrés avec ce dispositif.

La puce Clipper est annoncée le 16 avril 1993 par la Maison-Blanche, qui justifie la conception par l'argument que la puce pourrait « à la fois fournir aux citoyens respectueux de la loi un accès au chiffrement dont ils ont besoin et empêcher les criminels de l'utiliser pour cacher leurs activités illégales ». Cela ne manque pas de faire réagir les cypherpunks, qui s'opposent en bloc à ce projet orwellien.

Cependant, la lutte n'est pas longue. En juin 1994, Matt Blaze découvre une vulnérabilité au sein du dispositif d'autorité de séquestre et prépublie un papier sur la liste. Cette vulnérabilité rend inefficace le dispositif et permet à la puce d'être utilisée pour chiffrer les données normalement. À partir de là, le projet perd progressivement en ampleur et est définitivement abandonné en 1996. Cela n'empêchera pas les agences étasuniennes d'espionner leur propre population de manière massive, comme le montreront les révélations d'Edward Snowden en 2013.



John Perry Barlow, \emph{Déclaration d'indépendance du cyberespace} (1996). Le cyberespace est une nouvelle juridiction~: ce qui s'y trouve n'est pas dans une juridiction particulière.



Adam Back, Hashcash, 1997. L'idée sera implémentée par Adam Back en 1997, au travers de Hashcash, toujours pour éviter le spam et les attaques par déni de service.

Tor. Jacob Appelbaum.


---

\textcolor{red}{[article : préhistoire de bitcoin]}

\textcolor{gray}{On entend souvent dire que Bitcoin a été créé par les crypto-anarchistes. Comme je l'ai dit, on ne sait pas si Satoshi Nakamoto adhérait pleinement à cette idéologie ou non. Néanmoins, Bitcoin s'inscrit dans la plus pure tradition crypto-anarchiste.}

\textcolor{gray}{La crypto-anarchie est la réalisation dans le cyber-espace d'une forme d'anarchie par l'usage de la cryptographie. Les crypto-anarchistes sont aussi appelés des cypherpunks, les deux termes étant des synonymes. Le terme cypherpunk provient de l'anglais cypher, chiffre, et punk, voyou, et est calqué sur le genre de la science-fiction cyberpunk. Mais ces gens-là n'ont souvent rien de "voyous", et sont des personnes bien éduquées qui comprennent comment les ordinateurs et Internet fonctionnent.}

\textcolor{gray}{Pour les crypto-anarchistes (ou les cypherpunks), l'objectif est d'assurer le respect de la vie privée par l'utilisation proactive de la cryptographie. Ils s'opposent à la surveillance, à la censure et à l'exploitation des données personnelles.}







\section{Les extropiens}
% L'extropianisme

Prédecesseur FM-2030 Fereidoun M. Esfandiary

Le mouvement extropien a été fondé par Max T. O'Connor (Max More) et Tom Bell (T. O. Morrow) en janvier 1988.

Magazine \emph{Extropy}, automne 1988, \url{https://github.com/Extropians/Extropy/blob/master/Extropy-01.pdf}. Principes extropiens (Max More) dans la sixième édition de la revue publiée durant l'été 1990\sendnote{\url{https://github.com/Extropians/Extropy/blob/master/ext6.pdf}}~:

\begin{enumerate}[(1)]
    \item L'expansion illimitée \textcolor{gray}{(Boundless Expansion)}
    \item L'auto-transformation\textcolor{gray}{(Self-Transformation)}
    \item L'optimisme dynamique \textcolor{gray}{(Dynamic Optimism)}
    \item La technologie intelligente \textcolor{gray}{(Intelligent Technology)}
\end{enumerate}

L'extropianisme est un transhumanisme.

\textcolor{gray}{WP:fr. Le transhumanisme est un mouvement culturel et intellectuel international prônant l'usage des sciences et des techniques afin d'améliorer la condition humaine par l'augmentation des capacités physiques et mentales des êtres humains et de supprimer le vieillissement et la mort}

L'extropie (par opposition à l'entropie) est «~l'extension de l'intelligence, de l'ordre fonctionnel, de la vitalité, de la capacité et de la volonté d'amélioration d'un système vivant ou organisationnel\sendnote{Max More, \eng{Principles of Extropy}, 2003.}~». % "the extent of a living or organizational system's intelligence, functional order, vitality, and capacity and drive for improvement"

\textcolor{red}{article amorçage premiers systèmes d'argent liquide numérique}

\textcolor{gray}{Les extropiens étaient des futuristes transhumanistes optimistes qui envisageaient l'évolution technologique comme un moyen de libération de l'individu. Ils avaient avait donc des centres intérêts en commun avec les cypherpunks (le mouvement extropien avait été fondé 4 ans aurapavant), et beaucoup de personnes faisaient partie des deux mouvements comme Timothy C. May, Nick Szabo ou encore Hal Finney. La liste de diffusion extropienne était une liste de distribution de courrier électronique privée, par laquelle les extropiens communiquaient sur Internet et pouvaient discuter de nombreux sujets.}



Les extropiens étaient d'accord avec les libertariens. L'ordre spontané (Hayek) a été intégré aux principes extropiens en 1992\sendnote{\url{https://github.com/Extropians/Extropy/blob/master/ext9.pdf}}. Les extropiens étaient notamment inspirés par l'ouvrage de David Friedman \eng{The Machinery of Freedom}, qui a connu un regain de popularité en 1989 avec la publication de la seconde édition, et qui décrivait comment pouvait s'organiser une société sans État.

L'extropianisme s'oppose donc frontalement au technocratisme, l'idée que l'humanité devrait être guidée par des experts.

Aspect prométhéen de Bitcoin (qui rappelle l'entreprise de Prométhée, qui cherchait à aider les hommes à devenir les égaux des dieux)~: Bitcoin est le feu dérobé aux dieux, Satoshi serait alors Prométhée. Technohubris.

Thèmes~:

\begin{itemize}
    \item La conquête spatiale (échelle de Kardachev)
    \item L'augmentation de l'homme
    \item La lutte contre la mort (cryogénisation)
    \item L'intelligence artificielle
    \item La réalité virtuelle
\end{itemize}

Chip Morningstar. A présenté Tim May à Phil Salin\sendnote{\url{https://www.youtube.com/watch?v=TdmpAy1hI8g}, \wtime{6:00}.}. Habitat (MMORPG), première utilisation du terme «~avatar~».





Beaucoup de cypherpunks étaient également extropiens~: Hal Finney (cryogénisé en 2014), Wei Dai, Nick Szabo.

Ralph Merkle, promoteur de la cryogénisation\sendnote{\url{https://www.ralphmerkle.com/merkleDir/cryptoCryo.html}}.

\section{La cybermonnaie}

Les paiements sur Internet.

Cartes de crédit, terme évoqué par Edward Bellamy dans \emph{Looking Backward} (1887). Dee Hock, \emph{Electronic Value Exchange}, VISA.

Milton Friedman, 1999~:

\begin{quote}
«~Je pense qu'Internet va devenir l'une des forces majeures qui va réduire le rôle de l'État. La seule chose qui manque, mais qui sera bientôt développée, c'est un argent liquide électronique fiable, une méthode qui permette de transférer des fonds de A à B sur Internet sans que A connaisse B ou que B connaisse A.\sendnote{Milton Friedman, \eng{Milton Friedman Full Interview on Anti-Trust and Tech}, 1999~: \url{https://www.youtube.com/watch?v=mlwxdyLnMXM}.}~»
\end{quote}

% "I think that the internet is going to be one of the major forces for reducing the role of government. The one thing that's missing but that will soon be developed, is a reliable e-cash, a method whereby on the internet you can transfer funds from A to B without A knowing B or B knowing A."

\begin{quote}
«~Depuis l'invention du télégraphe, le règlement des transactions financières se heurte à un problème~: comment faire des affaires à distance alors que le moyen le plus simple d'exécuter, de compenser et de régler une transaction est l'échange de certificats au porteur~?\sendnote{Robert Hettinga, \eng{Digital Bearer Settlement}, avril 1998~: \url{http://www.systemics.com/legal/digigold/discovery/postings/Geoecon.pdf}~; alt.~: \url{https://nakamotoinstitute.org/the-geodesic-market/#digital-bearer-settlement}.}~»
\end{quote}


\textcolor{red}{article préhistoire de bitcoin}

\textcolor{gray}{Ensuite, les jeux vidéos ont eux aussi contribué à installer l'idée de monnaie numérique dans les esprits. Lors des années 2000, on a retrouvé en effet des systèmes de monnaie virtuelle dans beaucoup de jeux en ligne massivement multijoueurs, comme les pièces de métaux précieux d'Everquest (1999), le dollar Linden de Second Life (2003) ou encore l'or de World of Warcraft (2004). Bien qu'elles aient un caractère virtuel, ces monnaies ont acquis une valeur tout à fait réelle et étaient échangées contre de l'argent traditionnel~: de cette manière, on a vu le prix de l'or de WoW (qui existe toujours) franchir celui du bolivar vénézuélien en 2017.}

---

La monnaie numérique faisait partie des techniques prévues par les cypherpunks pour conserver leur liberté et leur confidentialité\sendnote{«~Nous défendons notre confidentialité avec la cryptographie, avec les systèmes anonymes de transfert de courriels, avec les signatures numériques, et avec la monnaie électronique.~» -- Eric Hughes, \eng{RANTS: A Cypherpunk's Manifesto}, \wtime{17/03/1993 19:51:06 UTC}~: \url{https://cypherpunks.venona.com/date/1993/03/msg00392.html}.}. Ils ont donc cherché à développer une telle monnaie.



---

La monnaie faisait aussi partie du plan des extropiens.

Argent liquide numérique. Hal Finney, \eng{Protecting privacy with electronic cash}, Extropy 10, 1993~: eCash / Digicash, Chaum.

Extropy Magazine 15~: digital cash special.

Monnaie comme une réserve de valeur pour l'avenir.

\begin{quote}
«~Chaum recherchait l'argent numérique pour la vie privée, et May pour une forme particulière et étroite de liberté [...]. Les Extropiens le voulaient également pour ces raisons, mais aussi comme une incitation à la transformation utopique.\sendnote{Finn Brunton, \emph{Digital Cash}, 2019, p. 131 : "Chaum sought digital cash for privacy, and May for a particular, narrow form of liberty [...]. The Extropians wanted it for these reasons, too, but also as a spur to utopian transformation."}~»
\end{quote}

---

Hawthorne Exchange, 24 mars 1993~: \url{https://diyhpl.us/~bryan/irc/extropians/raided-mailing-list-archives/unzipped/disk-07/DIG30152}.

Digital gold, 24 août 1993~: \url{https://cypherpunks.venona.com/date/1993/08/msg00698.html}.

Mais ces idées ne se sont vraiement concrétisées qu'avec eCash.

---

\textbf{Hawthorne Exchange.}

\textcolor{red}{article eCash David Chaum}

\textcolor{gray}{Malgré son échec, le concept d'eCash a ouvert la voie à de multiples projets plus ou moins bien conçus. Plusieurs expérimentations ont pu ainsi voir le jour à la même époque. Tout d'abord, il y a eu la place de marché Hawthorne Exchange (HEx) créée en 1993 par les extropiens, un groupe de transhumanistes optimistes dans lequel participaient Hal Finney (très connu pour s'être fait cryogénisé après sa mort), Timothy May et Nick Szabo. Le Hawthorne Exchange permettait d'échanger des unités de réputation, ainsi qu'une devise native, le thorne, qui possédait une valeur non nulle. Le service a fermé lors de l'année 1994.}


\textcolor{red}{article amorçage premiers systèmes d'argent liquide numérique}

\textcolor{gray}{Le Hawthorne Exchange (couramment abrégé en HEx) a été lancé le 24 mars 1993 par un individu du nom de Brian Holt Hawthorne comme un marché de réputation pour les membres de la liste de diffusion. Le système se basait sur un serveur qui gérait les courriels de manière automatique. Chaque membre de la liste de diffusion pouvait s'incrire pour acquérir des parts liées à son identité, ainsi que des parts de la plateforme possédant le sigle boursier HEX. Chaque part pouvait ensuite être échangée sur le marché selon l'offre et la demande, ce qui permettait théoriquement d'évaluer la réputation des membres de la liste. Le principe de base était que si un membre considérait que quelqu'un avait contribué positivement à la liste de diffusion, il achetait des parts de cette personne, et que dans le cas inverse il revendait ces parts.}

\textcolor{gray}{L'unité native pour effectuer ces échanges était le Thorne, et avait pour symbole ð ou p. La quantité monétaire émise au tout début était d'un million de Thornes et était détenue par le serveur. La distribution initiale se faisait lors de l'inscription~: au moment de leur entrée dans le système, les participants recevait, en plus de leurs parts représentant leur réputation, 100 HEX chacun (les parts du système d'échange) qu'ils pouvaient vendre au serveur pour un prix de 100 Thornes pièce, et donc obtenir au moins 10 000 Thornes chacun.}

\textcolor{gray}{Le système était très expérimental et beaucoup de membres de la liste de diffusion étaient sceptiques (à raison) sur sa pérennité. Néanmoins, certains se sont tout de même inscrits comme Hal Finney (il était l'un des premiers à s'inscrire et possédait la part HFINN), Perry E. Metzger (P) ou Nick Szabo (N) qui disait essayer pour «~le plaisir du jeu~» malgré ses réticences.}

\textcolor{gray}{Après une période de développement plus longue que prévue, les échanges ont pu débuter le 28 juin 1993. Les opérations étaient rares mais elles avaient lieu. Voici à quoi ressemblaient le cours des différentes parts dans le rapport du 22 juillet~: [capture d'écran]}

\textcolor{gray}{Plusieurs problèmes ont été évoqués dès le début. Le premier était le manque de liquidité du marché. Un groupe restreint de personnes possédait la majeure partie des Thornes et ne les faisaient pas bouger, ce qui n'aidait pas les autres à s'en procurer. Des propositions ont été faites pour améliorer les choses. Ainsi, le 6 juillet, un membre de la liste nommé Derek Zahn proposait d'augmenter drastiquement la quantité de Thrones en circulation. Suite à cette proposition, Perry Metzger rétorquait que les gens oubliaient que «~la taille de la masse monétaire [n'avait] pas d'importance~» et que «~les valeurs [pouvaient] augmenter indéfiniment même avec une masse monétaire fixe~». Néanmoins, il fallait pour cela que l'unité soit rendu divisible, chose qu'a faite Brian Hawthorne le 16 juillet en ajoutant deux chiffres après la virgule dans la représentation des Thornes.}

\textcolor{gray}{Le deuxième problème était la valorisation du Thorne et des jetons de réputation. Certaines personnes montraient en effet un certain scepticisme à propos de la mise en route du système, à l'instar de Hal Finney qui déclarait le 27 juillet~:}

\textcolor{gray}{«~De nombreuses personnes ont observé que les parts Hex ont peu ou pas de valeur intrinsèque. Cela remet en question toute la prémisse de la place de marché, à savoir que les valeurs des parts sont censées représenter d'une manière ou d'une autre la réputation des gens. Mais il n'y a aucune raison pour que la valeur des parts corresponde de quelque manière que ce soit à la réputation des gens, si ce n'est le fait qu'on se dit qu'il devrait en être ainsi. Il s'agit d'une tentative de créer une prophétie auto-réalisatrice, où si tout le monde croit X, alors tout le monde agit comme si X était vrai, et cela rend X vrai. [...] Il est important de comprendre que les Thornes ne sont pas comme les dollars. À moins que les parts HeX ne puissent recevoir une base autre que le caprice de leurs propriétaires, le marché s'effondrera sûrement, car il n'y a rien pour le soutenir.~»}

\textcolor{gray}{Plusieurs personnes ont réagi à cette conception. Ainsi, Dave Krieger a répondu à Hal Finney que les dollars fonctionnaient déjà comme cela. Perry Metzger, lui, a été plus loin en invoquant la conception subjective de la valeur~:}

\textcolor{gray}{«~L'une des grandes avancées de la théorie économique autrichienne est la notion que toute valeur est complètement subjective - c'est tout simplement ce que les gens sont prêts à payer pour la chose valorisée. [...] Toute monnaie est psychologique.~»}

\textcolor{gray}{HEx n'a jamais réellement fonctionné en tant que système de réputation car il n'y avait pas de sens à évaluer la réputation comme cela, et l'activité était de toute manière trop timide. Néanmoins, le Thorne lui a commencé à être utilisé comme monnaie d'échange. Par conséquent, bien que Brian Hawthorne lui-même avait déclaré que HEx n'était pas «~un système d'argent liquide numérique~» et n'avait «~aucune prétention à l'être~», les personnes présentes sur la liste se sont mises naturellement à effectuer des échanges contre du Thorne.}

\textcolor{gray}{Le premier achat d'un service a eu lieu le 31 août 1993 lorsque Dave Krieger a proposé à John McPherson de lui donner 1000 Thornes s'il recopiait et publiait une présentation de Vernor Vinge réalisée par The San Diego Union-Tribune. John McPherson a accepté dans la soirée, heure de Californie, concluant ainsi l'échange.}

\textcolor{gray}{Nick Szabo vendait quelques services contre du Thorne, et avait été jusqu'à mettre au point son propre catalogue de textes en tous genres, appelé «~Nick's Catalog~».}

\textcolor{gray}{Quelques paris étaient réalisés sur la liste de diffusion. De son côté, Tim May mettait à disposition des dossiers d'informations contre des Thornes, dans le cadre de son projet de BlackNet.}

\textcolor{gray}{Au cours du temps, le Thorne a aussi acquis un prix en dollars. Brian Hawthorne vendait du Thorne à un prix de 0,01 \$ pièce. Cependant, la demande pour le Thorne était moins forte que cela et la plupart des gens acceptaient d'acheter du Thorne un prix de 0,001 \$. Tim May en particulier cherchait à se procurer plus de Thornes~: il a par exemple acheté 10 000 Thornes à Edgar W. Swank pour 10 \$ en liquide. Le but de Tim May était d'«~accumuler plus de Thornes~» dans l'espoir que le système persiste et que Brian Hawthorne n'ait pas «~l'intention de dévaloriser le Thorne en imprimant plus~», chose que ce dernier confirmera~:}

\textcolor{gray}{«~Je vais répéter ce que j'ai déclaré publiquement auparavant. Il y a exactement un million de Thornes en circulation. Je n'en imprimerai pas plus.~»}

\textcolor{gray}{Cet amorçage du Thorne a étonné certains membres de la liste, et ce d'autant plus que cette utilisation n'était pas la vocation initiale du système. Ainsi, David Murray expliquait le 28 septembre~:}

\textcolor{gray}{Quand HEX a débuté, je ne pensais pas que le thorn[e] pouvait spontanément acquérir de la valeur. Je n'en suis plus si sûr maintenant. Les thorn[e]s offrent un avantage distinct par rapport aux dollars ici sur la toile~: ils sont électroniques et échangeables électroniquement. Cette marge d'efficacité peut suffire à leur permettre d'acquérir de la valeur, avec un peu d'aide (spontanée).}

\textcolor{gray}{Cela a également inquiété Brian Hawthorne, qui voyait son système être utilisé comme monnaie, et qui ne voulait pas subir les poursuites étatiques que subissait à l'époque le créateur de PGP, Philip Zimmermann~:}

\textcolor{gray}{«~Avertissement officiel, de sorte à ce que je ne me retrouve pas dans la même situation que Phil Zimmermann~: Le Hawthorne Exchange est un marché de réputation, pas un marché d'actions, de matières premières, de devises ou d'obligations. Le Thorne est un jeton avec lequel échanger des réputations. Le Hawthorne Exchange décline toute responsabilité quant à l'utilisation de Thornes comme monnaie réelle par ses clients.~»}

\textcolor{gray}{Toutefois, cette expérience est lentement tombée dans l'oubli et l'activité a commencé à décliner vers la fin de l'année 1993. Le 21 janvier 1994, Brian Hawthorne a mis le Hawthorne Exchange en vente, n'ayant plus le temps de s'en occuper. Il avait en effet lancé la chose de manière plus ou moins ironique et ne s'attendait pas à ce que les gens la prennent autant au sérieux. La plateforme a été rachetée par Bill Garland, qui a déclaré par la suite que «~HEx [avait été mis] en sommeil et qu'il le resterait encore un peu~» (HEx is now dormant and will be for a little while yet). Le Hawthorne Exchange n'a jamais réapparu et par conséquent le Thorne a fini par perdre sa valeur.}

\section{Le logiciel libre}

\textcolor{green}{Répartir entre le flux du propos et le ch. 9}


---

Bitcoin, licence MIT d'abord attribuée à Satoshi Nakamoto, puis aux «~développeurs de Bitcoin~» et enfin aux «~développeurs de Bitcoin Core~».


\begin{quote}
«~\textcolor{blue}{Si vous mettez des choses sous GPL, je dois éviter de les utiliser. Rien contre la GPL en soi, mais Bitcoin est un projet sous licence MIT. Tout ce qui est sous GPL doit être clairement marqué comme tel.}\sendnote{Satoshi Nakamoto, \eng{Re: Make your "we accept Bitcoin" logo}, \wtime{24/02/2010 21:53:52 UTC}~: \url{https://bitcointalk.org/index.php?topic=45.msg507\#msg507}.}~»
\end{quote} % If you GPL stuff, I have to avoid using it.  Nothing against GPL per-se, but Bitcoin is an MIT license project.  Anything GPL please clearly mark it as such.


% Si vous mettez des choses sous GPL, je dois éviter de les utiliser. Rien contre la GPL en soi, mais Bitcoin est un projet sous licence MIT. Tout ce qui est sous GPL doit être clairement signalé comme tel.

Suggestion de passer en GPL\sendnote{\url{https://bitcointalk.org/index.php?topic=989.msg12070\#msg12070}}, réponse de Satoshi\sendnote{\url{https://bitcointalk.org/index.php?topic=989.msg12494\#msg12494}}.

% If the only library is closed source, then there's a project to make an open source one.
%
% If the only library is GPL, then there's a project to make a non-GPL one.
%
% If the best library is MIT, Boost, new-BSD or public domain, then we can stop re-writing it.
%
% I don't question that GPL is a good license for operating systems, especially since non-GPL code is allowed to interface with the OS.  For smaller projects, I think the fear of a closed-source takeover is overdone.


Bitcoin Core est une implémentation de nœud complet qui permet d'accéder au réseau pair-à-pair de Bitcoin, et de recevoir, d'envoyer et de vérifier pleinement les transactions et les blocs constituants la chaîne. Il s'agit d'un logiciel programmé principalement en C++ et compatible avec les systèmes d'exploitation Linux, Windows et macOS. Celui-ci peut être utilisé sous la forme d'un logiciel à interface graphique (bitcoin-qt), ainsi que d'un démon (bitcoind) qui s'exécute en arrière-plan avec lequel l'utilisateur peut interagir grâce à bitcoin-cli.

% Solidité du logiciel
Comme tous les programmes informatiques complexes, Bitcoin Core n'est pas exempt de faiblesses, ce qui au cours de son histoire s'est matérialisé par deux incidents majeurs : août 2010 et mars 2013.

C'est pour cela qu'il est crucial que le logiciel derrière Bitcoin soit bien maintenu, optimisé, amélioré. Bitcoin représente aujourd'hui près de 300 milliards de dollars et déplace des dizaines de milliards de dollars chaque jour, et par conséquent il serait désastreux qu'un dysfonctionnement majeur survienne

Pour assurer la sécurité du logiciel, il existe donc des dizaines de personnes, identifiées ou anonymes, qui s'attellent à scruter et à perfectionner le code, à temps plein ou à temps partiel. Puisque Bitcoin Core est un logiciel libre disponible en source ouverte sur Internet, n'importe qui peut consulter le code, vérifier qu'il est conforme au résultat attendu ou même proposer de le modifier pour l'améliorer~!

Cette ouverture, couplée à une dette technique limitée, donne à Bitcoin une sûreté plus grande que de nombreux systèmes informatiques. En effet, au vu des sommes en jeu, la récompense pour l'exploitation réussie d'une faille dans le code serait énorme, ce qui renforce la confiance qu'on peut avoir dans le logiciel au cours du temps. C'est ce qu'on appelle l'effet Lindy\sendnote{«~Chaque jour qui passe sans que Bitcoin ne s'effondre en raison de problèmes juridiques ou techniques apporte de nouvelles informations au marché. Cela augmente les chances de succès de Bitcoin et justifie un prix plus élevé.~» -- Hal Finney, \eng{Re: Bitcoin and the Efficient Market Hypothesis}, \wtime{04/06/2011 23:36:04 UTC}~: \url{https://bitcointalk.org/index.php?topic=11765.msg169026\#msg169026}.}. % Every day that goes by and Bitcoin hasn't collapsed due to legal or technical problems, that brings new information to the market. It increases the chance of Bitcoin's eventual success and justifies a higher price.

Le modèle du protocole initialement publié par Satoshi Nakamoto est ainsi privilégié, en vertu de l'avantage du précurseur (\eng{first-mover advantage}) et de l'effet Lindy. Par exemple, le bogue de multisignature (qui impose de donner une variable supplémentaire avec les signatures, généralement 0) n'a jamais été corrigé, de sorte que tous les protocoles semblables l'ont retranscrit. Bitcoin est en quelque sorte semblable à un rat d'égoût\sendnote{Andreas Antonopoulos, \eng{Bitcoin Security: Bubble Boy and the Sewer Rat}, 16 octobre 2015~: \url{https://www.youtube.com/watch?v=810aKcfM__Q}.}.

De plus, les failles dans le code sont, outre leur rareté, le plus souvent très subtiles, ce qui fait que ce sont les développeurs bienveillants qui les découvrent et qui les rapportent. On peut par exemple citer le bogue d'inflation trouvé et révélé en septembre 2018 par Awemany\sendnote{\url{https://medium.com/@awemany/600-microseconds-b70f87b0b2a6}.}, développeur pour Bitcoin Unlimited, ou la faille permettant des attaques par déni de service rapportée en juin 2018 par Braydon Fuller, développeur pour Bcoin, et révélée publiquement plus deux ans plus tard, en septembre 2020\sendnote{\url{https://journalducoin.com/bitcoin/bitcoin-bug-severe-revele-2-ans-corrige/}}.

\textcolor{gray}{Tout ceci fait que la sécurité du logiciel s'améliore au cours du temps, que les vulnérabilités sont détectées et maîtrisées et que, en presque douze ans d'existence, seules deux d'entre elles ont provoqué un incident majeur. Bitcoin ne repose donc pas sur des logiciels magiques qui fonctionneraient parfaitement bien, mais sur l'action des développeurs qui maintiennent des implémentations faillibles et sur l'aide des mécènes qui financent ce développement.}

\section{eCash et les expériences des cypherpunks}

% --- eCash ---

Le modèle eCash est un concept d'argent liquide numérique dont le fonctionnement repose sur des tiers, appelés banques (\eng{banks}) ou monnaieries (\eng{mints}), qui valident les billets numériques. En tant qu'argent liquide, les transferts sont anonymes.

% Ces tiers sont appelés des banques (\eng{Chaumian banks}) ou des monnaieries (\eng{Chaumian mints}).

La sécurité et la confidentialité du système repose sur le mécanisme de signature aveugle, introduit par Chaum dans un article de recherche publié en 1983\sendnote{David Chaum, \eng{Blind signatures for untraceable payments}, 1983~: \url{https://sceweb.sce.uhcl.edu/yang/teaching/csci5234WebSecurityFall2011/Chaum-blind-signatures.PDF}.}. Le fonctionnement de ce mécanisme est analogue à la signature d'un billet physique en papier carbone représentant une quantité précise de monnaie (dénomination).

% Réseau distribué
Le système peut comporter une seule ou plusieurs banques. Dans le second cas, les banques ont besoin d'entretenir un registre des billets dépensés, pour vérifier qu'un billet n'a pas été utilisé deux fois. Le système est chapeauté par une autorité centrale qui délivre des habilitations.

% Création
Voici comment fonctionne la création d'un billet~:

\begin{enumerate}
\item Alice crée un billet en papier carbone (en générant aléatoirement un très grand nombre $x$)~;
\item Alice place le billet dans une enveloppe fermée (en utilisant une fonction de commutation $c$ qu'elle seule connaît)~;
\item Alice envoie l'enveloppe contenant son billet à la banque et communique la dénomination souhaitée~;
\item La banque signe l'enveloppe en indiquant le nombre de satoshis le billet représente (elle a une clé privée pour chaque dénomination), ce qui a pour effet de signer le billet en papier carbone à l'intérieur~;
\item La banque renvoie l'enveloppe à Alice~;
\item Alice ouvre l'enveloppe pour récupérer son billet signé (en utilisant la fonction d'inversion $c'$)~;
\item Alice vérifie que la signature de la banque est valide (c'est-à-dire qu'elle correspond à la clé publique liée à la dénomination demandée).
\end{enumerate}

% Transfert
La transaction se fait en donnant le billet signé à quelqu'un d'autre. Ainsi, si Alice veut payer Bob pour un service rendu, elle lui transmet le billet. Puis, Bob vérifie qu'il a bien été signé par la banque. Ensuite, Bob envoie immédiatement le billet réceptionné à la banque afin qu'elle vérifie que le billet est encore valide en s'assurant qu'il n'est pas déjà sur la liste des billets utilisés (pas de double dépense). Enfin, Bob pourra recevoir un nouveau billet de la même dénomination ou choisir de retirer les fonds que représentent le billet (rédemption).

% Confidentialité
\textcolor{gray}{Il est important de remarquer qu'aucune des banques du système ne peut relier le paiement à l'identité d'Alice, pas même sa propre banque. Tout ce qu'un observateur extérieur peut savoir, c'est que la banque de Alice a signé le billet qui a servi à payer Bob. Cela fait donc de eCash un système de paiement confidentiel qui préserve la vie privée du payeur, d'où sa dénomination de liquide électronique.}

% --- Application de eCash ---

Le concept d'eCash a été mis en application au cours des années 1990, d'abord par les cypherpunks, puis par l'entreprise de David Chaum, DigiCash.

% Magic Money
Le protocole Magic Money a été présenté sur la liste de diffusion des cypherpunks le 4 février 1994 par un développeur anonyme qui se faisait appeler Pr0duct Cypher et qui utilisait PGP pour s'identifier. Magic Money permettait de créer sa monnaie en faisant tourner un serveur de courrier électronique qui servait de monnaierie eCash\sendnote{«~Magic Money est un système d'argent liquide numérique conçu pour être utilisé par courrier électronique. Le système est en ligne et intraçable. En ligne signifie que chaque transaction implique un échange avec un serveur, pour éviter les doubles dépenses. Intraçable signifie qu'il est impossible pour quiconque de retracer les transactions, de faire correspondre un retrait avec un dépôt, ou de faire correspondre deux pièces de quelque manière que ce soit.~» -- Pr0duct Cypher, \eng{Magic Money Digicash System}, \wtime{04/02/1994 20:44:27 UTC}~: \url{https://cypherpunks.venona.com/date/1994/02/msg00247.html}.}. Magic Money utilisait l'algorithme RSA et les signatures aveugles et ces deux techniques brevetées à l'époque, de sorte que son déploiement était \eng{de facto} illégal et devait se confiner à l'expérimentation. Cette annonce a été accueillie favorablement sur la liste, notamment par Hal Finney\sendnote{«~Wow~! De la bombe~! [...] Chapeau bas à Pr0duct Cypher~!~» -- Hal Finney, \eng{Re: Magic Money Digicash System}, \wtime{04/02/1994 21:58:18 UTC}~: \url{https://cypherpunks.venona.com/date/1994/02/msg00251.html}.}.

% Tacky Tokens et autres jetons fantaisistes
Le premier système basé sur Magic Money a été mis en ligne de Mike Duvos quelques semaines plus tard au travers des Tacky Tokens\sendnote{Mike Duvos, \eng{Fun With Magic Money}, \wtime{26/02/1994 00:51:40 UTC}~: \url{https://cypherpunks.venona.com/date/1994/02/msg01391.html}.}, dont les pièces étaient émises en dénominations de 1, 2, 5, 10, 20, 50, et 100 unités. Malgré des propositions, aucune transaction réelle n'était réalisée, ce qui a poussé Tim May à réagir\sendnote{Timothy C. May, \eng{Why Digital Cash is Not Being Used}, \wtime{03/05/1994 19:48:18 UTC}~: \url{https://cypherpunks.venona.com/date/1994/05/msg00155.html}.}. D'autres implémentations fantaisistes de Magic Money ont vu le jour par la suite, comme les GhostMarks, les DigiFrancs ou les NexusBucks, mais n'ont pas connu un plus grand succès.

«~Je n'ai pas vu de Tacky Token depuis des mois, bien qu'il y avait pas mal d'activité lorsque j'ai rendu mon serveur disponible au début.~» \sendnote{\url{https://cypherpunks.venona.com/date/1994/08/msg00695.html}}

% DigiCash
Le concept a été mis en pratique au travers de la société DigiCash Inc., fondée par David Chaum en 1989 et basée à Amsterdam aux Pays-Bas, qui avait pour mission de mettre en application les idées du cryptographe. Plusieurs cypherpunks ont travaillé pour l'entreprise comme Eric Hughes, Bryce Wilcox (le futur Zooko Wilcox-O'Hearn) et Nick Szabo. Après quelques années de développement, un prototype a été présenté en mai 1994 lors de la première conférence internationale sur le World Wide Web au CERN à Genève\sendnote{DigiCash, \eng{World's first electronic cash payment over computer networks}, 27 mai 1994~: \url{https://chaum.com/wp-content/uploads/2022/01/05-27-94-World_s-first-electronic-cash-payment-over-computer-networks.pdf}.}.

\textcolor{gray}{L'entreprise DigiCash Incorporated est fondée par David Chaum en 1989 à Amsterdam aux Pays-Bas, dans le but de développer le système d'argent liquide électronique eCash reprenant ses idées développées les années précédentes. Digicash est en avance sur son temps~: à l'époque, le web vient tout juste d'apparaître et n'est pas encore bien installé~; le commerce électronique, lui, est inexistant. L'entreprise est donc aux premières loges pour tenter de construire un marché qui n'existe pas encore (mais qui comme on le sait viendra à se développer). Ainsi, DigiCash ne se concentre pas seulement sur eCash, mais aussi sur l'environnement autour, notamment par le développement de cartes à puce et d'applications de point de vente.}


% Cyberbucks
DigiCash a ensuite réalisé un essai qui a débuté le 19 octobre de cette année, au travers de l'émission de cyberbucks. Bien que leur nom fasse référence à la monnaie étasunienne («~\eng{a buck}~»), ceux-ci n'étaient pas adossés au dollar et possédaient donc un prix flottant. Une distribution initiale de 100 cyberbucks par nouvel utilisateur a été effectuée afin d'aider l'amorçage du système. Les cypherpunks se sont appropriés la chose en effectuant des échanges réels~: récompense pour la résolution d'un problème, vente de t-shirts, vente de logiciels, et bien sûr le change avec le dollar.\sendnote{Jim Crawley, «~Electronic Cash~», \eng{The Computists' Weekly}, vol. 5, no. 25, 11 juillet 1995, \url{https://www.nzdl.org/cgi-bin/library?e=d-00000-00---off-0tcc--00-0----0-10-0---0---0direct-10---4-------0-1l--11-ro-50---20-preferences---10-0-1-00-0--4----0-0-11-10-0utfZz-8-00&a=d&cl=CL2.5&d=HASH0199d48acda6ba6861de2d9e.2}.}

divers commerçants acceptent les cyberbucks..

% Mark Twain Bank
Mais les cyberbucks n'étaient qu'une monnaie d'essai.

En octobre 1995, la Mark Twain Bank lançait sa propre version de eCash en partenariat avec DigiCash, et, contrairement à l'essai précédant, l'unité échangée était adossée au dollar étasunien. Bien que l'expérience des cyberbucks ne se soit pas arrêtée là, leur valeur s'est effondrée à cause de cette nouvelle\sendnote{«~Mark Twain est arrivée sur la marché avec de l'argent liquide numérique \emph{réel}, et les gens ont complètement cessé d'échanger les certificats bêta. Je ne me souviens même pas du dernier prix de règlement, mais il s'agissait de quelques centimes de dollars.~» -- Robert Hettinga, \eng{e\$: Interbank Digital Cash Clearing, Better Living through Walletware, Microintermediation, Net.Currencies and ECM}, 3 juin 1996}. % Last summer, when I got back from my New Orleans trip, I was up in Montana hanging out while my wife was at an educator's conference, and Lucky Green sends me e-mail about having just sold, for cash, some of the demo "cyberbuck" certificates that Digicash was issuing at the time. I commented about this to cypherpunks, one thing led to another, and the next thing I knew, Rich Lethin had started up a mailing list and set up a protocol for trading these beta-certificates for cash over that list. He named the list ecm. (Send "info ecm" in the body of a message to majordomo@ai.mit.edu, if you want to see what the fuss was all about.) It was used sporadically up to the time when, you guessed it, Mark Twain came on line with actual digital cash, and people stopped trading beta-certificates altogether. I can't even remember what the last settlement price was, but it was pennies on the dollar. Actually, now that I remember it, there was a period where the beta-certificates were traded for real Mark Twain ecash on someone's web-page and then announced on ecm, but things have pretty much gone moribund on ecm lately. I haven't seen a trade go across in many months.

\textcolor{gray}{DigiCash conclue notamment des partenariats avec différentes banques et s'inscrit donc dans le milieu financier traditionnel. En 1995, la Mark Twain Bank, basée à Saint-Louis aux États-Unis, rentre dans la danse. En 1996, la Deutsche Bank est la deuxième banque à s'impliquer. Puis, d'autres banques suivent comme le Crédit suisse, la banque australienne Advance Bank, la Norske Bank située en Norvège ou encore la banque autrichienne Bank Austria.}

% Chute de DigiCash
DigiCash était dirigée par David Chaum...

\textcolor{gray}{Néanmoins, malgré tous ces développements, David Chaum est suspicieux, perfectionniste, et souhaite garder le contrôle sur son entreprise et sur le système qu'elle gère. D'après certaines sources, il refuse des partenariats avec de grands acteurs comme ING et ABN AMRO (deux des trois plus grandes banques néerlandaises à l'époque), Visa (le processeur de paiement par carte bancaire) et Netscape (l'entreprise qui développe le navigateur web du même nom). Une rumeur prétend même que Microsoft aurait proposé 100 millions de dollars pour inclure eCash dans Windows 95, offre que Chaum aurait refusé. Finalement, il quitte le poste de PDG en 1996.}

\textcolor{gray}{DigiCash fait faillite en 1998 et les brevets d'eCash sont revendus en 2002. En 1999, Chaum explique les raisons de l'échec, à savoir la manque d'adoption dû à la difficulté d'utilisation~:}

\begin{quote}
«~Il était difficile d'amener assez de commerçants à l'accepter de manière à ce qu'assez de consommateurs l'utilisent, ou vice versa. À mesure que le web grandissait, le niveau moyen de sophistication des utilisateurs baissait. Il était difficile de leur expliquer l'importance de la confidentialité.~»
\end{quote}

\textcolor{gray}{Ainsi, eCash était trop focalisé sur la protection de la vie privée et pas assez sur la facilité de paiement, contrairement aux sociétés de carte de paiement comme Visa et Mastercard, qui profiteront de l'espace laissé vide. Ces entreprises finiront pas dominer largement le marché du commerce en ligne, aux dépens de la confidentialité des utilisateurs.}

\textcolor{gray}{Puis, en août 1994, est apparu CyberCash, un concurrent direct à eCash. En 1996, l'entreprise CyberCash a lancé CyberCoin, un système de micropaiements, qui a été abandonné en 1999. CyberCash a échoué pour les mêmes raisons que eCash, et a fait faillite en 2001~: les brevets ont été revendus à l'entreprise Verisign, qui ont par la suite été rachetés par PayPal.}


% eCash n'aurait pas pu tenir
On dit souvent qu'eCash est confidentiel. Mais cela repose sur une certaine bienveillance des banques. Or dans un monde où les États cherchent à tout prix à scruter les transferts financiers, notamment au nom de la lutte contre le blanchiment des capitaux et contre le financement du terrorisme, les banques du marché blanc ne peuvent pas s'en sortir. Bien que celui-ci ne permette pas de procéder à une surveillance directe, les banques peuvent requérir toute information nécessaire pour autoriser le transfert\sendnote{Voir la section sur l'importance de la confidentialité dans le chapitre~\ref{ch:censure}.}. Ainsi, on peut avoir un système eCash qui respecte pleinement les normes de surveillance, comme le suggère l'implémentation de Chaum pour une MNBC conceptualisée en 2021\sendnote{David Chaum, Christian Grothoff, Thomas Moser, \eng{How to Issue a Central Bank Digital Currency}, mars 2021~: \url{https://www.snb.ch/n/mmr/reference/working_paper_2021_03/source/working_paper_2021_03.n.pdf}.}.



Satoshi Nakamoto reconnaissait la faiblesse de eCash. Dans un courriel adressé à la liste de diffusion p2p-research en février 2009, il réagissait à la comparaison entre Bitcoin et eCash en disant~:

\begin{quote}
«~Bien sûr, la plus grande différence est l'absence de serveur central. C'était le talon d'Achille des systèmes chaumiens~; lorsque l'entreprise centrale fermait ses portes, la monnaie disparaissait.\sendnote{Satoshi Nakamoto, \eng{[p2p-research] Re: Bitcoin open source implementation of P2P currency}, \wtime{12/02/2009 19:01:24 UTC}~: \url{https://diyhpl.us/~bryan/irc/bitcoin-satoshi/p2presearch-again/p2pfoundation.net/backups/p2p_research-archives/2009-February.txt.gz}.}~»
\end{quote}

Satoshi écrivait aussi à Dustin Trammell en 2009~:

\begin{quote}
«~Vous savez, je pense qu'il y avait beaucoup plus de gens qui étaient intéressés [par la monnaie électronique] dans les années 90, mais après plus d'une décennie d'échecs de systèmes basés sur des tiers de confiance (Digicash,~etc.), ils voient cela comme une cause perdue. J'espère qu'ils sauront distinguer que c'est la première fois, à ma connaissance, que nous essayons un système qui n'est pas fondé sur la confiance.\sendnote{Satoshi Nakamoto, \eng{Re: Bitcoin v0.1 released}, \wtime{13/01/2009 07:55:20 UTC}~: \url{http://web.archive.org/web/20131204164149/http://www.dustintrammell.com/files/Satoshi_Nakamoto.zip}.}~»
\end{quote}

Toutefois, avant d'arriver à Bitcoin il a fallu du temps. Temps au cours duquel les cyherpunks ont tenté de mettre au point un tel système.

---

\textcolor{red}{article eCash David Chaum}

\textcolor{gray}{COMMENT FONCTIONNAIT ECASH ? eCash est un système de monnaie électronique qui fonctionne sur Internet, et qui exploite les idées développées par David Chaum dans les années 80. Plus qu'une monnaie électronique, il s'agit surtout d'argent liquide (cash) en garantissant un certain anonymat. Tel que l'a dit David Chaum en 1996~:}

\begin{quote}
«~eCash est une forme numérique d'argent liquide sur Internet, où l'argent liquide papier ne peut pas exister. [...] Comme le liquide, il offre aux consommateurs un réelle possibilité de cacher ce qu'ils achètent.~»
\end{quote}

\textcolor{gray}{Comme on l'a dit, eCash utilise un procédé de signature aveugle permettant de créer une monnaie électronique. Ce procédé permet à un signataire de signer quelque chose sans voir ce qu'il signe et sans pour autant qu'il signe n'importe quoi. Pour illustrer la chose, prenons une situation où Alice souhaite payer Bob. Pour ce faire, Alice doit passer par deux étapes~: la création d'un billet numérique (analogue à un billet physique en papier carbone) signé par une banque eCash~; et le paiement à Bob en transmettant ce billet.}

\textcolor{gray}{Première étape~: la création du billet. Tout d'abord, Alice choisit un nombre x au hasard, nombre qui correspond à notre billet. Puis elle utilise une fonction de commutation c que seule elle connaît, par exemple en multipliant x par un nombre premier très grand~: dans notre analogie, cela consiste à placer le billet en papier carbone dans une enveloppe fermée. Puisqu'Alice est la seule à connaître c, elle est également la seule à connaître la fonction inverse c-1, donc à pouvoir ouvrir l'enveloppe. Elle envoie l'enveloppe contenant son billet (c(x)) à sa banque, et cette dernière la signe, ce qui fait que le billet à l'intérieur est également signé. La banque envoie ensuite l'enveloppe signée (s(c(x))) à Alice et débite son compte du montant équivalent. Enfin, Alice ouvre l'enveloppe, sort le billet signé (s(x)) et vérifie que la signature de la banque est valide.}

\textcolor{gray}{Alice dispose donc d'un billet numérique, signé par sa banque et qu'elle peut conserver autant qu'elle le veut comme tout autre objet physique. Cependant, vient le moment où elle souhaite acheter quelque chose dans un magasin tenu par un autre individu, Bob.}

\textcolor{gray}{Seconde étape~: le paiement. D'abord, Alice initie le paiement vers Bob en lui donnant le billet signé (l'information s(x)), puis Bob vérifie que ce dernier a bien été signé par l'une des banques du système. Mais cela ne suffit pas~: Alice pourrait évidemment envoyer la même information à plusieurs personnes et réaliser ainsi une double dépense. C'est pourquoi Bob envoie le billet réceptionné immédiatement à sa banque afin qu'elle vérifie que le billet est encore valide. Pour ce faire, la banque vérifie la signature et s'assure que le billet n'est pas déjà sur la liste des billets utilisés, registre partagé par toutes les banques du système. Une fois ceci réalisé, le banque peut créditer le compte de Bob du montant équivalent à celui du billet, ou recréer automatiquement un autre billet si Bob le désire. Quoi qu'il en soit, Bob est assuré d'avoir été payé et peut donc donner son bien à Alice en échange.}



\textcolor{gray}{Enfin, notons que le système n'était pas du tout ouvert dans la réalité. Outre la sélection des banques participantes, le système eCash était breveté, ce qui faisait que personne ne pouvait le reproduire légalement. L'entreprise DigiCash était donc la seule à pouvoir décider de l'évolution du protocole.}








\section{b-money (1998)}

Wei Dai, b-money

\textcolor{red}{article b-money JDC}

\textcolor{gray}{Wei Dai est un cryptographe de renommée mondiale, notamment connu pour avoir participé activement au mouvement crypto-anarchiste durant les années 1990. Avec Nick Szabo et Hal Finney, il fait partie des personnes soupçonnées d'être Satoshi Nakamoto et d'avoir inventé Bitcoin.}

\textcolor{gray}{Comme tout cypherpunk qui se respecte, Wei Dai est très secret et contrôle méticuleusement les informations qu'il laisse passer sur Internet. C'est pour cela que nous ne disposons que de peu de détails personnels sur cet homme. Nous savons cependant qu'il est d'origine chinoise et qu'il a obtenu un diplôme de sciences informatiques à l'Université de Washington, dans l'État du même nom aux États-Unis. Curieuse coïncidence~: Nick Szabo a fréquenté la même université à quelques années d'intervalle, mais Wei Dai affirme qu'ils ne sont jamais croisés dans la vraie vie.}

\textcolor{gray}{Wei Dai commence sa carrière d'ingénieur en informatique au début des années 90. Il travaille en tant que développeur pour TerraSciences, avant d'être engagé par Microsoft en tant que cryptographe. Il y publie notamment deux brevets (US5724279A et US6081598A), tous deux assignés à l'entreprise.}

\textcolor{gray}{Parallèlement à son activité professionnelle chez Microsoft, il travaille sur des projets personnels dans le domaine de la cryptographie et de l'open source. On peut notamment citer Crypto++, une bibliothèque de fonctions cryptographiques en C++, qu'il développe en 1995 (et qu'il continuera de maintenir jusqu'à ce jour). Tout au long de sa carrière, ses contributions dans le domaine seront notables~: le 6 février 2002, il découvrira une vulnérabilité dans SSH2 qu'il signalera immédiatement, et en avril 2007, il développera avec Ted Krovetz l'algorithme VMAC, un modèle de code d'authentification de message basé sur le chiffrement par bloc.}

\textcolor{gray}{À côté de l'aspect purement académique de la cryptographie, Wei Dai s'intéresse aussi de près au mouvement cypherpunk. Dans la présentation de sa b-money en 1998, il dira~:}

\begin{quote}
«~Je suis fasciné par la crypto-anarchie de Tim May. Contrairement aux communautés traditionnellement associées au mot ‘anarchie', en crypto-anarchie le gouvernement n'est pas temporairement anéanti mais définitivement oublié et inutile. C'est une communauté où la menace de violence est impuissante parce que la violence est impossible, et où la violence est impossible parce que ses membres ne peuvent pas être reliés à leur vrai nom ou leur localisation géographique.\sendnote{Wei Dai, \eng{b-money}, \wtime{26/11/1998 23:33:49}~: \url{http://www.weidai.com/bmoney.txt}.}~»
\end{quote}

\textcolor{gray}{En 1994, il se met donc naturellement à participer sur la liste de diffusion cypherpunk de l'époque. Il y aborde des sujets variés comme la réputation numérique, la théorie des jeux, la vie privée et l'anonymat dans les systèmes de monnaie électronique, faisant preuve d'une vivacité d'esprit au-dessus de la moyenne. Il propose également des outils pour faire avancer l'idéal crypto-anarchiste, dont un système d'horodatage, un protocole de communication anonyme (nommé PipeNet), et un système de partage de fichiers sécurisé.}

\textcolor{gray}{En avril 1997, il développe un concept de crédit anonyme, concept qui fait réagir beaucoup de personnes, dont le célèbre Hal Finney, lui aussi présent sur la liste de diffusion. Quelques mois plus tard, l'idée de Dai est reprise par Nick Szabo dans sa formalisation des smart contracts.}

\textcolor{gray}{Par la suite, il s'éloignera du mouvement cypherpunk, mais ne quittera pas le réseau Internet pour autant, et s'intéressera à d'autres sujets comme le transhumanisme, l'extropianisme, l'intelligence artificielle, l'éthique et l'épistémologie. À partir de 2009, il passera beaucoup de temps sur le site web LessWrong, une plateforme de blogging axée sur la rationalité humaine.}

\textcolor{gray}{Comme nous l'avons dit dans l'introduction, les monnaies numériques basées sur des systèmes centralisés comme e-gold et Liberty Reserve n'ont jamais pu se développer correctement à cause de l'inévitable intervention étatique. C'est la raison pour laquelle Bitcoin se base sur un système pair-à-pair de nœuds entièrement décentralisé – pour répartir le risque et éviter de reposer sur un point de défaillance unique. En revanche, Bitcoin n'a pas été le premier système à avoir été conçu de la sorte~: b-money, décrit en 1998, le préfigurait.}

\textcolor{gray}{Wei Dai travaille sur son idée de monnaie électronique à partir de 1995. Interrogé par la journaliste Morgen Peck en 2012, il expliquera pourquoi il souhaitait mettre en place un tel système~:}

\begin{quote}
«~Ma motivation pour b-money était de rendre possible une économie en ligne qui soit purement volontaire, une économie qui ne puisse pas être taxée et réglementée par la menace de violence.~»
\end{quote}

\textcolor{gray}{Le 26 novembre 1998, Wei Dai annonce fortuitement, au détour d'un courriel sur la liste de diffusion cypherpunk, la présence d'un texte descriptif sur son site web~:}

\begin{quote}
«~Il y a également une description de b-money, un nouveau protocole d'échange monétaire et d'exécution des contrats pour les pseudonymes.~»
\end{quote}

\textcolor{gray}{Le texte est court (un peu plus de 1000 mots) mais riche conceptuellement. Comme pour Bitcoin, «~b-money~» désigne à la fois le protocole (l'ensemble des règles à respecter pour participer) et la monnaie en elle-même. Dans sa description du modèle, Wei Dai décrit deux versions du protocole légèrement différentes~: l'une est irréalisable mais simple, l'autre est plus complexe mais fait des hypothèses plus réalistes.}

\textcolor{gray}{b-money suppose l'existence d'un réseau intraçable où les participants sont identifiés par des pseudonymes numériques (c'est-à-dire des clés publiques) et où chaque message transactionnel est signé par l'envoyeur et chiffré pour le destinataire. Chaque participant maintient une base de données séparée qui recense combien d'unités de b-money possède chaque pseudonyme. Le transfert de monnaie a donc lieu comme dans Bitcoin, à la seule différence que le système est basé sur des comptes, plutôt que sur des pièces / UTXO.}

\textcolor{gray}{Dans la première version du protocole, la création monétaire peut être réalisée par tous les participants. Il suffit pour l'un d'entre eux de diffuser la solution d'un problème informatique connu et précédemment non résolu. Le nombre d'unités créées dépend alors du coût de cet effort exprimé par rapport à un panier standard de marchandises (pouvant inclure des métaux précieux par exemple). Il s'agit d'une sorte de preuve de travail permettant de garantir une certaine stabilité de la b-money~: lorsque son cours par rapport au panier de marchandises augmente, les acteurs économiques déploient plus de puissance de calcul pour inonder le marché~; à l'inverse lorsque son cours baisse, les acteurs économiques sont incités à utiliser moins de puissance de calcul et donc la création monétaire est ralentie. Un stablecoin, pourrait-on dire.}

\textcolor{gray}{Une caractéristique de b-money est également la possibilité de créer et d'exécuter des contrats directement sur le système, bien que le procédé soit rudimentaire. Dans un contrat, les parties impliquées sont contraintes de mettre en jeu une caution pour s'assurer que personne ne fasse défaut. De plus, elles désignent un arbitre qui a pour rôle d'intervenir s'il y a un litige. Chaque partie (y compris l'arbitre) diffuse sa signature, ce qui permet de créer un compte spécial, crédité par les cautions des parties.}

\textcolor{gray}{L'exécution du contrat peut alors se faire de deux manières. D'une part, la plupart du temps, elle se fait à l'amiable, avec ou sans l'intervention de l'arbitre~: les parties se mettent d'accord sur une répartition des fonds prenant en compte ce qui a été réalisé dans la réalité.}

\textcolor{gray}{D'autre part, si un accord n'est pas trouvé entre les parties, même avec l'aide de l'arbitre, la résolution se passe comme dans un procès. Chacun diffuse une répartition possible et les éléments (arguments, preuves) en faveur de cette répartition, et c'est alors le réseau entier qui tranche (quels comptes doivent être mis à jour et de quelle manière), la position de l'arbitre étant en théorie privilégiée.}

\textcolor{gray}{La seconde version du protocole diffère de la première par le fait que le registre de propriété n'est plus conservé par tout le monde, mais uniquement par un sous-ensemble de participants appelés serveurs. Les participants à une transaction doivent alors vérifier que leur transaction a bien été traitée en envoyant des requêtes à un échantillon aléatoire de serveurs.}

\textcolor{gray}{Puisqu'il est nécessaire de faire confiance aux serveurs dans une certaine mesure, un mécanisme économique rappelant la preuve d'enjeu est mis en place pour faire en sorte qu'ils restent honnêtes. Chaque serveur dépose un montant de b-money sur un compte spécial afin d'être pénalisé en cas de mauvaise conduite. De plus, il est contraint de publier régulièrement sa création de monnaie et son registre. De leur côté, les participants vérifient que leur solde est correct et que la somme des soldes des différents comptes n'excède pas le montant annoncé de monnaie créée, ce qui permet d'empêcher une inflation arbitraire.}

\textcolor{gray}{On reconnaît ainsi beaucoup des caractéristiques de Bitcoin dans ce modèle, bien qu'il soit loin d'être parfait et présente des défauts évidents. La première version du protocole est ainsi impossible à mettre en place à grande échelle, notamment parce qu'elle n'est pas tolérante aux pannes byzantines et qu'elle ne résout donc pas le problème de la double dépense. La seconde version semble beaucoup plus réaliste, mais centralise considérablement le système en un petit nombre de serveurs, le rendant ainsi plus vulnérable aux attaques et à la corruption. Enfin, l'idée que les contrats dépendent du consensus social (menant à la création de plusieurs b-moneys en cas de désaccord profond) semble aujourd'hui complètement farfelue.}

\textcolor{gray}{b-money attirera un intérêt certain parmi les cypherpunks, et en particulier celui d'Adam Back. Néanmoins, Wei Dai n'implémentera jamais le système, en particulier à cause de son éloignement progressif de la crypto-anarchie.}

\begin{quote}
«~Je n'ai pris aucune mesure pour coder b-money, dira-t-il. Cela a en partie été dû au fait que b-money n'était pas encore un concept complètement pratique, mais je n'ai pas continué à travailler sur ce concept parce que j'étais un peu désenchanté par la cryptoanarchie au moment où j'ai fini d'écrire b-money, et que je ne prévoyais pas qu'un système comme celui-ci, une fois mis en œuvre, pourrait attirer autant d'attention et d'utilisation en dehors d'un petit groupe de cypherpunks inconditionnels.~»
\end{quote}



\textcolor{gray}{Néanmoins, sa curiosité pour le projet est ravivée par un membre de LessWrong en février 2011, et Wei Dai se met à miner du bitcoin avec une carte graphique. Il minera comme cela des centaines de bitcoins, qu'il ne vendra jamais.}

\textcolor{gray}{En réalité, si l'on veut trouver une descendance de la b-money et de la vision de Wei Dai, il faut plutôt chercher du côté d'Ethereum. En effet, nombre de caractéristiques d'Ethereum correspondent à ce qu'aurait pu être b-money~: l'ether n'a pas de quantité limitée bien que son émission soit restreinte, des stablecoins indexés sur des monnaies fiat ou de l'or fleurissent sur la chaîne, le système est basé sur des comptes et il est prévu qu'il passe à une preuve d'enjeu telle que le proposait la deuxième version du protocole b-money. De plus, Ethereum est une sorte de fils spirituel de b-money puisque la plus petite unité du système Ethereum est nommée le wei (tout comme celle de Bitcoin est le satoshi), et que le premier stablecoin décentralisé de la chaîne, géré par Maker DAO, s'appelle le dai !}


\textcolor{gray}{Le 20 juillet 2010, au sein d'une discussion sur le forum Bitcointalk parlant de la potentielle suppression de l'article concernant Bitcoin par Wikipédia, Satoshi Nakamoto écrivait~:}

\begin{quote}
«~Bitcoin est une implémentation de la b-money proposée par Wei Dai sur la liste de diffusion Cypherpunks en 1998 et du Bitgold proposé par Nick Szabo.\sendnote{Satoshi Nakamoto, \eng{Re: They want to delete the Wikipedia article}, \wtime{20/07/2010 18:38:28} \url{https://bitcointalk.org/index.php?topic=342.msg4508\#msg4508}.}~»
\end{quote}

\textcolor{gray}{Cette phrase est restée gravée dans les esprits, à tel point que la b-money et le bit gold sont régulièrement cités comme des précurseurs du bitcoin. Cependant, comme on le verra, Satoshi Nakamoto n'avait pas connaissance de ces deux tentatives au moment de concevoir son système. En réalité, ce message servait à montrer comment Bitcoin s'inscrivait dans l'histoire des monnaies électroniques et pourquoi il devrait être considéré sérieusement par les contributeurs de Wikipédia.}

\section{Bit gold (1998)}

Nick Szabo, bit gold

\textcolor{red}{article bit gold JDC}

\textcolor{gray}{Ce système était censé gérer la création et les échanges d'une ressource virtuelle appelée aussi le bit gold. Contrairement à l'e-gold qui était garanti par de l'or physique, ou la b-money indexée en théorie sur un panier de marchandises, le bit gold ne devait être adossé à aucun autre bien, mais posséder une rareté infalsifiable, et constituer ainsi un or entièrement numérique. Nick Szabo avait donc, avant Bitcoin, prophétisé qu'une unité électronique pouvait obtenir de la valeur sans pour autant être liée à une autre marchandise au préalable.}

\textcolor{gray}{Tout comme b-money, bit gold n'a jamais été implémenté, mais constitue une concept assez intéressant pour que l'on s'attarde sur son fonctionnement et sur son créateur, Nick Szabo.}

\textcolor{gray}{Nicholas J. Szabo, est un informaticien, juriste et cryptographe américain connu pour ses travaux sur les contrats numériques (appelés aujourd'hui smart contracts) et sur la monnaie, ainsi que pour son implication dans le mouvement crypto-anarchiste des années 1990. Avec Wei Dai et Hal Finney, il fait partie des personnes soupçonnées d'être Satoshi Nakamoto et d'avoir inventé Bitcoin.}

\textcolor{gray}{Nick Szabo naît aux États-Unis vraisemblablement à la fin des années 60, d'un père hongrois ayant fui le régime soviétique après l'insurrection de Budapest de 1956, et d'une mère américaine. Il a une enfance studieuse, il apprend à lire très tôt grâce à sa mère, et s'initie à la programmation en Basic sur l'Apple II.}

\textcolor{gray}{À la fin des années 80, Szabo étudie à l'université de Washington dans le nord-ouest des États-Unis. Il y obtient un diplôme de sciences informatiques en 1989, à peu près à la même époque que Wei Dai. Il travaille ensuite pour des entreprises technologiques comme JPL, Cuesta Technology ou Arcot Systems. En particulier, il occupe un poste de consultant pendant 6 mois pour Digicash, l'entreprise de David Chaum qui développe et qui gère le système eCash.}

\textcolor{gray}{De son expérience à Digicash, il apprend les techniques et les problématiques liées aux monnaies numériques. Mais surtout il retiendra le rôle néfaste (et, finalement, fatal) des tiers de confiance.}

\textcolor{gray}{Dans les années 1990, il s'implique pleinement dans le courant cypherpunk, si bien qu'il déménage dans la région de la baie de San Francisco pour assister aux réunions en personne auxquelles participent régulièrement Timothy May et Hal Finney. Sur Internet, il contribue à différentes listes de diffusion crypto-anarchistes et mène notamment le mouvement d'opposition contre la puce Clipper, un cryptoprocesseur conçu par la NSA ayant pour but de faciliter l'écoute des communications. Comme beaucoup d'autres cypherpunks de cette époque, il s'intéresse aussi à la science-fiction et à l'extropianisme, un type optimiste de transhumanisme désireux d'améliorer radicalement la condition humaine à l'aide de la technologie.}

\textcolor{gray}{Nick Szabo est également connu pour être la personne qui a théorisé les smart contracts, ces programmes autonomes qui s'exécutent sans le besoin d'un tiers de confiance, et qui sont aujourd'hui mis en application grâce à Bitcoin et Ethereum. En 1994, il invente le terme smart contract qu'il définit comme «~un protocole de transaction informatisé qui exécute les termes d'un contrat~» dans son premier écrit public sur le sujet. En 1996, il publie l'article Smart Contracts: Building Blocks for Digital Markets dans le magazine Extropy, et en 1997, il finit de formaliser le concept dans son essai Formalizing and Securing Relationships on Public Networks.}

\textcolor{gray}{Son intérêt pour les contrats le pousse à retourner étudier dans les années 2000, afin d'appronfondir sa connaissance de la loi humaine et d'affiner sa compréhension. En 2006, il obtient un diplôme en droit (Juris Doctor) de l'université George-Washington à Washington D.C.}

\textcolor{gray}{Nick Szabo est donc une personne curieuse, éclectique, qui s'intéresse à beaucoup de domaines. Tout au long des années, il écrira de manière prolifique, d'abord sur sa page personnelle, puis sur son blog, Unenumerated, un endroit où «~la liste des sujets […] est si vaste et si variée qu'elle ne peut pas être énumérée~». Nick Szabo parlera ainsi (entre autres) de droit, d'informatique, d'économie, de politique, de biologie, et cette diversité fera qu'il sera considéré par certains comme un «~esprit universel~».}

\textcolor{gray}{Dès 1998, Nick Szabo possède une conception libérale du monde, et considère, à l'instar de nombreux auteurs libéraux comme John Locke, Frédéric Bastiat ou Lysander Spooner, que le droit commun provient de la nature de l'homme, et non pas du contexte politique. Unenumerated, le titre de son blog actuel, n'évoquera pas seulement la diversité des sujets abordés, mais fera également référence aux unenumerated rights, les droits naturels qui ne sont pas inscrits dans la loi positive, qui sont évoqués dans le Neuvième amendement de la Constitution des États-Unis.}

\textcolor{gray}{En 1994, peu après s'être impliqué dans le mouvement cypherpunk, Nick Szabo crée sa propre liste de diffusion privée appelée libtech-l. Celle-ci regroupe un certain nombre de personnes dont Wei Dai, Hal Finney, ainsi que les économistes Larry White et George Selgin. Le sujet de la monnaie et la théorie de la banque libre sont discutés passionnément, ce qui donnera l'idée à Szabo de concevoir un système de monnaie de réserve, au-dessus duquel serait construit un système de paiement.}

\textcolor{gray}{Au cours de l'année 1998, Szabo développe son idée de bit gold qu'il décrit sur cette liste de diffusion (sur laquelle Wei Dai décrira par ailleurs sa b-money). Sur son site personnel, il héberge à l'époque une ébauche de livre blanc intitulée Bit Gold: Towards Trust-Independent Digital Money, qui contient surtout les échanges qu'il a pu avoir avec Hal Finney. Bit gold sera finalement présenté au grand pubic en décembre 2005 dans un article publié sur Unenumerated.}

\textcolor{gray}{La logique derrière bit gold est la minimisation de la confiance. Cette dernière, en réduisant le rôle des tiers de confiance au minimum, permettrait de reproduire la cherté infalsifiable des métaux précieux dans le cyberespace.}

\textcolor{gray}{L'élément central du protocole est que la création monétaire se fait par preuve de travail~: les morceaux de bit gold sont ainsi créés grâce à la puissance de calcul des ordinateurs. Chaque solution est calculée à partir d'une autre, formant une chaîne de preuve de travail. Pour ce faire, bit gold n'utilise pas un algorithme basé sur l'inversion partielle d'une fonction de hachage comme Hashcash (et comme Bitcoin), mais ce que Szabo appelle une secure benchmark function. L'idée, vaguement définie par Szabo, est de mesurer précisément la difficulté et donc d'évaluer la valeur d'une solution.}

\textcolor{gray}{Le processus est le suivant~:}

\textcolor{gray}{
\begin{itemize}
\item[$\bullet$] Une information de référence publique c0 («~challenge bits~») est créée.
\item[$\bullet$] Un «~mineur de bit gold~» génère la preuve de travail s0 à partir de c0.
\item[$\bullet$] Cette preuve de travail est horodatée (t0) de manière sécurisée et distribuée grâce à de multiples services d'horodatage.
\item[$\bullet$] Le mineur ajoute cette preuve de travail et son horodatage au registre de propriété, qui est un registre distribué.
\item[$\bullet$] La preuve de travail devient la nouvelle information de référence c1, à partir de laquelle est calculée une nouvelle preuve de travail s1, etc.
bit gold chaîne de preuve de travail
\end{itemize}
}

\textcolor{gray}{Dans ce modèle, ce sont les preuves de travail horodatées qui représentent les morceaux de bit gold. Afin de vérifier la validité d'un morceau, n'importe qui peut consulter l'information de référence, la preuve de travail et l'horodatage qui sont des données publiques.}

\textcolor{gray}{Les échanges se font par le biais d'un registre public de titres de propriété décrit par Szabo dans l'article Secure Property Titles with Owner Authority. Les utilisateurs sont identifiés par leur clé publique et signent les transactions grâce à leur clé privée. Le registre est vérifié et maintenu par un réseau de serveurs appelé «~club de propriété~», serveurs qui se mettent d'accord sur qui possède quoi par l'algorithme de tolérance aux pannes byzantines appelé Byzantine Quorum System. Ainsi, n'importe qui peut se référer à ce registre pour connaître le propriétaire d'un morceau de bit gold.}

\textcolor{gray}{Bien évidemment, le système imaginé par Szabo possède quelques problèmes conceptuels. Tout d'abord, le caractère fixe des règles fait que les morceaux de bit gold produits ne sont pas fongibles, c'est-à-dire qu'ils ne peuvent pas être mélangés entre eux. En effet, avec l'augmentation de l'efficacité des machines ou l'avancement de la cryptanalyse, un morceau de bit gold produit à une époque peut ne pas avoir la même valeur qu'un morceau produit des années plus tard. C'est pourquoi Szabo imagine un modèle de marchés mis en place pour évaluer la valeur de chaque morceau de bit gold, l'idée étant que ces morceaux de bit gold seraient combinés et divisés pour former une réelle devise échangeable et fongible.}

\textcolor{gray}{Ensuite, outre le problème de la fongibilité, afin de situer les preuves de travail dans le temps, bit gold repose sur des services d'horodatage centralisés. Ces derniers, bien qu'ils soient censés être multiples et indépendants, forment des points de défaillance uniques.}

\textcolor{gray}{Enfin, pour son registre de propriété, bit gold repose sur un protocole de consensus classique qui requiert que les serveurs soient choisis à l'avance, connus par tous, et que 66~\% d'entre eux se comportent bien, c'est-à-dire qu'ils ne mentent pas sur les informations qu'ils relaient. Le système n'est ainsi pas très robuste et la seule solution (très imparfaite) qu'imagine Szabo en cas d'attaque est une migration des services honnêtes et qui conduirait à la formation d'un nouveau club de propriété – que les serveurs honnêtes copient le registre et forment un nouveau club de propriété. «~Si les règles sont violées par les votants qui gagnent, les perdants qui ont raison peuvent quitter le groupe et former un nouveau groupe, héritant des anciens titres~», explique-t-il. Une telle mesure serait aujourd'hui appelée un hard fork, et un exemple proche serait la création d'Ethereum Classic en 2016.}

\textcolor{gray}{À l'époque, bit gold est pensé comme un système de règlement permettant de gérer une monnaie de réserve rare, et au-dessus duquel serait construit un système de paiement, qui pourrait être, par exemple, un modèle proche de eCash. Nick Szabo réfléchira longtemps à comment mettre en application son idée, redemandant même de l'aide le 10 avril 2008, dans un commentaire sur Unenumerated~:}

\begin{quote}
«~[Bit gold] bénéficierait grandement d'une démonstration, d'un marché expérimental (avec par ex. un tiers de confiance pour se substituer à la sécurité complexe nécessaire au système réel). Quelqu'un veut m'aider à en programmer une ?~»
\end{quote}

\textcolor{gray}{Cependant, l'implémentation de bit gold ne verra jamais le jour à cause de ses défauts majeurs, et il faudra attendre quelques mois pour qu'un système qui les corrige se matérialise~: Bitcoin.}

\textcolor{gray}{On ne peut pas s'empêcher de constater que bit gold et Bitcoin sont très proches conceptuellement~: outre leurs noms respectifs, on retrouve en effet les éléments constitutifs de bit gold dans Bitcoin (une chaîne de preuve de travail, un service d'horodatage et un registre de propriété) à la seule différence que, dans ce dernier, ces trois élements sont fusionnés en un seul concept~: la chaîne de blocs. Bit gold ressemble donc beaucoup à Bitcoin, à tel point que certains s'imaginent que Bitcoin n'est qu'une mise en pratique directe de l'idée derrière bit gold, allant jusqu'à affirmer que Nick Szabo est Satoshi Nakamoto. Cependant, en apparence, tout indique que ce n'est pas le cas.}

\textcolor{gray}{Comme on l'a vu dans l'article précédent, quand il finit de concevoir Bitcoin lors de l'été 2008, Satoshi n'a pas connaissance de la b-money de Wei Dai et apprend son existence grâce à un échange avec Adam Back. Il contacte Wei Dai par courriel pour avoir des précisions sur l'année de publication de de la description de b-money et pour lui demander son avis sur ce qui deviendra Bitcoin. Dai suggérera que Satoshi ne connaissait pas le bit gold de Nick Szabo non plus~:}

\begin{quote}
«~Dans les premiers courriels que Satoshi m'a envoyés, il semblait ignorer les idées de Nick Szabo.~»
\end{quote}

\textcolor{gray}{Wei Dai dira aussi au passage que Nick Szabo n'est pas Satoshi~:}

\begin{quote}
«~[Satoshi] parle de comment bitcoin 'étend mes idées à un système complètement fonctionnel' et 'atteint presque tous les objectifs que je m'étais proposé de résoudre dans mon papier sur b-money'. Je ne vois pas pourquoi Nick, s'il était Satoshi, me dirait des choses comme cela en privé.~»
\end{quote}

\textcolor{gray}{Ainsi, le système imaginé par Nick Szabo n'est pas cité dans le livre blanc au moment de sa publication, le 31 octobre 2008. Satoshi apprend l'existence du système imaginé par Szabo plus tard, probablement grâce à l'intervention de Hal Finney sur la Cryptography Mailing List le 7 novembre, qui remarque la similarité du bitcoin avec le bit gold de Szabo~:}

\begin{quote}
«~Je crois aussi qu'il y a une valeur potentielle dans une forme de jeton infalsifiable dont le taux de production est prédictible et qui ne peut pas être influencé par des personnes corrompues. Ceci serait plus comparable à l'or qu'aux monnaies fiat. Nick Szabo a décrit il y a plusieurs années ce qu'il appelait "bit gold" et ceci serait une implémentation de ce concept.~»
\end{quote}

\textcolor{gray}{La référence à bit gold est finalement ajoutée sur bitcoin.org au début de l'année 2009, aux côtés de la b-money de Wei Dai.}

\textcolor{gray}{De son côté, Nick Szabo apprend visiblement l'existence de Bitcoin assez tôt et il l'évoque dans un article de blog le 7 mai 2009. Il est néanmoins plutôt sceptique au début, un tel projet n'étant pas si nouveau pour lui. Deux ans plus tard, soit à peu près au même moment où Satoshi Nakamoto disparaît complètement, Szabo écrit un texte sur Bitcoin, où il explique pourquoi le concept a mis autant de temps à arriver, et quels sont les problèmes que bit gold n'a pas su résoudre.}

\textcolor{gray}{À partir de la fin de l'année 2013, avec la montée du cours, le soupçon Satoshi se répand au sein du grand public, notamment à cause d'études linguistiques qui révéleraient une proximité entre les termes utilisés par Szabo et par Nakamoto. Nick Szabo dément fermement. Comme il le dira à Nathaniel Popper en mai 2015~:}

\begin{quote}
«~Comme je l'ai déclaré à plusieurs reprises auparavant, toute cette spéculation est flatteuse, mais incorrecte — je ne suis pas Satoshi.~»
\end{quote}

\textcolor{gray}{Bit gold constitue donc un prédécesseur majeur de Bitcoin~: une monnaie numérique, possédant une rareté infalsifiable garantie par la preuve de travail, dont la valeur émerge directement de cette caractéristique, et non d'un adossement à une autre marchandise. Bien que l'influence de bit gold sur Bitcoin n'ait pas été directe (à supposer que Wei Dai et Nick Szabo ne mentent pas), la création de Satoshi Nakamoto a émergé du même terreau technique et idéologique. De plus, comme on vient de le voir, les idées véhiculées par Nick Szabo ont joué un rôle dans l'évolution de Bitcoin, dont la fonction principale tend aujourd'hui à devenir un système de monnaie de réserve semblable à l'or numérique.}

\textcolor{gray}{À Nick Szabo, il a manqué la mise en œuvre. Concentré sur la théorie monétaire, qu'il a étudié en long, en large et en travers, il n'a jamais concrètement essayé de mettre son système en pratique. Cependant, d'autres individus ont tenté après lui d'implémenter des idées similaires~: c'est le cas de Hal Finney, et de son système de preuves de travail réutilisables (RPOW) mis au point en 2004.}


\section{RPOW (2004 -- 2005)}

mint = monnaierie, atelier de monnayage

Un autre système

Hal Finney, RPOW\sendnote{Ludovic Lars, Les RPOW de Hal Finney~: un dernier essai avant Bitcoin, 21/06/2020, \url{https://journalducoin.com/analyses/rpow-hal-finney-dernier-essai-avant-bitcoin/}}

% Preuve de travail (Hashcash)
La preuve de travail avait été décrite par Adam Back au travers de Hashcash (1997) dans le but de lutter contre le spam\sendnote{Pour une explication technique, voir la section sur la preuve de travail dans le chapitre~\ref{ch:confirmation}.}.  Celui-ci avait conscience, car il n'était pas transférable\sendnote{\url{https://cypherpunks.venona.com/date/1997/04/msg00822.html}}

% Adam Back : "hashcash is not directly transferable because to make it distributed, each service provider accepts payment only in cash created for them. You could perhaps setup a digicash style mint (with chaumian ecash) and have the bank only mint cash on receipt of hash collisions addressed to it. However this means you've got to trust the bank not to mint unlimited amounts of money for it's own use.
% So, perhaps you could have multiple banks and let reputation sort them out, if you could arrange the protocols so that it would be apparent if a bank was minting more cash than it had received hash collisions for.  (Say by publishing the collisions, and making it possible to publically verify the quantity of cash in circulation). But if you've got multiple banks then you've got to have an exchange mechanism.  The market could probably take care of this, setting exchange rates based on banks reputations.
% However it would be nicer to have something which required no trust and which had no posssibility of cheating rather than relying on reputation to sort them out."

% RPOW
L'idée est de reprendre ce projet en rendant les preuves de travail réutilisables. Les jetons de preuve de travail réutilisable sont gérés par un serveur qui se charge de les signer à l'aide du chiffrement RSA. Ils sont fabriqués à partir d'une preuve de travail (Hashcash) ou bien d'un jeton de RPOW précédent. Chaque jeton de RPOW ainsi obtenu se compose d'une valeur (définie comme une puissance de 2) et des données relatives à la signature du serveur. Un utilisateur peut par conséquent vérifier lui-même l'intégrité du jeton.

Pour s'assurer qu'une preuve de travail, simple ou réutilisable, ne serve pas plusieurs fois, le serveur maintient une base de données des preuves de travail utilisées qu'il consulte à chaque opérations.

Pour assurer la divisibilité de l'unité de compte, le système inclut aussi la possibilité de séparer une RPOW en plusieurs RPOW de degré moindre et de combiner plusieurs RPOW pour en obtenir une seule.

Le fonctionnement des RPOW est ainsi similaire aux billets numériques d'eCash~: les jetons de RPOW reposent sur les informations qu'ils contiennent et peuvent être transférés de personne à personne~; à chaque transfert, celui qui reçoit un jeton doit interagir avec le serveur pour recevoir un ou plusieurs nouveaux jetons, dont la valeur globale est égale à la valeur en entrée.


La quantité de jetons de RPOW dépend du rapport entre la difficulté de hachage et la valeur d'échange des jetons, ce qui assimile les jetons de RPOW à de l'or.

% Modèle de sécurité
Le système RPOW repose sur l'utilisation d'un serveur central qui s'occupe de détruire et de recréer les preuves de travail à chaque échange, notamment en vérifiant qu'une preuve de travail n'a pas fait l'objet d'une double dépense. Il faut donc faire confiance au serveur. Cependant, il a cherché à minimiser la confiance. ... l'utilisation du cryptoprocesseur IBM 4758 Secure Cryptographic Coprocessor, certifié FIPS-140 (Federal Information Processing Standards) de niveau 4. Il s'agit d'un cryptoprocesseur de haute sécurité qui résiste aux falsifications, et qui permet, grâce à un système d'authentification conçu par IBM, de vérifier les programmes qui sont exécutés sur la machine. De cette manière, un utilisateur externe peut s'assurer à tout instant que le serveur RPOW fait tourner le bon programme.\sendnote{«~Plus important encore, le système RPOW est conçu dans un but primordial~: empêcher quiconque, y compris le propriétaire du serveur RPOW et le développeur du logiciel RPOW, de violer les règles du système et de falsifier des jetons RPOW. Sans cette garantie contre la falsification, les jetons RPOW ne représenteraient pas de manière crédible le travail effectué pour les créer. Des jetons falsifiables ressembleraient davantage à du papier-monnaie qu'à du bit gold. Mon objectif avec ce projet était de donner vie à une concrétisation simple qui démontre la puissance du concept de bit gold. Pour ce faire, une résistance à la falsification est nécessaire, et c'est cet objectif qui a dominé tous les aspects de la conception.~» -- Hal Finney, \eng{RPOW Theory}, 15 août 2004~: \url{http://rpow.net/theory.html}~; archive~: \url{https://web.archive.org/web/20040815154951/http://rpow.net/theory.html}.} % "Most importantly, the RPOW system is architected with one overriding goal: to make it impossible for anyone, even the owner of the RPOW server, even the developer of the RPOW software, to be able to violate the system's rules and forge RPOW tokens. Without such a guarantee against forgeability, RPOW tokens would not credibly represent the work that was done to create them. Forgeable tokens would be more like paper money than bit gold. My goal in this project was to bring to life a simple realization which demonstrates the power of the bit gold concept. This requires resistance to forgeability, and this goal has dominated every part of the design."

Le modèle de sécurité repose encore sur une sorte de confiance, dans le sens où le ou les serveurs devaient être connus et pouvaient donc être arrêtés facilement.

%
Le système a été lancé le 15 août 2004. RPOW l'annonce sur la liste des cypherpunks\sendnote{Hal Finney, \eng{RPOW - Reusable Proofs of Work}, \wtime{15/08/2004 17:43:09 UTC}~: \url{https://lists.cpunks.org/pipermail/cypherpunks-legacy/2004-August/134945.html}.}, une annonce qui est retransmise sur la liste de Metzdowd.com par Robert Hettinga\sendnote{Robert Hettinga, \eng{FW: RPOW - Reusable Proofs of Work}, \wtime{15/08/2004 18:36:51 UTC}~: \url{https://www.metzdowd.com/pipermail/cryptography/2004-August/007362.html}.}. Le système a été mis à jour plusieurs fois pour améliorer son fonctionnement et est resté opérationnel pendant des mois. Hal Finney a présenté son système à la CodeCon 2005 organisée à San Francisco. Il faisait part des usages qu'il envisageait pour RPOW à savoir le transfert de la valeur, la régulation du courrier indésirable (dans la droite lignée de Hashcash), le commerce dans les jeux vidéos, le jeu d'argent en ligne comme le poker, et l'anti-parasitisme sur les protocoles de partage de fichiers comme BitTorrent\sendnote{Hal Finney, \eng{Reusable Proofs of Work}, 1\ier{} février 2005~: \url{http://rpow.net/slides/slide001.html}~; archive~: \url{https://web.archive.org/web/20050204193327/http://rpow.net/slides/slide001.html}.}.

% usage
L'usage réel a été anecdotique. Le système RPOW n'était pas parfait et ne pouvait pas, de toute évidence, devenir un système monétaire solide. Toutefois, il a eu le mérite de constituer une preuve de concept expérimentale, quatre ans avant Bitcoin.









---

\textcolor{gray}{Le célèbre Hal Finney reprend ensuite le concept pour l'appliquer par le biais des Reusable Proofs of Works (RPOW) en 2004, qui préfigurent la façon dont les bitcoins seront conçus plus tard. Sur son site web, Hal Finney dit : «~Miner et fabriquer des pièces d'or demande un effort et une dépense, ce qui les rend rares de manière inhérente. Les pièces d'or peuvent alors être transmises d'une personne à une autre, et chaque bénéficiaire peut vérifier l'authenticité de la frappe monétaire. De la même manière, la création de jetons de RPOW demande un certain degré d'effort et de dépense. Ils débutent tous avec une collision Hashcash [i.e. une résolution du problème mathématique] qui, au plus haut degré, prendra des heures voire des jours de calcul pour être créée. Les jetons de RPOW peuvent être validés et vérifiés à la réception en étant échangés contre un nouveau jeton de RPOW sur un serveur RPOW. Cela leur permet d'être transmis d'une personne à une autre tout comme des pièces.~»}

\textcolor{red}{article RPOW JDC}

\textcolor{gray}{Contrairement à ce qu'on peut parfois imaginer, le concept de preuve de travail utilisé pour donner de la rareté à un jeton numérique n'est pas nouveau. En effet, dès 1998, l'algorithme Hashcash se retrouvait dans l'idée de b-money de Wei Dai afin de maintenir la valeur du jeton autour de celle d'un panier de marchandises, et Nick Szabo imaginait une méthode similaire pour donner une rareté infalsifiable à son bit gold, une monnaie de réserve numérique.}

\textcolor{gray}{À côté de cela, on a également vu le concept apparaître dans des projets académiques. En 1996, Ronald Rivest et Adi Shamir (le R et le S du chiffrement RSA) imaginaient MicroMint, un système de micropaiement centralisé dont les coins devaient être impossibles à contrefaire. En 2003, le système Karma était décrit par Emin Gün Sirer et deux de ses étudiants : il s'agissait d'un modèle de devise permettant d'empêcher le parasitisme dans les protocoles de partage de fichiers tels que BitTorrent. Dans ce modèle, les utilisateurs devaient en principe disposer d'unités de karma pour télécharger, unités qu'ils obtenaient soit en partageant du contenu, soit en les fabriquant avec le processeur de leur ordinateur.}

\textcolor{gray}{Cependant, toutes ces idées sont restées à l'état théorique et n'ont jamais été mises en application de manière publique sur le réseau Internet. Il a fallu attendre 2004 pour voir un modèle être implémenté de manière correcte : le système des preuves de travail réutilisables, aussi appelé système RPOW, inventé par Hal Finney. Voyons donc qui est cet homme avant de nous intéresser au système qu'il a conçu et l'impact potentiel qu'il a eu sur Bitcoin.}





\section{Ripple Classique (2004 -- 2012)}

Ripple. Protocole développé par Ryan Fugger à partir de 2004 sous le nom de Ripplepay\sendnote{Mike Hearn, \eng{How did classical Ripple work?}, 18 août 2021~: \url{https://blog.plan99.net/how-did-classical-ripple-work-13fefe1870cf}.}. Basé sur un système de prêts~: «~un système de paiement où tout le monde est un banquier\sendnote{\url{https://web.archive.org/web/20110513085135/https://ripplepay.com/}}~».

Satoshi à Martien van Steenbergen~:

\begin{quote}
«~En ce qui concerne les systèmes de confiance, Ripple est unique en ce qu'il répartit la confiance plutôt que de la concentrer\sendnote{Satoshi Nakamoto, \eng{Re: Questions about BitCoin}, \wtime{13/02/2009 02:31:20}~: \url{https://www.bitcoin.com/satoshi-archive/emails/p2p-research/3/}.}~»
\end{quote}

Satoshi à Mike Hearn (qui s'était intéressé au projet quelques années auparavant)~:

\begin{quote}
«~Ripple est intéressant dans la mesure où c'est le seul autre système qui fait quelque chose de la confiance en dehors de la concentrer au sein d'un serveur central.\sendnote{Satoshi Nakamoto, \eng{Re: Questions about BitCoin}, \wtime{12/04/2009 22:44}~:, \url{https://plan99.net/\~mike/satoshi-emails/thread1.html}.}~»
\end{quote}

Le système s'appelle aujourd'hui Rumplepay\sendnote{\url{https://rumplepay.com/}}. Fiatjaf a proposé d'utiliser ce système en surcouche de Bitcoin\sendnote{\url{https://fiatjaf.com/rumple.html}}.

\section{Les systèmes distribués}

Paul Baran, \eng{On Distributed Communications Networks}

Un pair est une personne de même condition et de même rang dans un système organisationnel. En informatique, les systèmes dits pair-à-pair sont basés sur des réseaux où les ordinateurs possèdent tous le même niveau de privilège, par opposition au modèle client-serveur dans lequel les serveurs centraux répondent aux requêtes des clients. Ces ordinateurs sont appelés des nœuds.

Théorie des graphes : sommets (nœuds), arêtes (liens).

informatique distribuée % L'architecture distribuée ou l'informatique distribuée désigne un système d'information ou un réseau pour lequel l'ensemble des ressources disponibles ne se trouvent pas au même endroit ou sur la même machine.
% Distributed computing

La contribution de David Chaum \textcolor{gray}{au monde de l'informatique commence à la fin des années 1970, lorsqu'il réalise sa thèse de doctorat à l'université de Californie à Berkeley. Cette dernière, qui sera publiée en 1982 sous le titre "Computer Systems Established, Maintained and Trusted by Mutually Suspicious Groups", présente un système de coffres (vaults) multiples qui rappelle la façon dont la chaîne de blocs de Bitcoin est gérée. Comme le titre l'indique, le but est d'arriver à un consensus au sein d'un ensemble d'acteurs ne se faisant pas confiance, ce qui préfigure le problème des généraux byzantins énoncé en juillet 1982.}

\textcolor{red}{[article : préhistoire de bitcoin]}

\textcolor{gray}{Le développement du pair-à-pair (peer-to-peer en anglais) dans les années 2000 marque aussi une étape importante dans le chemin qui mène à Bitcoin. Plutôt qu'une infrastructure client-serveur, le pair-à-pair passe, comme son nom l'indique, par un réseau de pairs. Celui-ci s'est popularisé grâce au partage de fichiers sur Internet, partage qui enfreint la plupart du temps les droits d'auteur.} En 1999, Napster (pionnier à son époque) permettait de partager de la musique en pair-à-pair. Néanmoins, il reposait sur un serveur central pour référencer les fichiers, ce qui l'a contraint à fermer en 2001 sous la pression de la RIAA, l'association représentant l'industrie du disque aux États-Unis. Bittorrent a repris le flambeau en 2002 en constituant une alternative beaucoup plus fiable, si bien qu'il est toujours utilisé aujourd'hui. On a également vu d'autres protocoles émerger comme Gnutella (premier protocole purement pair-à-pair créé en 2000) ou eDonkey.

Lorsqu'il a conçu Bitcoin, Satoshi Nakamoto a repris ce modèle pair-à-pair pour sa robustesse~: en distribuant la gestion de la chaîne de blocs sur un réseau d'ordinateurs, le pair-à-pair permettait au système d'éviter de posséder un point de défaillance unique qui puisse être attaqué pour le tuer. Satoshi disait ainsi dans son courriel du 6 novembre~:

\begin{quote}
«~Les États sont bons pour couper les têtes des réseaux contrôlés de manière centralisée comme Napster, mais les réseaux purement pair-à-pair comme Gnutella et Tor semblent tenir le coup.\sendnote{Satoshi Nakamoto, \eng{Re: Bitcoin P2P e-cash paper}, \wtime{06/11/2008 20:15:40 UTC}, \url{https://www.metzdowd.com/pipermail/cryptography/2008-November/014823.html}.}~»
\end{quote}

\textbf{Partage des risques} : En se basant sur un système distribué, Bitcoin permet de répartir les risques pour ne pas subir ce qu'ont subi les systèmes centralisés.

\section{Vers Bitcoin}

\textcolor{gray}{Satoshi Nakamoto était-il un crypto-anarchiste ? On pourrait le supposer. En effet, premièrement il faut rappeler que "Satoshi Nakamoto" est un pseudonyme, et que, à ce jour, son identité légale reste inconnue : Satoshi a donc réussi à conserver son anonymat, pratique cypherpunk courante. Secondement, un autre détail important est que Satoshi a initialement publié le livre blanc sur la liste de diffusion de cryptographie de metzdowd.com qui était à l'époque fréquentée par des cypherpunks comme James A. Donald et Hal Finney. Ainsi, on peut imaginer que Satoshi connaissait bien le mouvement et qu'il était de facto un crypto-anarchiste, dans le sens où il a programmé un outil pour arriver à un résultat amenant plus de liberté individuelle.}



Par l'utilisation d'un pseudonyme et de certaines bonnes pratiques, il a réussi à conserver son anonymat. Il utilisait PGP\sendnote{\url{https://bitcointalk.org/index.php?topic=458.msg5772\#msg5772}}. Il utilisait Tor. Par sa prudence et son usage de Tor, de Namecheap, etc., il parviendra à rester anonyme malgré ses deux années d'activité.

Satoshi avait connaissance de eCash et des expériences de monnaie numérique. Il avait connaissance de Ripple comme le témoignent ses courriels à Mike Hearn. Il avait connaissance de Hashcash. Cependant, il ne connaissait pas b-money et bit gold, dont il a appris l'existence par la suite.

Enfin, et c'est peut-être l'élément le plus important, il a programmé un outil pour arriver à un résultat amenant plus de liberté individuelle, ce qui en fait de facto un cypherpunk.

\textcolor{gray}{Si Wei Dai a abandonné son idée en 1998, elle a laissé une grande trace dans l'histoire des monnaies numériques et elle est régulièrement citée comme l'un des ancêtres incomplets de Bitcoin. Cependant, son influence sur Bitcoin est à minimiser, car d'après Wei Dai lui-même, Satoshi Nakamoto n'avait pas connaissance de la b-money avant de créer Bitcoin. Comme il le dit dans un commentaire de février 2011~:}

\begin{quote}
«~Ce que je comprends c'est que le créateur de Bitcoin, qui se fait appeler Satoshi Nakamoto, n'a même pas lu mon article avant de réinventer l'idée lui-même. Il l'a appris par la suite et m'a crédité dans son papier. Donc ma connexion avec le projet est assez limitée.~»
\end{quote}

\textcolor{gray}{En effet, si la b-money est citée dans le livre blanc de Bitcoin, ce n'est pas parce que Satoshi s'en est inspiré, mais parce qu'Adam Back lui en a parlé. Tel que ce dernier l'écrit sur Bitcointalk en 2013~:}

\begin{quote}
«~Je crois que c'était grâce à moi si la référence à la b-money de Wei Dai a été ajoutée au papier de Bitcoin lorsque Satoshi m'a envoyé un courriel à propos d'Hashcash en 2008.~»
\end{quote}

\textcolor{gray}{Par conséquent, en août 2008, après être entré en communication avec Adam Back, Satoshi a écrit plusieurs courriels à Wei Dai. Quelques uns de ces courriels sont publiés au sein d'un article de Gwern Branwen et sont, si l'on en croit Wei Dai lui-même, authentiques.}

\textcolor{gray}{Le 22 août 2008, Satoshi envoie donc un premier courriel à Wei Dai pour lui dire qu'il «~se prépare à publier un document qui étend [ses] idées à un système complètement fonctionnel~» et pour lui demander «~l'année de publication de [sa] page sur la b-money~» afin de la citer dans son papier. Avec ceci, Satoshi lui joint le brouillon du livre blanc (ecash.pdf), dont le titre de l'époque est «~Electronic Cash Without a Trusted Third Party~» et qui ne mentionne pas encore le nom de Bitcoin, bien que le nom de domaine bitcoin.org soit déjà réservé. En réponse, Dai lui donne le lien vers le courriel d'annonce de la b-money, et lui promet de jeter «~un coup d'œil~» à son papier et «~qu'il l'informera s'il a des commentaires ou des questions~», chose qu'il ne fera pas.}

\textcolor{gray}{Quelques mois plus tard, en janvier 2009, après le lancement de Bitcoin, Satoshi envoie un autre courriel à Dai pour lui signaler que la première version du logiciel est disponible. Wei Dai ne répondra pas, négligeant la chose et étant passé à autre chose.}



\printendnotes